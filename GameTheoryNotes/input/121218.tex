%\documentclass[pt11,a4paper,twoside,reqno,openright]{paper}
%\usepackage[latin1]{inputenc}
%\usepackage{amsmath}
%\usepackage{amsfonts}
%\usepackage{amssymb}
%\usepackage[english]{babel}
%\usepackage{subfigure}
%\usepackage{graphicx}
%\usepackage{float}
%\usepackage{listings,lstautogobble}
%\usepackage[T1]{fontenc}

%\begin{document}

%\noindent {\huge Mechanism Design} 

%!TeX root = ./GameTheoryNotes.tex

\bigskip
\noindent \textbf{Slide 3}

\noindent It is the opposite of game theory, in the sense that in GT the rules 
of the game and the players' preferences are given and we look for the 
equilibria of the game. Here we look for a specific outcome and we design 
the rules of a game in such a way that this outcome is reached (sort of an 
inverse problem). Possible example is the prisoners' dilemma, where we can 
imagine the judge to design a way to make the prisoners confess their guilt. 

\noindent Other example: the professor designs the rules of the exam iot get 
the best possible outcome for him, knowing that students go for a Nash 
equilibrium. Rules are designed well if, for example, students get a good mark 
and they learn a lot of game theory.

\bigskip
\noindent \textbf{Slide 6}

\noindent In a direct mechanism, we need to invent a way to guarantee that you 
have no reasons to lie, otherwise nothing prevents you to do it.

\bigskip
\noindent \textbf{Slide 7}

\noindent Of course, to manipulate the game is not easy, since I may imagine 
that you are telling the truth and then I lie to maximize my utility, but 
since the situation for you is symmetric maybe we both lie and therefore we 
end up with a bad outcome.

\bigskip
\noindent \textbf{Slide 8}

\noindent A direct mechanism is truthful if to declare the truth (i.e. 
$s_i = v_i$)is a weakly dominant strategy.

\noindent We want the mechanism to have two properties: it must be social 
efficient and to tell the truth must be a dominant strategy for all the players.

\bigskip
\noindent \textbf{Slide 9}

\noindent $v_i(a)$ is $w_i$ is the object is assigned to myself, zero otherwise.

\noindent In this example, the maximum bidder is player1 and therefore he gain 
the object and he pays 80, which is the evaluation of the second highest bid, 
i.e. the bid of player2.

\noindent Player1 has no reason to offer more than 100; on the other hand, if he 
offers $100 > v^* > v_2$ he will gain the object and pay in any case $v_2$, thus 
he is indifferent. If he offers less than $v_2$, he will lose the object, then 
this is not a good strategy.

\bigskip
\noindent \textbf{Slide 10}

\noindent The maximum of the evaluations is exactly the maximum of the sum of 
the evaluations, since the evaluations are all zero but for the maximum one.

\bigskip
\noindent \textbf{Slide 12}

\noindent Idea of VCG mechanism: if I charge you with the difference in utility 
of the other players, for you to lie is not convenient. As another example, 
let's consider \textbf{slide17}.

\noindent The payment for a player is the damage he creates to the other 
players.

%\end{document}

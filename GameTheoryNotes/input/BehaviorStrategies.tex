%\documentclass[pt11,a4paper,twoside,reqno,openright]{paper}
%\usepackage[latin1]{inputenc}
%\usepackage{amsmath}
%\usepackage{amsfonts}
%\usepackage{amssymb}
%\usepackage[english]{babel}
%\usepackage{subfigure}
%\usepackage{graphicx}
%\usepackage{float}
%\usepackage{listings,lstautogobble}
%\usepackage[T1]{fontenc}

%\begin{document}

%!TeX root = ./GameTheoryNotes.tex

\noindent A \textbf{behavior strategy} for a player in an extensive form game 
is a map defined on the collection of his information sets and providing 
to each one of them a probability distribution over the actions of the player.

\noindent Idea of the different kind of strategies:
\begin{itemize}
	\item Pure strategies: they specify what a player would do in any possible 
	situation in which he has to make a decision.
	\item Mixed strategies give a probability distribution over the pure 
	strategies, which is necessary at any time we do not have a pure strategy 
	which is worth to be played despite of all the other posssibilities.
	\item Since a randomization over all the pure strategies can be difficult 
	when those strategies are many, it may be convenient to randomize over the 
	decision steps, which leads to the idea of behavior strategies.
\end{itemize}
Of course, we would like to have an equivalence between the different kind of 
strategies, i.e. we would like they lead to the same result.

\bigskip
\noindent \textbf{Equivalence of strategies:} if two strategies are equivalent, 
they must lead the players to the same results. Let $p(x; \sigma)$ be the 
probability that the vertex $x$ is reached under the strategy $\sigma$. 

\noindent Two strategies $\sigma_i$ and $b_i$ are equivalent for player $i$ if 
the probability to reach a certain vertex $x$ using $\sigma_i$ or $b_i$ is 
the same, once we have fixed every possible choice of the other players.

\noindent We denote by $\sigma_{-i}$ the choice of all the players but player 
$i$.  

\noindent The strategies mixed/behavior $\sigma_i$ and $b_i$ are equivalent for 
player $i$ if for every strategy $\sigma_{-i}$ (that can be formed by either 
behavior or mixed strategies) of the other players it holds
\[
	p(x; \sigma_i, \sigma_{-i}) = p(x; b_i, \sigma_{-i})
\]
Notice that this equivalence can be checked only on the leaves.

\bigskip
\noindent \textbf{Theorem:} let $s = (s_1,...,s_n)$ be a mixed strategies 
equilibrium profile. Let $b_i$ be a behavior strategy for player $i$ equivalent 
to $s_i$. Then for every player $j$ it is $u_j(s) = u_j(b)$, where 
$b = (b_i,s_{-i})$.
\noindent $\implies$ using equivalent strategies provides the players the same 
outcome. 

\bigskip
\noindent \textbf{Theorem:} let $\Gamma$ be a game in extensive form such that 
every vertex which is not a leaf has at least two children. The every behavior 
strategy has an equivalent mixed strategy iff each information set of player $i$ 
intersects every path starting at the root at most once.

\bigskip
\noindent Player $i$ has \textbf{perfect recall} if:
\begin{itemize}
	\item every path from the root to a leaf intersects every information set of 
	player $i$ at most once.
	\item every pair of paths from the root ending in the same information set 
	passes through the same information sets of $i$ and in the same order; 
	moreover, in each information set the two paths takes the same actions
\end{itemize}

\noindent \textbf{Remark:} a game has perfect recall if all the players have 
perfect recall.

\bigskip
\noindent \textbf{Theorem:} let $\Gamma$ be a game. If player $i$ has perfect 
recall, then every mixed strategy for player $i$ has an equivalent behavior 
strategy.

\bigskip
\noindent \textbf{Corollary:} in a game with perfect recall, the equilibria with 
mixed and behavior strategies are the same.

%\end{document}

%\documentclass[pt11,a4paper,twoside,reqno,openright]{paper}
%\usepackage[latin1]{inputenc}
%\usepackage{amsmath}
%\usepackage{amsfonts}
%\usepackage{amssymb}
%\usepackage[english]{babel}
%\usepackage{subfigure}
%\usepackage{graphicx}
%\usepackage{float}
%\usepackage{listings,lstautogobble}
%\usepackage[T1]{fontenc}

%\begin{document}

%\noindent {\huge Social Choice}

%!TeX root = ./GameTheoryNotes.tex

\bigskip
\noindent Social choice problems are mathematical models that looks for a way to aggregate 
individual preferences in social preferences. We want to derive a good social 
ranking given individual rankings (example: elections).

\noindent Problem: there is not an individual way to define fairness. A property 
that should be fullfilled is that if we agree on a list of preferences, there 
is no reason to change them (unanimity property). There other properties that 
we must agree on iot define a solution for the problem.

\noindent Remember: a social choice function only needs to determine a winner: 
we do not care about the second (example: elezione del Papa).

\bigskip
\noindent The primitive data for social choice problems are:
\begin{itemize}
	\item a set $A$, called the set of alternatives
	\item a set $N$, called the set of the agents
	\item a list $\succeq_i ~\forall i \in N$ of the preferences of agent $i$ over the set of alternatives.
\end{itemize}

\noindent \textbf{Preference profile:} $\succeq = (\succeq_1,...,\succeq_n)$.

\bigskip
\noindent \textbf{Total preorder / preference relation} over the elements of a finite set of alternatives $A$ is a subset $P \subset A \times A$ s.t. the following properties hold:
\begin{itemize}
	\item reflexivity: $\forall x \in A, (x,x) \in P$
	\item transitivity: $\forall x, y, z \in A, (x,y) \in P \wedge (y,z) \in P \implies (x,z) \in P$.
	\item completeness: $\forall x,y \in A$, either $(x,y) \in P$ or $(y,x) \in P$.
\end{itemize}

\bigskip
\noindent \textbf{Total strict preorder} over the elements of a finite set of alternatives $A$ is a subset $P \subset A \times A$ s.t. transitivity and completeness are fulfilled and $\nexists x \in A$ s.t. $(x,x) \in P$.

\bigskip 
\noindent We denote with $\mathcal{P}$ the set of all possibile preferences over $A$, while we denote with $\mathcal{SP}$ the set of all possible strict preferences over $A$.

\bigskip
\noindent \textbf{Social welfare function:} $f: \mathcal{SP}^n \rightarrow \mathcal{P}$, that receives as input a preference profile and returns a preference that represents the social ranking of the alternatives.

\bigskip
\noindent \textbf{Social choice function:} $f: \mathcal{SP}^n \rightarrow A$, that receives as input a preference profile and returns the social preferred alternative (the first in the hypothetical list of social preferences).

\section{Simple majority rule}

\noindent This rule can be applied only in the case of two alternatives and with an odd number of voters.

\noindent It assigns the first place to the alternative that receives the higher number of votes.

\bigskip
\noindent \textbf{Theorem:} when $|A| = 2$ and there are an odd number of voters, the simple majority rule is the unique social/welfare function that fulfills the following properties:
\begin{itemize}
	\item \textbf{anonimity:} the social welfare function $F$ is anonymous if for every permutation $\pi: N \rightarrow N$, $\pi(\succ_1,...,\succ_n) = (\succ_{\pi(1)},...,\succ_{\pi(n)})$ it holds 
	\[
		F(\pi(\succ_1,...\succ_n)) = F(\succ_1,...,\succ_n)
	\]
	
	\item \textbf{neutrality:} the social welfare function $F$ is neutral if for every permutation $\pi: A \rightarrow A$ s.t. $x~ \pi(\succ_i)~ y$ iff $\pi(x) \succ_i \pi(y)$ it holds
	\[
		F(\pi(\succ_1,...\succ_n)) = F(\succ_1,...\succ_n)
	\]
	
	\item \textbf{monotonicity:} the social choice function $f$ is monotonic if for every pair of preference profiles $\succ$ and $\supset$ s.t. $f(\succ)$ selects $x$ as a winner and $x \succ_i y$ implies $x \supset_i y ~\forall i \in N$, then $f(\supset)$ selects again $x$ as a winner.
\end{itemize}

\bigskip
\noindent \textbf{Comments:}

\noindent Anonimity means that I can have a permutation of the voters without 
changing the result, i.e. it is not important if I am labelled as player1 and 
you as player2 or viceversa. When anonimity is not satisfied? For example, 
when I have a tie and I have to decide that, for example, between two players 
with the same score the youngest is the winner (ex. concorsi per assegnazioni di 
150ore). Another example: between two players with the same score, I choose the 
woman because I have to fulfill "quote rosa".

\noindent Iot avoid to kill anonimity, people toss a coin.

\bigskip
\noindent Neutrality is the same as anonimity, but related on alternatives: you 
cannot distinguish two alternatives if all the cryteria are the same. 

\bigskip
\noindent Iot understand monotonicity, consider the following example: suppose that I have 3 alternatives:\\
x 	y 	z (player1)\\
y 	z	x (player2)\\
z	x 	y (player3)\\
and function $f$ selects $x$ as a result. Now suppose that I have another 
profile:\\
x 	z 	y (player1)\\
z 	x 	y (player2)\\
x 	z 	y (player3)\\
function $f$ must select again $x$ because $x$ is ranked better in this second 
case than in the previous one and, since he was winning in the first case, he 
cannot have a worse result than this.

\bigskip
\noindent \textbf{Remember:} simple majority rule means mutual comparison between 
alternatives. In case the alternatives are more than 2, you cannot get a ranking, but a circular argument! 
This means that simple majority rule cannot be extended to more than two 
alternatives.

\bigskip
\noindent \textbf{More sophisticated method:}

\noindent Instead of simply ranking the alternatives wrt the number of votes they get, which is not possible if $|A| > 2$, we can decide to assign a certain number of points to the alternatives, according to their ranking in the single preferences list.

\noindent For example, if there are four alternatives we can give 4 points to the first one, 3 to the second, etc and then sum up the total number of points got by each alternative iot determine the social preference list.

\noindent Since the numbers can be totally ordered, in this case I always get 
a ranking.

\section{Arrow's Theorem}

%\documentclass[pt11,a4paper,twoside,reqno,openright]{paper}
%\usepackage[latin1]{inputenc}
%\usepackage{amsmath}
%\usepackage{amsfonts}
%\usepackage{amssymb}
%\usepackage[english]{babel}
%\usepackage{subfigure}
%\usepackage{graphicx}
%\usepackage{float}
%\usepackage{listings,lstautogobble}
%\usepackage[T1]{fontenc}
%
%\begin{document}

\section{Arrow's Theorem}

\bigskip
\noindent \textbf{Notation:}
\begin{itemize}
	\item [-] $N$: set of agents
	\item [-] $O$: set of outcomes (i.e. set of candidates on which the agents 
	must express their preference), with $|O| \geq 3$.
	\item [-] $L$: set of all possible strict preference orderings over $O$.
	\item [-] $\succ_i$: a preference ordering for the agent $i$.
	\item [-] $W$: social welfare function.
	\item [-] $\succ_W$: preference ordering selected by the social welfare 
	function $W$.

	\noindent In particular, we denote by $\succ_{W_{([\succ'])}}$ the social 
	order selected by $W$ when it receives as input $[\succ']$.
\end{itemize}

\bigskip
\noindent \textbf{Pareto Efficiency (PE) - Unanimity property}

\noindent A social welfare function $W$ is Pareto efficient if for any $o_1, o_2 
\in O$, $\forall~i \in N$, $o_1 \succ_i o_2$ implies $o_1 \succ_W o_2$.

\bigskip
\noindent \textbf{Independence of irrelevant alternatives (IIA)}

\noindent $W$ is independent of irrelevant alternatives if, for any $o_1,o_2 
\in O$ and two preference profiles $[\succ']$ and $[\succ''] \in L^n$, 
$\forall~i \in N$ ($o_1 \succ' o_2$ if and only if $o_1 \succ'' o_2$) implies 
that ($o_1 \succ_{W_{[\succ']}} o_2$ if and only if $o_1 \succ_{W_{[\succ'']}} 
o_2$).

\noindent This means that the selected ordering between two outcomes depends 
only on their relative ordering given by the agents.

\noindent \textit{NOTE:} the fact that we assumed $|O| \geq 3$ depends on the 
fact that the IIA property is not meaningful in the case of only two outcomes.

\bigskip
\noindent \textbf{Non-dictatorship (ND)}

\noindent $W$ is non-dictatorial if it does not have a dictator, i.e. 
$\nexists ~i \in N$ s.t. $\forall ~o_1,o_2 \in O$, $o_1 \succ_i o_2 \implies 
o_1 \succ_W o_2$.

\bigskip
\noindent \textbf{Arrow's Theorem}

\noindent Any social welfare function $W$ that is Pareto efficient and 
independent of irrelevant alternatives is dicatorial.

\bigskip
\noindent \textbf{Proof:}

\noindent The proof of the Arrow's theorem is subdivided into four steps:
\begin{enumerate}
	\item \textit{If every voter puts an outcome $b$ at either the very top or 
	the very bottom of his preference list, then $b$ must be at either the very 
	top or the very bottom of $\succ_W$ as well}.

	\noindent Consider an arbitrary preference profile $[\succ]$ in which every 
	voter ranks some $b \in O$ either at the very bottom or at the very top and 
	assume by contradiction that the assumption above does not hold. Then, there 
	must exist some pair $a,c \in O$ s.t. $a \succ_W b \succ_W c$.

	\noindent Let's modify $[\succ]$ so that every voter moves $c$ just above 
	$a$ in his preference list and let's call this new preference profile 
	$[\succ']$. \textbf{IIA} entails that, if we want $a \succ_W b$ or $b 
	\succ_W c$ to change, we need the pairwise relationship between $a$ and $b$ 
	and/or the one between $b$ and $c$ to change as well. However, since $b$ 
	occupies an extremal position for all the voters, the relative position 
	between $a$ and $c$ can be changed without modifying the one of $b$. Thus, 
	in preference profile $[\succ']$ it is also the case that $a \succ_W b 
	\succ_W c$. By transitivity, we have that $a \succ_W c$. However, in 
	$[\succ']$ every voter ranks $a$ better than $c$ and then \textbf{PE} 
	requires $c \succ_W a$.

	\noindent $\implies$ CONTRADICTION.

	\item \textit{There is some voter $n^*$ who is \textbf{extremely pivotal}, 
	in the sense that, by changing his vote at some profile, he can move a 
	given output $b$ from the bottom to the top of the social ranking}.

	\noindent Consider a preference profile $[\succ]$ n which every voter ranks 
	$b$ at the very bottom of his preference list, while all the other 
	preferences are arbitrary. By \textbf{PE}, also $W$ must rank $b$ last.

	\noindent Now let voters from 1 to $n$ successively modify $[\succ]$ by 
	moving $b$ from the bottom to the top, preserving all the other relative 
	rankings. Denote as $n^*$ the first voter whose change causes the social 
	ranking of $b$ to change. This $n^*$ exists for sure: when all the voters 
	rank $b$ at the top of their preference list, by \textbf{PE} $b$ must be 
	ranked at the top also by the social welfare function.

	\noindent Let's denote by $[\succ^1]$ the preference profile just before 
	$n^*$ changes the position of $b$ and let's denote by $[\succ^2]$ the 
	preference profile just after this change. In $[\succ^1]$, $b$ is at the 
	bottom in $\succ_W$. In $[\succ^2]$ $b$ has changed its position in 
	$\succ_W$. Since also in $[\succ^2]$ all the voters rank $b$ either at the 
	top or at the bottom of their preference lists, by \textbf{step1} $b$ must 
	be ranked at the top in $\succ_W$.

	\noindent $\implies$ Q.E.D.

	\item \textit{$n^*$ is a dictator over any pair $a,c$ not involving $b$}

	\noindent First of all, we choose the element $a$ from the pair $a,c$ (of 
	course, we can choose it without loss of generality). Then, we construct a 
	new preference profile $[\succ^3]$ from $[\succ^2]$ by making the following 
	changes:
	\begin{itemize}
		\item We move $a$ to the top of $n^*$'s preference list, leaving all 
		the others unchanged, so that $a \succ_{n^*} b \succ_{n^*} c$.
		\item We arbitrarily rearrange he relative rankings of $a$ and $c$ for 
		all the agents but $n^*$, while leaving $b$ in its extremal position.
	\end{itemize}
	In $[\succ^1]$ we had $a \succ_W b$ since $b$ was at the bottom of 
	$\succ_W$. When we compare $[\succ^1]$ and $[\succ^3]$, relative rankings 
	for $a$ and $b$ are the same for all voters. Thus, by \textbf{IIA}, we must 
	have $a \succ_W b$ also in $[\succ^3]$.

	\noindent In $[\succ^2]$ we had $b \succ_W c$, since $b$ was at the very top 
	of the ranking. Relative rankings between $b$ and $c$ are the same in 
	$[\succ^2]$ and $[\succ^3]$, which entails that $b \succ_W c$ in $[\succ^3]$.

	\noindent By transitivity, it must be $a \succ_W c$ in $[\succ^3]$.

	\noindent Now we construct one more preference profile $[\succ^4]$ by 
	changing $[\succ^3]$ in two ways:
	\begin{itemize}
		\item We arbitrarily change the position of $b$ in each voter's ordering 
		while keeping all other relative preferences the same.
		\item We move $a$ to an arbitrary position in $n^*$'s preference list, 
		with the only constraint that $a \succ_{n^*} c$.
	\end{itemize}
	Observe that all voters but $n^*$ have entirely arbitrary references in 
	$[\succ^4]$, while $n^*$'s preferences are arbitrary but for the fact that 
	$a \succ_{n^*} c$.

	\noindent In $[\succ^3]$ and $[\succ^4]$ all agents have the same relative 
	preferences between $a$ and $c$. Thus, since $a \succ_W c$ in $[\succ^3]$ 
	and by \textbf{IIA}, $a \succ_W c$ in $[\succ^4]$. 

	\noindent $\implies$ we determined the social ranking between $a$ and $c$ 
	by assuming only that $a \succ_{n^*} c$ $\implies$ Q.E.D.

	\item \textit{$n^*$ is a dictator over all the pairs $a,b$}

	\noindent Consider an arbitrary outcome $c$. By the argument in 
	\textbf{step2}, there is a voter $n^{**}$ who is extremely pivotal for $c$. 
	By the argument in \textbf{step3}, $n^{**}$ is a dictator over all the pairs 
	$\alpha,\beta$ not involving $c$. Of course, the pair $a,b$ is one of such 
	pairs $\alpha,\beta$. We have already observed that $n^*$ is able to affect 
	$W$'s $a,b$ ranking (for example, when $n^*$ was able to change 
	$a \succ_{W_{([\succ^1])}} b$ in $b \succ_{W_{([\succ^2])}} a$). Hence $n^*$ and 
	$n^{**}$ must be the same agent.

	\noindent $\implies$ Q.E.D.
\end{enumerate}

%\end{document}


\section{Social choice functions}

\noindent Example: we are three people and we want to select a tv program to 
watch together. We are only interested to decide who is the winner, we are not 
interested in a full ranking (for the society: of course any player has his 
own full ranking). With more than two alternatives, this is relevant (with only 
two alternatives, once you have the winner you also have the full ranking): we 
are requiring less...are we in the same situation of the Arrow theorem or can 
we get something more? Of course we have to change some properties, for example 
the irrelevant alternative: since I do not have to write a full ranking for the 
society, this property is useless.

\bigskip
\noindent \textbf{Properties:}
\begin{itemize}
	\item A social choice function $f$ is \textbf{monotonic} if for every pair of preference profiles $\succ$ and $\supset$ s.t. $f(\succ)$ selects $x$ as the winner and $x ~\succ_i ~y$ implies $x \supset_i ~y ~\forall i \in N$, then $f(\supset)$ selects again $x$ as the winner.
	
	\item A social choice function $f$ is \textbf{dictatorial} if there exists $i \in N$ s.t. for every preference profile $\succ$, $f(\succ)$ is the most preferred alternative in $\succ_i$.
	
	\item A social choice function is \textbf{unanimous} if $a \succ_i b$ for every $i \in N$ and for every $b \in A$ implies $f(\succ) = a$.
	
	\item  A social choice function is \textbf{manipulable} if there exist a preference profile $\succ$, a voter $i \in N$ and $\supset_i$ s.t.
	\[
		f(\supset_i, \succ_{-i}) ~\succ_i~ f(\succ)
	\]
	
	\noindent Meaning:
	
	\noindent suppose that I have a ranking $\succ_1$ s.t.\\
	A \hspace{.6cm} C \hspace{.6cm} C\\
	B \hspace{.6cm} B \hspace{.6cm} B\\
	C \hspace{.6cm} A \hspace{.6cm} A\\
	and a function $f$ that selects C as the winner.
	
	\noindent Now suppose that I change my ranking in $\supset_1$ getting\\ 
	B \hspace{.6cm} C \hspace{.6cm} C\\
	A \hspace{.6cm} B \hspace{.6cm} B\\
	C \hspace{.6cm} A \hspace{.6cm} A\\
	and that now the function $f$ selects B as the winner. This means that the 
	function is manipulable: player1 can lie and submit the ranking $\supset$ iot 
	get a result (B) which is better for him in the true ranking $\succ$.
	
	\noindent Of course, this precise situation cannot happen in Life::real: it is 
	based on the assumption that player1 knows the ranking of other players and 
	they do not know his ranking (otherwise he could not lie). However, true 
	manipulation can occur when I have strong feelings on what you could choose.
\end{itemize}

\bigskip
\noindent \textbf{Theorem:} if $|A| \geq 3$, a social choice function satisfying monotonicity and unanimity is dictatorial.

\bigskip
\noindent \textbf{Gibbard-Satterwhite Theorem:} if $|A| \geq 3$, a social choice function satisfying unanimity and non-manipulability is dictatorial.

\bigskip
\noindent \textbf{Proof of the GS theorem:}

\noindent If I prove that non-manipulability $\implies$ monotonicity, then I can 
use the previous theorem (that required monotonicity and unanimity), since 
I have unanimity also here. I go by contradiction.

\noindent Suppose that there are $\succ$ and $\supset$ and two alternatives $a \neq b$ s.t. for every $c \in A$, $a \succ_i c \implies a \supset_i c ~\forall i \in N$, $f(\succ) = a$ and $f(\supset) = b$.

\noindent Suppose that the set $I = \{i \in N \text{s.t.} \succ_i \neq \supset_i\}$ is minimal.

\noindent \textbf{Claim:} $I$ is minimal $\implies$ if $i \in I$, then $f(\succ_i, \supset_{-i}) = a$. By only claiming that the change is minimal, I want 
to prove that, coming back to the previous situation, I get again $a$ as the 
winner.

\noindent Indeed, if this is not the case, then there exists $i \in N$ s.t. $f(\succ_i, \supset_{-i}) = d$ for some $d \neq a$. This means that the profile $\succ^* = (\succ_i, \supset_{-i})$ is s.t. for every $c \in A$, $a \succ_i c \implies a \succ^*_i c ~\forall i \in N$, $f(\succ) = a$ and $f(\succ^*) = d$ and $\succ^*$ has less differences from $\succ$ with respect to $\supset$, contradicting minimality.

\noindent Then I end up with $f(\supset) = b$ and $f(\succ_i, \supset_{-i}) = a$. 

\noindent By non manipulability, I get that\\
$f(\supset) = b \implies b \supset_i a ~\forall i \in N$\\
and\\ 
$f(\succ_i, \supset_{-i}) = a \implies a \succ_i b ~\forall i \in N$.

\noindent However, at the beginning we assumed that $a \succ_i b \implies a \supset_i b$ and therefore we obtained a contradiction.

%\end{document}

\documentclass[pt11,a4paper,twoside,reqno,openright]{paper}
\usepackage[latin1]{inputenc}
\usepackage{amsmath}
\usepackage{amsfonts}
\usepackage{amssymb}
\usepackage[english]{babel}
\usepackage{subfigure}
\usepackage{graphicx}
\usepackage{float}
\usepackage{listings,lstautogobble}
\usepackage[T1]{fontenc}

\begin{document}

\noindent {\huge Matching Problems}

\noindent Questions:
\begin{enumerate}
	\item Can you think of a situation in which a man can improve 
		the solution by lying on his preferences?

		\noindent \textbf{ANSWER:} 

		\noindent \textbf{Osservazioni:} in generale, nella vita 
		reale non conosciamo le funzioni di utilità dei giocatori, 
		quindi abbiamo bisogno di metterci in situazioni in cui 
		per i giocatori non conviene mentire. In particolare, la 
		situazione migliore possibile è che dire la verità sia una 
		strategia dominante; se questo non è possibile, almeno 
		dovrebbe essere un Nash equilibrium.

	\item How can we extend the definition of stable solution if there 
		are more men than women?

		\noindent \textbf{ANSWER:} 

		\noindent \textbf{Osservazione:} se stare con qualcuno è 
		sempre l'opzione preferita rispetto a stare soli, l'insieme 
		di uomini che rimane single in una soluzione stabile 
		rimarrà single in tutte le soluzioni stabili.

	\item Do you think is it possible to extend the Gale-Shapley algorithm 
		if one player can be matched with more people (f.i. with 
		college and students admission)? How?

		\noindent \textbf{ANSWER:}
\end{enumerate}

\noindent In generale, se abbiamo N uomini ed N donne, esistono N! possibili 
soluzioni, ma il numero di soluzioni stabili è molto più piccolo.

\end{document}

\documentclass[pt11,a4paper,twoside,reqno,openright]{paper}
\usepackage[latin1]{inputenc}
\usepackage{amsmath}
\usepackage{amsfonts}
\usepackage{amssymb}
\usepackage[english]{babel}
\usepackage{subfigure}
\usepackage{graphicx}
\usepackage{float}
\usepackage{listings,lstautogobble}
\usepackage[T1]{fontenc}

\begin{document}

\noindent Solution:
Exists $v,\bar{x}$ and $\bar{y}$ s.t. $f(x,\bar{y})\leq v ~\forall x$ and 
$f(\bar{x},y)\geq x ~\forall y$.

\noindent If player2 uses $\lambda_1...\lambda_n$ as mixed str., this 
generates the 
vector $\lambda_1p_1...\lambda_np_n$ (which is a vector inside the polytop 
since it is a convex combination of the vertexes).
But knowing the choice of player2, player1 looks at the possible outcomes 
and he chooses the best for him.

\noindent \section{Proof of Neumann theorem - slide 20,21}

{\huge God help us}

\noindent Suppose that $P_{ij}>0 \forall i,j$. This can be done without 
issues because in any case, if an entry is negative, I can decide to add 
a constant everywhere iot recover positive outcomes without changing the 
game (from the point of view of optimal strategies). Indeed, if I pass 
from $P_{ij}$ to $P_{ij}+10 ~\forall i,j$, I get that $f(x,y)=f(x,y)+10$.

\noindent Let $p_1...p_m$ the cols of $P$ and let C be the convex hull 
of the cols.

\noindent Let $Q_t = {x \in \mathbb{R}^n ~s.t.~ x_i\leq t ~\forall i}$.

\noindent $Q_0$ does not intersect C, but $Q_t$ intersects C for $t$ large 
enough.

\noindent Let $v = \sup{t: Q_t \bigcap C = \emptyset}$.

\noindent $C \bigcap int(Q_v) = \emptyset$.

\noindent $\exists x^* \neq 0$ and $b \in R$ s.t.
\[
	\langle \bar{x},u \rangle \leq b \leq \langle \bar{x},w\rangle 
	~\forall u\in Q_v~ and~ w \in C
\]

\noindent Remarks: $\bar{x_i} \geq 0 \forall i$.

\noindent Since the vector is not 0, I cannot have all the components 
equal to 0 and therefore I can assume, for example, $\sum_i{\bar{x_i}}=1$.

\noindent I want the sum to be 1 iot recover the optimal strategy for the 
first player.

\noindent Remember:
\[
	\langle \bar{x},u \rangle \leq b \leq \langle \bar{x},w\rangle 
\]

\noindent Now I claim $b=v$. 

\noindent $Q_{v+a} \bigcap C = \emptyset$ because they are separated 
by $b$. But this is impossible because $v+a > v$ and $v$ is defined 
as the supremum of the numbers for which $Q_t \bigcap C = \emptyset$.

\noindent Other remark: $Q_v \bigcap C \neq \emptyset$

\noindent This means that they must have at least one point in common. 
We can call $\bar{w} \in Q_v \bigcap C$. This means that $\bar{w}\in C$ 
and therefore $\exists \bar{y_1}...\bar{y_n} s.t. w=\sum{\bar{y_j}p_j}$ 
and $\bar{y_j}\geq0 \forall j$ and $\sum{\bar{y_j}}=1$.

\noindent I want to prove that v is the value of the game, i.e. the 
conservative values of the players are the same and $\bar{x},\bar{y}$ 
are optimal for player1 and player2, respectively.

\begin{enumerate}
	\item $f(x,y)\geq v \forall y \in \Sigma_m$. 
		I know that $w=\sum{y_jp_j} \in C$. 
	\item $f(x,\bar{y})\leq v \forall x \in \Sigma_n$. 
		\textbf{NOTE: in slide 21 we have $\bar{\beta}$ instead of 
		$\bar{y}$}.
\end{enumerate}

\noindent \textbf{Example}\\
Consider the matrix:\\
4	3	1\\
2	4	5\\

\noindent Possiamo rappresentare su un piano cartesiano i valori della 
matrice, dove la riga corrisponde alla x e la colonna alla y. Otteniamo 
un triangolo di vertici $p_1=(4,2), p_2=(3,4), p_3=(1,5)$.

\noindent La retta che passa per $p_1,p_3$ � $\frac{1}{2}x+\frac{1}{2}y=3$, 
quindi $\bar{x}=(\frac{1}{2},\frac{1}{2})$ e $v=3$.

\noindent Per trovare la soluzione ottima per il giocatore2, devo avere
\[3 = 4\bar{y} + (1-\bar{y})\]
\[3 = 2\bar{y} + 5(1-\bar{y})\]
ossia $\bar{y}=(\frac{2}{3},0,\frac{1}{3})$, il che significa che il giocatore 
2 gioca la prima colonna con probabilit� $\frac{2}{3}$, la seconda con 
probabilit� 0 etc. Guardando la matrice, verifichiamo che non pu�
effettivamente pagare pi� di $v=3$.

\noindent To understand the proof of vN theorem, I can solve games...
apparently! Actually, this works only if one player has at most two 
strategies. If both players have more than two str., it is not possible 
to represent the game on a cartesian plot: I cannot realize where the 
bisector intersects the polytop...in general, the intersection can be 
a line, a plane or whatever.

\noindent How can I solve a zero-sum-game in general?

\noindent Player1 must choose a mixed strategy with the characteristics 
listed in \textbf{slide22}. Notice that $p_{11}...p_{m1}$ is the 
first col of the payoff matrix $P$. This means that I want to get at least 
v by playing every column of the matrix.

\noindent Remember: the utiliy function is $x^TPy$ and player2 chooses in 
$\Sigma_m$. Given x, the minimum of the utility function is linear in y. 

\noindent \textbf{I always check your strategy in the vertexes of your 
simplex: maybe the minimum is on a face, but even in that case its value 
can be seen in the vertexes}.

\noindent The same holds for player2, but you consider the rows. 
Notice that w is unknown! The value is always an unknown of the problem.

\noindent We can see this as a linear programming problem (see 
\textbf{slide24}).


\section{Linear programming - slide 25 sgg.}

\noindent In (P) there are no equality constraints (that must be considered 
in a different way).

\noindent You find the dual problem by exchanging the role of unknowns and 
of constraints.

\noindent In the form of slide 26, notice that in (P) you don't have a 
constraint like $x\geq 0$, while you have it in (D), where you also have 
and equality constraint. This can be understood when you read the inequalities 
of slide 30: iot be able to write:
\[
	c^Tx \geq (A^Ty)^Tx
\]
you must know that $c \geq A^Ty$ and that $x \geq 0$, so that you're sure 
that the inequality does not change sign. In the second case, you don't 
know which is the sign of x and therefore you need $c = A^Ty$ so that 
you can write $c^Tx = (A^Ty)^Tx$. But then you need $y \geq 0$ iot be able 
to write:
\[
	y^TAx \geq y^Tb
\]
knowing that $Ax=b$, because you must be sure that the ineq. does not change 
its sign.

\noindent The second form of slide 26 is useful in \textbf{core problems}.

\noindent A problem is feasible if the set of vectors which satisfy the 
constraints admits at least one element (in this case, you also have a 
solution).

\noindent \textbf{Proof of slide 30}

\noindent I'm considering
\[
	min c^Tx
\]
\[
	Ax \geq b
\]
\[
	x \geq 0
\]
and
\[
	max b^Ty
\]
\[
	A^Ty \leq c
\]
\[
	y \geq 0
\]
but then I can say $c^Tx \geq b^Ty \forall x,y$ which are feasible.

\noindent
\textbf{Note:} the theorem of slide 31 comes from Weierstrass theorem: you 
only prescribe the set of constraints to be bounded iot to grant you have 
a solution (i.e. a minimum or a maximum, according to the fact that you 
are considering (P) or (D)).

\noindent \textbf{Complementarity conditions:} you lose the optimization 
part: you express the optimality of your solution using only 
inequalities and equalities. (CC) means that if, for example, 
$\bar{x_i} > 0 \forall i$ (strictly $>$), then the constraints of (D) 
must be active, which means that they must hold with $=$ instead of 
$\leq$ (and the same in the other case).

\bigskip
{\huge E' proprio Rinaldi, comunque}

\bigskip
\noindent \textbf{Slide 34:} what I usually write, for player1, is:
\[
	min 1^T_nx
\]
\[
	P^Tx \geq 1_m
\]
\[
	x \geq 0
\]
and, for player2:
\[
	max 1_my
\]
\[
	Py \geq 1_n
\]
\[	
	y \geq 0
\]

\noindent \textbf{Example:}\\
Consider the game:\\
(1,1)	(0,0)\\
(0,0) 	(1,1)\\
all cryteria we know cannot be applied, but we all agree that 
the solution should be the first or the fourth element of the matrix. 
The situation is similar in:\\
(3,4) 	(0,0)\\
(0,0)	(4,3)\\
where, however, the first and the fourth elements are not indifferent. This 
means that the two players are forced to coordinate if they do not want to 
end up in (0,0).

\noindent In a zero-sum-game players do not need to coordinate: at any 
time in which both are optimal, one player gains $v$ and the other pays $v$, 
which is the optimal solution for both. They naturally agree because the 
optimal solution for each player is optimal for both.

\section{Symmetric games}

\noindent A symmetric game is fair because both the players have the same 
options. The matrix is squared, which means that pure strategies and mixed 
str. coincide. Moreover, I expect the outcome to be 0 and I expect that 
the same stragegy is optimal for both the players.

\noindent \textbf{Remember:} if P is antisymmetric, then $P^T=-P$.

\noindent $v_1=\sup_x{\inf_y{f(x,y)}}$, therefore $f(x,x)=0$ means that 
$inf_y{f(x,x)}\leq 0$, which entails $v_1\leq 0$.

\noindent We are in a situation where $v=v_1=v_2$ and the fact that 
$v_1\leq 0$ and $v_2\geq 0$ entails that $v=0$.

\end{document}

%\documentclass[pt11,a4paper,twoside,reqno,openright]{paper}
%\usepackage[latin1]{inputenc}
%\usepackage{amsmath}
%\usepackage{amsfonts}
%\usepackage{amssymb}
%\usepackage[english]{babel}
%\usepackage{subfigure}
%\usepackage{graphicx}
%\usepackage{float}
%\usepackage{listings,lstautogobble}
%\usepackage[T1]{fontenc}

%\begin{document}

%!TeX root = ./GameTheoryNotes.tex

{\huge God help us}

\noindent Let $C$ be the convex hull of the vectors $p_1,...,p_m$ and define
\[
	Q_t = \{x \in \mathbb{R}^n: x_i \leq t\} ~\text{and}~ v = \sup\{t \geq 0: 
	Q_t \cap C = \emptyset\}
\]
Since $int(Q_v) \cap C = \emptyset$, $Q_v$ and $C$ ca be separated by an 
hyperplane, i.e. there are coefficients $\bar{x_1},...,\bar{x_n}$, not all 
zero, and $b \in \mathbb{R}$ s.t., for all $u = (u_1,...,u_n) \in Q_v$ and 
for all $w = (w_1,...,w_n) \in C$,
\[
	\sum_{i=1}^n{\bar{x_i}u_i} = \langle \bar{x},u \rangle \leq 
	b \leq \sum_{i=1}^n{\bar{x_i}w_i} = \langle \bar{x},w \rangle
\]
It holds:
\begin{enumerate}
	\item $\bar{x_i} \geq 0 ~\forall i$ and, since they cannot be zero, we can 
	assume $\sum{\bar{x_i}} = 1$ (I will need this in order to recover the 
	optimal strategy for the first player).

	\item $b = v$. Indeed, first of all, since $\bar{v} = (v,...,v) \in Q_v$, 
	from $\langle \bar{x},\bar{v} \rangle = v$ we get $b \geq v$.

	\noindent Now suppose $b > v$ and take $a > 0$ so small that $b > v+a$. 
	Then $\sup\{\sum_{i=1}^n{\bar{x_i}u_i}: u \in Q_{v+a}\} < b$ and this 
	implies $Q_{v+a} \cap C = \emptyset$ which is in contradiction with the 
	definition of $v$.

	\item $Q_v \cap C \neq \emptyset$. Indeed, let $w \in Q_v \cap C$. Then 
	$\bar{w} = \sum_{j=1}^m{\bar{y_j}p_j}$ for some $\bar{y} = (\bar{y_1},...,
	\bar{y_m}) \in \Sigma_m$. Observe that, since $\bar{w} \in Q_v$, then 
	$\bar{w_i} \leq v ~\forall i$.
\end{enumerate}

\noindent Now we can prove that $\bar{x}$ is optimal for the first player, that 
$\bar{y}$ is optimal for the second player and that $v$ is the value of the 
game.

\noindent First of all, let $y \in \Sigma_m$ be a mixed strategy for the second 
player and let $w = \sum_{j=1}^m{y_jp_j} \in C$ (thus $w_i = \sum_{j=1}^m
{y_jp_{ij}}$). Thus
\[
	f(\bar{x},y) = \sum_{i,j}{\bar{x}y_jp_{ij}} = \sum_{i=1}^n{\bar{x_i}w_i} 
	\geq v
\]
This shows that, by playing $\bar{x}$, the first player is able to guarantee 
himself at least $v$.

\noindent Consider now $\sum_{j=1}^m{\bar{y_j}p_j} = \bar{w} \in Q_v \cap C$ 
as before. For every mixed strategy $x \in \Sigma_n$ of the first player, we 
get
\[
	f(x,\bar{y}) = \sum_{ij}{x_i\bar{y_j}p_{ij}} = \sum_i{x_i\bar{w_i}} \leq 
	v\sum_i{x_i} = v
\]
which means that the second player is able, by playing $\bar{y}$ to pay no 
more than $v$.

\bigskip
\noindent \textbf{Example}\\
Consider the matrix:\\
4	3	1\\
2	4	5\\

\noindent Possiamo rappresentare su un piano cartesiano i valori della 
matrice, dove la riga corrisponde alla x e la colonna alla y. Otteniamo 
un triangolo di vertici $p_1=(4,2), p_2=(3,4), p_3=(1,5)$.

\noindent La retta che passa per $p_1,p_3$ � $\frac{1}{2}x+\frac{1}{2}y=3$, 
quindi $\bar{x}=(\frac{1}{2},\frac{1}{2})$ e $v=3$.

\noindent Per trovare la soluzione ottima per il giocatore2, devo avere
\[3 = 4\bar{y} + (1-\bar{y})\]
\[3 = 2\bar{y} + 5(1-\bar{y})\]
ossia $\bar{y}=(\frac{2}{3},0,\frac{1}{3})$, il che significa che il giocatore 
2 gioca la prima colonna con probabilit� $\frac{2}{3}$, la seconda con 
probabilit� 0 etc. Guardando la matrice, verifichiamo che non pu�
effettivamente pagare pi� di $v=3$.

\bigskip
\noindent To understand the proof of vN theorem, I can solve games...
apparently! Actually, this works only if one player has at most two 
strategies. If both players have more than two str., it is not possible 
to represent the game on a cartesian plot: I cannot realize where the 
bisector intersects the polytope...in general, the intersection can be 
a line, a plane or whatever.

\noindent How can I solve a zero-sum-game in general?

\noindent Player1 must choose a mixed strategy, i.e. a probability distribution 
$x = (x_1,...,x_n) \in \Sigma_n$, s.t.\\
$x_1p_{11} + ... + x_np_{n1} \geq v$\\
...\\
$x_1p_{1j} + ... + x_np_{nj} \geq v$\\
...\\
$x_1p_{1m} + ... + x_np_{nm} \geq v$\\
where $v$ must be as large as possible.

\noindent Notice that $[p_{11},...,p_{m1}]$ is the 
first col of the payoff matrix $P$. This means that I want to get at least 
$v$ by playing every column of the matrix.

\noindent Remember: the utiliy function is $x^TPy$ and player2 chooses in 
$\Sigma_m$. Given x, the minimum of the utility function is linear in y. 

\noindent \textbf{I always check your strategy in the vertices of your 
simplex: maybe the minimum is on a face, but even in that case its value 
can be seen in the vertices}.

\noindent The same holds for player2, but you consider the rows:\\
$y_1p_{11} + ... + y_mp_{1n} \leq w$\\
...\\
$y_1p_{i1} + ... + y_mp_{jm} \leq w$\\
...\\
$y_1p_{n1} + ... + y_mp_{nm} \leq w$\\
where $w$ must be as small as possible.


\noindent Notice that $w$ is unknown! The value is always an unknown of the 
problem.

\noindent We can see this as a linear programming problem: for player1,
\begin{equation*}
    \begin{cases}
        \max_{x,v} v:\\
        P^Tx \geq v1_m\\
        x \geq 0\\
        \langle 1,x \rangle = 1
    \end{cases}
\end{equation*}
while for player2
\begin{equation*}
    \begin{cases}
        \min_{y,w} w:\\
        Py \leq w1_n\\
        y \geq 0\\
        \langle 1,y \rangle = 1
    \end{cases}
\end{equation*}
where both problems are feasible.


\section{Linear Programming}

\noindent \textbf{Definition:} two linear problems (P) and 
(D) are said to be \textbf{in duality} if they are of the form:
\begin{equation}
	\tag{P1}
    \begin{cases}
        \min{c^Tx}\\
        Ax \geq b\\
        x \geq 0
    \end{cases}
    \label{primal1}
\end{equation}
\begin{equation}
	\tag{D1}
    \begin{cases}
        \max{b^Ty}\\
        A^Ty \leq c\\
        y \geq 0
    \end{cases}
    \label{dual1}
\end{equation}
or, equivalently, if they are written in the form:
\begin{equation}
	\tag{P2}
    \begin{cases}
        \min{c^Tx}\\
        Ax \geq b
    \end{cases}
    \label{primal2}
\end{equation}
\begin{equation}
	\tag{D2}
    \begin{cases}
        \max{b^Ty}\\
        A^Ty = c\\
        y \geq 0
    \end{cases}
    \label{dual2}
\end{equation}

\noindent In (\ref{primal1}) there are no equality constraints (that must be 
considered in a different way).

\noindent You find the dual problem by exchanging the role of unknowns and 
of constraints.

\bigskip
\noindent \textbf{Definition:} a problem is \textbf{feasible} if the set of 
vectors which satisfy the constraints admits at least one element (in this case, you also have a solution).

\noindent Given two problems in duality, they can be:
\begin{itemize}
	\item both unfeasible
	\item one feasible and one unfeasible
	\item both feasible
\end{itemize}

\bigskip
\noindent \textbf{Theorem:} let $v$ be the value of the primal min problem and 
let $V$ be the value of the dual max problem. Then $v \geq V$.

\bigskip
\noindent \textbf{Proof:} we can distinguish two cases:
\begin{enumerate}
	\item consider two problems in duality written in the form \ref{primal1},
	\ref{dual1}. Then
	\[
		c^Tx \geq (A^Ty)^Tx = y^TAx \geq y^Tb = b^Ty
	\]
	Since this is true for all admissible $x$ and $y$, the result follows.

	\item consider two problems in duality written in the form \ref{primal2},
	\ref{dual2}. Then
	\[
		c^Tx = (A^Ty)^Tx = y^TAx \geq y^Tb = b^Ty
	\]
	Since this is true for all admissible $x$ and $y$, the result follows.
\end{enumerate}	

\bigskip
\noindent In (\ref{primal2}) you don't have a constraint like $x\geq 0$, while 
you have it in (\ref{dual2}), where you also have an equality constraint. This 
can be understood when you read the inequalities above: iot be able to write:
\[
	c^Tx \geq (A^Ty)^Tx
\]
you must know that $c \geq A^Ty$ and that $x \geq 0$, so that you're sure 
that the inequality does not change sign. In the second case, you don't 
know which is the sign of x and therefore you need $c = A^Ty$ so that 
you can write $c^Tx = (A^Ty)^Tx$. But then you need $y \geq 0$ iot be able 
to write:
\[
	y^TAx \geq y^Tb
\]
knowing that $Ax=b$, because you must be sure that the ineq. does not change 
its sign.

\bigskip
\noindent The second form of the problems in duality is useful in \textbf{core 
problems}.

\bigskip
\noindent \textbf{Strong duality theorem:} the following results hold:
\begin{itemize}
	\item[-] if the primal and the dual problems are feasible, then both have 
	optimal solutions $\bar{x},\bar{y}$ and the optimal values coincide, i.e. 
	$v = c^T\bar{x} = b^T\bar{y} = V$.

	\noindent In this case we say that there is no duality gap.

	\item[-] if the primal is feasible and the dual is not feasible, then 
	$v = V = -\infty$.

	\item[-] if the dual is feasible and the primal is not feasible, then 
	$v = V = +\infty$.

	\item[-] if both the problems are unfeasible, then $v = +\infty > V = 
	-\infty$.
\end{itemize}

\bigskip
\noindent
\textbf{Note:} the theorem comes from Weierstrass theorem: you 
only prescribe the set of constraints to be bounded iot to grant you have 
a solution (i.e. a minimum or a maximum, according to the fact that you 
are considering (P) or (D)).

\bigskip
\noindent \textbf{Corollary:} if one problem is feasible and has an optimal 
solution, them also the other is feasible and have an optimal solution; 
moreover, there is no duality gap.

\bigskip
\noindent \textbf{Complementarity conditions:} consider two problems in duality 
expressed in the form \ref{primal1},\ref{dual1}.

\noindent \textbf{Theorem:} let $\bar{x}$ and $\bar{y}$ be admissible solutions 
for (\ref{primal1}) and (\ref{dual1}). Then $\bar{x}$ and $\bar{y}$ are 
simultaneously optimal iff they satisfy
\begin{equation}
	\tag{CC}
    \begin{cases}
        \bar{x_i} > 0 \implies \sum_{j=1}^m{a_{ij}\bar{y_j}} = c_i ~
        \forall i=1:n\\
        \bar{y_j} > 0 \implies \sum_{i=1}^n{a_{ij}\bar{x_i}} = b_j ~
        \forall j=1:m
    \end{cases}
    \label{CC}
\end{equation}

\bigskip
\noindent \textbf{Comment:} you lose the optimization 
part: you express the optimality of your solution using only 
inequalities and equalities. (\ref{CC}) means that if, for example, 
$\bar{x_i} > 0 \forall i$ (strictly $>$), then the constraints of (D) 
must be active, which means that they must hold with $=$ instead of 
$\leq$ (and the same in the other case).

\bigskip
\noindent \textbf{Proof:} since $c^Tx \geq y^TAx \geq b^Ty$, then $\bar{x}$ 
and $\bar{y}$ are optimal iff they satisfy $c^T\bar{x} = \bar{y}^TA\bar{x} = 
b^T\bar{y}$.

\noindent This is equivalent to $\bar{x}^T(A^T\bar{y} - c) = 0$ and 
$\bar{y}^T(A\bar{x} - b) = 0$.

\noindent Since $\bar{x},\bar{y} \geq 0$ and $A\bar{x} \geq b$, $A^T\bar{y} 
\leq c$, the latter are equivalent to (\ref{CC}).

\bigskip
{\huge E' proprio Rinaldi, comunque}

\subsection*{Linear programming for zero-sum games}

\noindent Consider a zero sum game described by a payoff matrix P s.t. 
$P_{ij} > 0 ~\forall i,j$, so that $v > 0$. Set $\alpha_i = \frac{x_i}{v}$. Then 
the condition $\sum{x_i} = 1$ becomes $\sum{\alpha_i} = \frac{1}{v}$ and to 
maximize $v$ is equivalent to minimize $\sum{\alpha_i}$.

\noindent Set $\beta_j = \frac{y_j}{v}$; the same argument can be repeated, so 
that we can write, for player1,
\begin{equation}
	\begin{cases}
		\min 1^T_n\alpha\\
		P^T\alpha \geq 1_m\\
		\alpha \geq 0
	\end{cases}
\end{equation}
and, for player2:
\begin{equation}
	\begin{cases}
		\max 1_m\beta\\
		P\beta \leq 1_n\\
		\beta \geq 0
	\end{cases}
\end{equation}
Denote by $v$ the common value of the two problems. We have that:
\begin{enumerate}
	\item $x$ is an optimal strategy for player1 iff $x = v\alpha$ for some 
	$\alpha$ optimal solution of (P).
	\item $y$ is an optimal strategy for player2 iff $y = v\beta$ for some 
	$\beta$ optimal solution of (D).
\end{enumerate}

\subsection*{Complementarity conditions for zero-sum games}

\noindent Being $x$ and $y$ the strategies of the two players, we can write:
\begin{equation}
    \begin{cases}
        \bar{x_i} > 0 \implies \sum_{j=1}^m{p_{ij}\bar{y_j}} = v ~
        \forall i=1:n\\
        \bar{y_j} > 0 \implies \sum_{i=1}^n{p_{ij}\bar{x_i}} = v ~
        \forall j=1:m
    \end{cases}
\end{equation}
Which means that the fact that player1 plays row $i$ with positive probability 
implies that player2 pays exactly $v$ (using an optimal strategy against row 
$i$) and viceversa.

\bigskip
\noindent \textbf{Remark:} there are stil situations in which we do not know 
what does rationality mean.

\noindent \textbf{Example:} consider the game:\\
(1,1)	(0,0)\\
(0,0) 	(1,1)\\
all cryteria we know cannot be applied (it is not a game in extensive form, it 
is not a zero-sum game and there are no strictly dominated strategies), but we 
all agree that the solution should be the first or the fourth element of the 
matrix. 

\noindent The situation is similar in:\\
(3,4) 	(0,0)\\
(0,0)	(4,3)\\
where, however, the first and the fourth elements are not indifferent. This 
means that the two players are forced to coordinate if they do not want to 
end up in (0,0), which is the worst situation for both.

\noindent In a zero-sum-game players do not need to coordinate: there is a 
convex set of optimal strategies. At any time in which both are optimal, one 
player gains $v$ and the other pays $v$, which is the optimal solution for both. 
They naturally agree because the optimal solution for each player is optimal 
for both.

\noindent We will see that there are Nash equilibria that are fair for one 
player but not for the other. Therefore, if one player wants to reach a Nash 
equilibrium and the other wants to reach another, they will end up in a third 
solution, which is not a Nash equilibrium and therefore it is not rational.\\
$\implies$\\
Players need to coordinate.

\section{Symmetric Games}

\noindent \textbf{Definitions:} a square matrix $P \in \mathbb{R}^{n \times n}$ 
is said to be \textbf{antisymmetric} if $P_{ij} = -P_{ji} ~\forall i,j$.

\noindent A (finite) zero-sum game is said to be \textbf{fair} if its payoff 
matrix $P$ is antisymmetric.

\bigskip
\noindent \textbf{Remember:} if P is antisymmetric, then $P^T=-P$.

\bigskip
\noindent \textbf{Note:} in fair games there is no favorite player.

\noindent A symmetric game is fair because both the players have the same 
options. The matrix is squared, which means that pure strategies and mixed 
str. coincide. Moreover, I expect the outcome to be 0 and I expect that 
the same stragegy is optimal for both the players.

\bigskip
\noindent \textbf{Proposition:} if $P$ is antisymmetric, then the value of the 
game is 0 and $\bar{x}$ is an optimal strategy for player1 iff it is optimal 
for player2.

\bigskip
\noindent \textbf{Proof:} since $x^TPx = (x^TPx)^T = x^TP^Tx = -x^TPx$, then we 
have that $f(x,x) = 0$ for all $x$.

\noindent By definition, $v_1=\sup_x{\inf_y{f(x,y)}}$, therefore 
$f(x,x)=0$ means that $inf_y{f(x,x)}\leq 0$, which entails $v_1\leq 0$.

\noindent The converse holds for $v_2$, which means that $v_2 \geq 0$.

\noindent We are in a situation where $v=v_1=v_2$ and the fact that 
$v_1\leq 0$ and $v_2\geq 0$ entails that $v=0$.

\noindent If $\bar{x}$ is optimal for player1, then $\bar{x}^TPy \geq 0$ for all 
$y$ and, by transposing, we get $y^TP\bar{x} \leq 0$ for all $y \in \Sigma_n$.

\noindent Notice that $y \in \Sigma_n$ means that $y$ is a strategy for player1, 
besides its name.

\noindent This entails that $\bar{x}$ is optimal also for the second player, and 
conversely.

\bigskip
\noindent \textbf{Finding optimal strategies in a fair game:} we have to solve 
the system of inequalities:\\
$x_1p_{11} + ... + x_np_{n1} \geq 0$\\
...\\
$x_jp_{1j} + ... + x_np_{nj} \geq 0$\\
...\\
$x_1p_{1m} + ... + x_np_{nm} \geq 0$\\
with the extra conditions $x_i \geq 0 ~\forall i$ and $\sum_{i=1}^n{x_i} = 1$.

\noindent This means that in the case of fair games there is no need to go for 
linear programming, because the value of the game is known and it is equal to 
zero.

%\end{document}

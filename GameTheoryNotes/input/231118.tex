%\documentclass[pt11,a4paper,twoside,reqno,openright]{paper}
%\usepackage[latin1]{inputenc}
%\usepackage{amsmath}
%\usepackage{amsfonts}
%\usepackage{amssymb}
%\usepackage[english]{babel}
%\usepackage{subfigure}
%\usepackage{graphicx}
%\usepackage{float}
%\usepackage{listings,lstautogobble}
%\usepackage[T1]{fontenc}

%\begin{document}

%\noindent {\huge Bargaing problem}

%!TeX root = ./GameTheoryNotes.tex

\bigskip
\noindent \textbf{Slide 4}

\noindent If player2 has the possibility to make a counteroffer, then player1 
is in his hands. The last one who can make an offer is the winner of the game.

\bigskip
\noindent \textbf{Slide 6}

\noindent Suppose that in every day a player makes an offer. Then you have a 
discount factor for every day. In Life::real, it is not very natural to 
assume a stopping time for bargaining; however, let's suppose that for every 
day we have a discount factor. The discount factor is different for 
every player. 

\noindent If you reject my offer, which is (1,0), you will propose (0,1): you 
want to have everything. But now your utility is $1 \times \delta_2$: the 
maximum you can get is $\delta_2$. Then I can make you accept an offer by 
simply proposing you $\delta_2$: by refusing, you cannot get more. 

\bigskip
\noindent \textbf{Slide 8 - 9}

\noindent A play, in the game, is a series of offers. It is possible that the 
game is endless: everybody rejects the offer. In turn, it is possible that the 
offer is accepted at a certain time $T$. The utilities in these cases, 
considering that we are in a discount situation, we have:
\begin{enumerate}
	\item In the first case, we have (0,0)
	\item In the second case, we have an utility which depends on the discount 
	factor. The point is: the more you remain in the bargain situation, the 
	less is your outcome. If you postpone the decision, you will get less.
\end{enumerate}

\noindent In this context, we cannot speak about backward induction: we need 
the game to be finite (i.e. there is a stopping rule, which in this case we 
do not have). However, we can express backward induction in finite games by 
considering each node of the tree as the root of a smaller subgame; we can 
define backw. ind. by saying that its strategy is a str. s.t. its restriction 
to every subgame is a Nash equilibrium for the subgame itself.

\noindent With this definition, we do not care any more that the game must be 
finite: what is important is that the subgame, even if infinite, is actually 
a game and it has a Nash equilibrium.

\noindent \textbf{Remark:} consideriamo il gioco in cui la madre puo' comprare o 
non comprare il gelato e il figlio puo' piangere o no. L'equilibrio di Nash del 
gioco e' (buy,cry). Se restringiamo l'osservazione al subgame di destra, per 
cui la madre non ha possibili scelte e il figlio puo' piangere o non piangere, 
notiamo che "cry" non e' un equilibrio di Nash per il figlio: per lui e' meglio 
non piangere $\implies$ un equilibrio di Nash puo' imporre decisioni irrazionali 
ai giocatori nei rami decisionali che all'atto pratico non verranno mai 
raggiunti (infatti, se la mamma compra il gelato, il bambino non piange).

\bigskip
\noindent \textbf{Slide 11}

\noindent Idea: un giocatore propone, ad esempio (60,40), ma è pronto ad 
accettare anche se l'altro propone (50,50) (e viceversa). Occorre trovare 
un equilibrio tra le due proposte. 

\bigskip
\noindent \textbf{Slide 19}

\noindent We forget alternate offers and strategies, we only speak about 
utilities. The set $C$ is the set of all the possible distributions of utilities; 
in principle we do not know how these utilities are obtained.

\noindent Disagreement point $d$: utility if we do not reach an agreement.

\noindent This is a Non Transferrable Utility cooperative game.

\bigskip
\noindent \textbf{Slide 20}

\noindent \textbf{Remark:} this approach is possible only with two players.

\noindent Assumptions:
\begin{enumerate}
	\item This is obvious: we always take closed sets, otherwise we could 
	miss existence.
	\item Oh, be'
	\item This means that players can reach an agreement s.t. their utility 
	is greater than what they get in case of disagreement. If this assumption 
	were not fullfilled, then there would be at least one player for which 
	there is not reason to bargain. Iot have an interesting bargaining, 
	players must have the possibility to meet somewhere.
\end{enumerate}

%\end{document}

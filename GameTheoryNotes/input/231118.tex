%\documentclass[pt11,a4paper,twoside,reqno,openright]{paper}
%\usepackage[latin1]{inputenc}
%\usepackage{amsmath}
%\usepackage{amsfonts}
%\usepackage{amssymb}
%\usepackage[english]{babel}
%\usepackage{subfigure}
%\usepackage{graphicx}
%\usepackage{float}
%\usepackage{listings,lstautogobble}
%\usepackage[T1]{fontenc}

%\begin{document}

%\noindent {\huge Bargaing problem}

%!TeX root = ./GameTheoryNotes.tex

\noindent Situation: two players have to share a certain good (example: a pie). It is not a zero-sum game, because players have a common interest in reaching an agreement, othewise the good is lost.

\noindent It can be seen either as a cooperative or as a non-cooperative game.

\bigskip
\noindent \textbf{Bargaining as extensive game}

\noindent Suppose that we have two players and the following rules:
\begin{enumerate}
	\item At the first stage, player1 makes an offer: $(x_1,x_2)$ s.t. $(x_1+ x_2) = 1$.
	\item Player2 has the possibility either to accept the offer or to reject; the outcome is $(x_1,x_2)$ in the first case, $(0,0)$ otherwise.
\end{enumerate}

\noindent By backward induction, the solution of the game is $(1,0)$.

\noindent However, if player2 has the possibility to make a counteroffer, then player1 
is in his hands. The last one who can make an offer is the winner of the game. Indeed, in this case we have a game in multiple stages:
\begin{enumerate}
	\item At the first stage, player1 makes an offer: $(x_1,x_2)$ s.t. $(x_1+ x_2) = 1$.
	\item Player2 has the possibility either to accept the offer or to propose $(y_1,y_2)$.
	\item Player1 have the possibility either to accept or to reject the offer (i.e. the game follows as before, with the role of the two players interchanged).
\end{enumerate}

\noindent If the game ends after two stages, the outcome is, by backward induction, $(0,1)$, etc.

\bigskip
\noindent \textbf{Impatient players}

\noindent Suppose that in every day a player makes an offer and suppose there is a 
discount factor $0 < \delta_i < 1$ for every day. In Life::real, it is not very natural to 
assume a stopping time for bargaining; however, let's suppose that for every 
day we have a discount factor. The discount factor is different for 
every player. 

\noindent The game becomes:
\begin{enumerate}
	\item At the first stage, player1 makes an offer: $(x_1,x_2)$ s.t. $(x_1+ x_2) = 1$.
	\item Player2 has the possibility either to accept the offer or to propose $(y_1,y_2)$, but the outcome would be $(\delta_1y_1,\delta_2y_2)$.
	\item etc.
\end{enumerate}

\noindent If you reject my offer, which is (1,0), you will propose (0,1): you 
want to have everything. But now your utility is $1 \times \delta_2$: the 
maximum you can get is $\delta_2$. Then I can make you accept an offer by 
simply proposing you $\delta_2$: by refusing, you cannot get more. 

\noindent $\implies$

\noindent The unique solution, by backward induction, is
\begin{enumerate}
	\item player1 proposes $(1-\delta_2, \delta_2)$
	\item player2 accepts the offer
\end{enumerate}

\noindent Indeed, what happens is the following: if player2 rejects the offer that player1 made at the first stage, the game becomes an ultimatum game in which is player2's turn to make an offer $\implies$ he will propose (0,1), which gives as utilities $(0 \cdot \delta_1, 1 \cdot \delta_2)$ $\implies$ if player1 accepts the offer, he will get nothing and player2 will get exactly $\delta_2$.

\noindent $\implies$ Player2 will accept, at the first stage, any offer $x_2 \geq \delta_2$ because he knows that, in any case, he cannot get more.

\noindent Since player1 knows this, and he knows that, in case he proposes less than $\delta_2$, he will get nothing, his optimal offer is exactly $\delta_2$.

\bigskip
\noindent \textbf{Strategies for the player:}

\noindent
\begin{enumerate}
	\item Player1 offers $(1-\delta_2, \delta_2)$ at the first stage; he accepts every offer in the second stage.
	\item Player2 accepts any offer $x_2 \geq \delta_2$ at the first stage; otherwise he rejects and he proposes $(0,1)$.
\end{enumerate}

\bigskip
\noindent \textbf{Game with infinite horizon}

\noindent A play, in the game, is a series of offers. It is possible that the 
game is endless: everybody rejects the offer. In turn, it is possible that the 
offer is accepted at a certain time $T$. The utilities in these cases, 
considering that we are in a discount situation, are:
\begin{enumerate}
	\item (0,0) in the first case
	\item $(\delta_1^{T-1}x_1^T,\delta_2^{T-1}x_2^T)$ in the second case: the utilities depend on the discount 
	factor. The point is: the more you remain in the bargain situation, the 
	less is your outcome. If you postpone the decision, you will get less.
\end{enumerate}

\noindent In this context, we cannot speak about backward induction: we need 
the game to be finite (i.e. there must be a stopping rule, which in this case we 
do not have). However, we can express backward induction in finite games by 
considering each node of the tree as the root of a smaller subgame; we can 
define backw. ind. by saying that its strategy is a str. s.t. its restriction 
to every subgame is a Nash equilibrium for the subgame itself.

\bigskip
\noindent \textbf{Subgame perfect NEp:} is a NEp s.t. its restriction to every subgame of the initial game represents a NEp for the subgame itself.

\bigskip
\noindent With this definition, we do not care any more that the game must be 
finite: what is important is that the subgame, even if infinite, is actually 
a game and it has a Nash equilibrium.

\noindent \textbf{Remark:} consideriamo il gioco in cui la madre puo' comprare o 
non comprare il gelato e il figlio puo' piangere o no. L'equilibrio di Nash del 
gioco e' (buy,cry). Se restringiamo l'osservazione al subgame di destra, per 
cui la madre non ha possibili scelte e il figlio puo' piangere o non piangere, 
notiamo che "cry" non e' un equilibrio di Nash per il figlio: per lui e' meglio 
non piangere $\implies$ un equilibrio di Nash puo' imporre decisioni irrazionali 
ai giocatori nei rami decisionali che all'atto pratico non verranno mai 
raggiunti (infatti, se la mamma compra il gelato, il bambino non piange).

\bigskip
\noindent \textbf{Strategies for the players - part 2:}

\noindent Idea: a player proposes $w$ and accepts any offer $z$ iff it allows him to get at least a fixed quota (un giocatore propone, ad esempio (60,40), ma e' pronto ad 
accettare anche se l'altro propone (50,50), e viceversa. Occorre trovare 
un equilibrio tra le due proposte).

\noindent In particular
\begin{enumerate}
	\item player1 proposes $\bar{x}$ and accepts $y$ iff $y_1 \geq \bar{y}_1$.
	\item player2 proposes $\bar{z}$ and accepts $w$ iff $w_2 \geq \bar{w}_2$
\end{enumerate}
where $\bar{x}$, $\bar{y}$, $\bar{z}$ and $\bar{w}$ are fixed \textit{a priori}.

\noindent If we consider the two stage game, we notice that $\bar{w}_2$ represents the minimum level of acceptance for player2, i.e. he will reject $\bar{x}_2$ if $\bar{x}_2 < \bar{w}_2$. $\implies$ optimality for player1 implies $\bar{x}_2 = \bar{w}_2$.

\noindent Conversely, we can apply the same argument and we get that optimality for player2 implies $\bar{z}_1 = \bar{y}_1$.

\noindent $\implies$

\noindent $\bar{x} = \bar{w}$ and $\bar{z} = \bar{y}$, which means that the strategies for the two players are the following:
\begin{enumerate}
	\item player1 proposes $\bar{x}$ and accepts $y$ iff $y_1 \geq \bar{y}_1$.
	\item player2 proposes $\bar{y}$ and accepts $x$ iff $x_2 \geq \bar{x}_2$.
\end{enumerate}

\noindent If we consider the two-stages game with impatient players (and therefore discount factors $\delta_i$), we can relate $\bar{x}$ and $\bar{y}$ by saying that
\[
	\bar{y}_1 = \delta_1\bar{x}_1 \hspace{1cm} \bar{x}_2 = \delta_2\bar{y}_2
\]
where we know that $\bar{x}_2 = 1 - \bar{x}_1$ and $\bar{y}_2 = 1 - \bar{y}_1$. 

\bigskip
\noindent \textbf{Theorem:} there is an unique subgame perfect NEp for the bargaining game with alternate offers and impatient players, with the following strategies:
\begin{enumerate}
	\item player1 offers $\bar{x}$ and accepts $y$ iff $y_1 \geq \bar{y}_1$
	\item player2 offers $\bar{y}$ and accepts $x$ iff $x_2 \geq \bar{x}_2$
\end{enumerate}
where
\[
	\bar{x} = \left( \frac{1-\delta_2}{1-\delta_1\delta_2}, 
					     \frac{\delta_2(1-\delta_1)}{1-\delta_1\delta_2} \right)
	\hspace{1cm}	
	\bar{y} = \left( \frac{\delta_1(1-\delta_2)}{1-\delta_1\delta_2},
	 					\frac{1-\delta_1}{1-\delta_1\delta_2} \right)
\]

\bigskip
\noindent \textbf{Outcome of the game:}

\noindent Player1 offers $\bar{x}$ to player2 and player2 accepts the offer at the first stage $\implies$ the utilities for the two players are
\[
	\frac{1-\delta_2}{1-\delta_1\delta_2}
	\hspace{1cm}
	 \frac{\delta_1(1-\delta_2)}{1-\delta_1\delta_2}	
\]

\bigskip
\noindent \textbf{Partial proof:} let $\sigma_i$ be the strategy of player $i$. There are two possibility: either we are considering a subgame starting with an offer or we are considering a subgame starting with the response to an offer.

\noindent In any case, we want to prove that the strategy profile $\sigma = (\sigma_1,\sigma_2)$ restricted to the subgame is a NEp.

\begin{enumerate}
	\item \textbf{Subgame starting with an offer:} suppose we are in node $v$ and it is player1's turn to make an offer. With the NEp $(\sigma_1,\sigma_2)$ he gets $\bar{x}_1$. We want to show that he has no incentive to change his strategy, which means that $(\sigma_1,\sigma_2)$ is a NEp for the subgame.
	\begin{itemize}
		\item if player1 offers $x_2 > \bar{x}_2$, player2 would accept the offer and player1 would get $1 - x_2 < 1 - \bar{x}_2$, therefore he has no reason to offer more than $\bar{x}_2$.
		\item if player1 offers $x_2 < \bar{x}_2$, player2 would reject the offer and would offer $\bar{y}_1 = \delta_1\bar{x}_1 < \bar{x}_1$. Player1 can accept (but he gets less than before) or reject, but in this case the situation is the same as before.
	\end{itemize}
	We need to show that also for player2 there is no incentive to change his strategy; player2 is facing the opposite situation.
	
	\item \textbf{Subgame starting with the response to an offer:} for player2, the strategy $\sigma_2$ prescribes to accept any offer $x_2 \geq \bar{x}_2$. We want to show that he has no incentive to deviate, taking for granted that player1 follows the strategy $\sigma_1$.
	\begin{itemize}
		\item case $x_2 < \bar{x}_2$: according to $\sigma_2$, player2 refuses the offer and makes a counteroffer $\bar{y}_1$. Since player1 plays the strategy $\sigma_1$, the offer is accepted and the outcome for player2 is $\delta_2\bar{y}_2 = \bar{x}_2 > x_2$.
		
		\noindent If, in turn, player2 accepts the offer, he gets $x_2 < \bar{x}_2$, then for him it is not convenient to deviate.
		
		\item case $x_2 > \bar{x}_2$: according to $\sigma_2$, player2 accepts the offer and gets $x_2 > \bar{x}_2$.
		
		\noindent If, in turn, he rejects the offer, knowing that player1 follows $\sigma_1$ he can only offer $\bar{y}_1$, which is accepted, and therefore he gets $\delta_2\bar{y}_2 = \bar{x}_2 < x_2$. Again, for him it is not convenient to deviate.
	\end{itemize}
\end{enumerate}

\bigskip
\noindent \textbf{Symmetric case:} 

\noindent $\delta_1 = \delta_2 = \delta$ $\implies$ the utilities for the players are
\[
	\left( \frac{1}{1 + \delta}, \frac{\delta}{1 + \delta} \right)
\]
which entails that the first player who makes an offer has the better outcome.


% the following is reported in 301118.tex according to the topic of the lesson
%
%
%\bigskip
%\noindent \textbf{Bargaining problem according to Nash}
%
%\noindent We forget alternate offers and strategies, we only speak about 
%utilities. The set $C$ is the set of all the possible distributions of utilities; 
%in principle we do not know how these utilities are obtained.
%
%\noindent Disagreement point $d$: utility if we do not reach an agreement.
%
%\noindent This is a Non Transferrable Utility cooperative game.
%
%\bigskip
%\noindent \textbf{Set of bargaining problems:}
%
%\noindent \textbf{Remark:} this approach is possible only with two players.
%
%\noindent Assumptions: the set $\mathcal{C} = \{(C,d)\}$ is s.t.
%\begin{enumerate}
%	\item $C$ is a closed bounded convex subset of $\mathbb{R}^2$.
%	
%	\noindent This is obvious: we always take closed sets, otherwise we could 
%	miss existence.
%	
%	\item $d \in \mathbb{R}^2$. 
%	
%	\noindent Oh, be'
%	
%	\item $\exists x \in C$ s.t. $x_1 > d_1$ and $x_2 > d_2$.
%	
%	\noindent This means that players can reach an agreement s.t. their utility 
%	is greater than what they get in case of disagreement. If this assumption 
%	were not fullfilled, then there would be at least one player for which 
%	there is not reason to bargain. Iot have an interesting bargaining, 
%	players must have the possibility to meet somewhere.
%\end{enumerate}

%\end{document}

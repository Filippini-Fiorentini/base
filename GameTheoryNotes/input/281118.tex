%\documentclass[pt11,a4paper,twoside,reqno,openright]{paper}
%\usepackage[latin1]{inputenc}
%\usepackage{amsmath}
%\usepackage{amsfonts}
%\usepackage{amssymb}
%\usepackage[english]{babel}
%\usepackage{subfigure}
%\usepackage{graphicx}
%\usepackage{float}
%\usepackage{listings,lstautogobble}
%\usepackage[T1]{fontenc}

%\begin{document}

%\noindent {\huge The Bargaining problem}

%!TeX root = ./GameTheoryNotes.tex

\bigskip
\noindent \textbf{/*}

\noindent \textbf{RIPASSO}

\noindent First idea is: player1 makes an offer and player2 can only accept or 
refuse the offer.

\noindent The idea of Nash eq. can be seen as a necessary condition of 
rationality: a player cannot accept a proposal which is not a Nash eq.

\noindent The game has perfect information $\implies$ we can apply backward 
induction iot analyse the game.

\noindent Second possibility: we can consider alternating offers from player1 
and player2. If there is a stopping cryterion (ex. we can make only 10 offers), 
then the equilibrium is: one player offers "I keep all and you takes nothing" 
and the other accepts.

\noindent Third possibility: players are impatient $\implies$ we apply to the 
offers a discount factor for any day they spend in bargaining. In this case, 
player1 should offer to player2 exactly his discount factor $\delta_2$, so that 
he accepts because at an hypothetical second stage he cannot get more than this.

\noindent What are the strategies of the game? We have to tell what the players 
would do at any node, not only in the equilibrium path!\\
$\implies$\\
Strategies: see slide 7.

\noindent Possible generalization: infinite number of stages (i.e. there is not 
a stopping rule). We can have an infinite play, in the sense that a player makes 
an offer and the other always refuses, or a finite game, in the sense that, 
after a certain number of offers and counteroffers, one player accepts. In 
the first case, utilities are zero, while in the second case we have the 
strategies given in slide 8, where $T$ is the time at which the offer is 
accepted.

\noindent We cannot apply backward induction if the length of the game is 
infinite. However, backward induction has the following property: if we 
consider a node which is not the root and we take it as root for a subgame, 
the subgame itself is a game $\implies$ the restriction to the subgame of the 
players' strategies produces a Nash equilibrium at the root of the subgame, if 
we apply backward induction.

\noindent Nash equilibria can prescribe an irrational choice to a player in 
some parts of the tree, which actually are not reached by the game, while 
this is not possible with backward induction (example: the mother and the son).

\noindent Good solution for infinite games with perfect information: subgame 
perfect Nash equilibrium profile (see slide 9).

\noindent If at the $1^{st}$ stage I am player1 (I make an offer) and you are 
player2, at the second stage the game is exactly the same (same preferences 
and same rules), but for the fact that now you are player1 and I am player2. In 
the third game, we return exactly to the situation of the first day etc. This 
means that in day3 I should offer exactly the same I offered in day1.

\noindent In Life::real, it may happen that I offer in the first day (80,20) and 
finally we agree for (60,40), but this is possible because I actually do not 
know your utility function and I have to try to say something and see your 
reaction! In our model, we assume to have perfect information and such a 
behaviour is not necessary.

\bigskip
\noindent To prove that it is a subgame perfect NEp means that I must 
analyse any subgame and see what happens.

\noindent \textbf{*/}

%\end{document}

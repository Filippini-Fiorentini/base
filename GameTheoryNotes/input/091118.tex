%\documentclass[pt11,a4paper,twoside,reqno,openright]{paper}
%\usepackage[latin1]{inputenc}
%\usepackage{amsmath}
%\usepackage{amsfonts}
%\usepackage{amssymb}
%\usepackage[english]{babel}
%\usepackage{subfigure}
%\usepackage{graphicx}
%\usepackage{float}
%\usepackage{listings,lstautogobble}
%\usepackage[T1]{fontenc}

%\begin{document}

%\noindent {\huge \textbf{Cooperative2}}

%!TeX root = ./GameTheoryNotes.tex

\bigskip
\noindent \textbf{Revision of the last lecture}

\noindent In a TU game, to say that the value of coalition $\{1,2\}$ is 10 
means that the best value players 1 and 2 can get when they play alone is 10. 
They are called transferrable because player1 and 2 can decide how to share 
this value among them.

\noindent The game is described by a vector space, which has dimension $2^N-1$, 
where the $-1$ is due to the fact that the value of the empty set is zero (I 
actually have $2^n-1$ free parameters).

\noindent There exists a canonical basis, whose number of elements is the 
number of coalitions.

\noindent "Unanimity games": all the players in a winning coalition are 
necessary and sufficient.

\noindent Additive games: $v(A \cup B) = v(A)+v(B)$. When I make the union of 
two coalitions, the value of the resulting coalition is the sum of the values 
of the two coalitions $\implies$ being together is neautral: it does not give 
neither an advantage nor a disadvantage.

\noindent Superadditive games: to form any coalition it always convenient.  

\noindent Convex games: they enforce the condition of superadditivity, because 
we do not ask the coalitions to be disjoint. Convex $\implies$ superadditive.

\noindent In simple games we're not interested in the values of the coalitions, 
but we only divide them in "winning coalitions" and "losing coalitions". We 
assume that there are no players which are harmful to other players: we can add 
any player to a winning coalition and it remains winning $\implies$ we are 
interested in the minimal winning coalitions.

\noindent In particular, unanimity games are simple games in which the minimal 
winning coalition is unique. In turn, weighted majority game are games in which 
there may exist more than one minimal winning coalition.

\noindent Solution vector: a vector of utilities. Solution concept: it assigns 
a bunch (a set) of possibile utilities to the game. Remember that in the 
solution concept the coalitions disappear: utilities are assigned to the single 
players, not to the coalitions. Indeed, the players participate to the game, 
but they are egoistic: they want to know what THEY would get from the game.

\noindent The first idea of solution is the imputation set (see slide 18).

\noindent {\huge min 30 circa: la stessa cosa che non ho capito nella 
lezione del 31...vediamo se qui è meglio XD}

\noindent Idea of the core: we want to have a solution to whom no coalition 
would object.

\bigskip
\noindent \textbf{Slide 24}

\noindent If it is not empty, the core is a polytop: indeed, it is generated by 
a finite number of inequalities (which is precisely the definition of polytop).

\noindent For additive games, the core is given by $C(v)~=~[v(1),v(2)...v(n)]$. 
Indeed, when we define the core we want to have 
$\sum_{i \in S}{v(i)} \geq v(S)$, but actually, by definition of additive 
games, this relation holds with the equality.

\bigskip
\noindent \textbf{Slide 25}

\noindent A veto player is someone which is essential to approve something. In 
particolar, a coalition is losing whenever it does not contain the veto player, 
while in general it is not true that a coalition for sure wins if the veto 
player is present (the veto player is necessary but not sufficient).

\noindent \textbf{Proof of the theorem:} first of all, I prove that if there is 
no veto player, then the core is empty, i.e. $N-\{i\}$ is a winning coalition 
$\forall i$ (otherwise $i$ would be a veto player).

\noindent (1) with $\geq~\implies$ I need to give at least 1 to all the 
coalitions which exclude $i$.

\noindent We compute $\sum_{i=1}^n$ of the inequality (1) and we get:
\[
	\sum_{j=1}^n{x_j} \geq \frac{n}{n-1}
\]
which is impossible since the sum at the lhs is actually 1.

\noindent In turn, suppose that there exists a veto player $i$. We claim that 
$(0,...,1,0...,0) \in C(v)$, where we have 1 in position $i$. We have to prove 
that $\sum_{j \in S}{x_j} \geq v(S)$. In particular, $v(S)=0$ if $S$ is not 
winning (and the former inequality is trivial). $v(S)=1$ if $S$ is winning, 
which means that $S$ contains the veto player. But then the sum at lhs is 1.

\noindent For each veto player, $(0,0,...,1,0,...,0)$ belongs to the core. But 
the core is convex, which means that all the convex combinations of $e_i$ 
belong to the core $\implies$ not all the distributions which belong to the 
core are fair, in the sense that they give a symmetric outcome to symmetric 
players (for example, two veto players have no symmetric result if we consider 
either $(1,0)$ or $(0,1)$: one gets all and the other nothing). On the other 
hand, if two players are symmetric, for sure the core contains a symmetric 
solution (in the previous example, $(\frac{1}{2},\frac{1}{2})$). 

\bigskip
\noindent \textbf{Slide 27}

\noindent I want to minimize the total sum, but making all coalitions happy. 

\noindent This problem has always solution.

\noindent \textbf{Note:} there is no feasibility constraint: I do not have 
any condition "sum equal to $v(N)$".

\noindent Consider a function $f:X \rightarrow \mathbb{R}$. How do I prove that 
$f$ admits a minimum? Consider the sequence $\{x_n\}$ s.t. $f(x_n) \rightarrow 
\inf{f}$. Now you can suppose $x_n \rightarrow x_0$
{\huge min 1:09 circa}

\noindent I know that the problem has a value $v$. Now I have two possibilities:
\begin{enumerate}
\item $v > v(N)$, which implies that the core is empty: iot make all the 
coalitions happy, we would have to spend more than $v(N)$, i.e. more than what 
we have, which is not possible.
\item $v=v(N)$, which implies that the core coincides with $S$. Note that $S$ 
is always not empty, but that $S$ is actually the core iff its value is 
feasible. 
\end{enumerate}

\noindent Equivalent formulation in terms of linear programming problem: there 
is a LP problem which has always solution; the core is empty if the solution is 
$>v(N)$ and it is $S$ if the solution is $v(N)$.

\bigskip
\noindent \textbf{Slide 28 - He loves this teorem}

\noindent Having an LP problem, we want to dualize it.

\noindent The index $S$ of the vector $\lambda_S$ is a coalition.

\noindent Condition (1) tells that, supposing you are player1, the sum of all 
the coalitions containing you is 1 $\implies$ you can interpret $\lambda_S$ as 
the quota, the percentage you assign to you in the coalition $S$. What is 
important is that if, in the coalition $\{1,2\}$, the percentage assigned to 
player1 is $\frac{1}{2}$, then the same must hold for player2: 
$\lambda_{\{1,2\}}$ does not depend on the player you are, but only on the 
coalition.

\noindent Condition (2) {\huge min 07/08}

\bigskip
\noindent \textbf{Slide 29}

\noindent The vector $b$ represents the game. The matrix $A$ is given by 
$\sum_{i \in S}{x_i}$. $A_{ij}=1$ if player $j$ belongs to the coalition $i$.

\noindent The number of variables is equal to the number of coalitions.

\noindent In the dual, I put the equality and not the inequality because in the 
primer problem I do not have nonnegativity constraints.

\bigskip
\noindent \textbf{Slide 30}

\noindent Matrix $A$ for the three players game:\\
1	0	0\\
0	1	0\\
0	0	1\\
1	1	0\\
1	0	1\\
0	1	1\\
1	1	1\\

\noindent Transposed matrix:\\
1	0	0	1	1	0	1\\
0	1	0	1	0	1	1\\
0	0	1	0	1	1	1\\

\bigskip
\noindent \textbf{Slide 31}

\noindent Examples:
\begin{enumerate}
\item A partition is a balancing family because all the players belong to 
one and only one subset (coalition) and therefore every player is fully 
represented by this coalition.
\item All the players but one belong to two coalitions; the 
last player belongs to only one coalition.
\item All the players are present in one coalition and in the grand coalition. 
They can give whatever value to one coalition and $1-"that~value"$ to the 
grand coalition.
\end{enumerate}

\bigskip
\noindent \textbf{Slide 33}

\noindent Example (3) in slide 31 is not a minimal balancing family: for 
example, we can delete the grand coalition and we still have a balancing 
family because we have a partition; on the other hand, we can erase $\{1\}, 
\{2\},\{3\}$ and we still have a balancing family: we are in a comunist 
regime and all the players are fully represented by the estate.

\noindent Iot have the extreme points, we only need to find the minimal 
balancing family.

%\end{document}

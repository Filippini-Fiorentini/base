\documentclass[pt11,a4paper,twoside,reqno,openright]{paper}
% dimensione caratteri 11pt
% foglio A4
% stampa fronte/retro
% numero equazioni stampato sulla destra
% inizio capitoli sempre sulla pagina destra
\usepackage[latin1]{inputenc} % codifica
\usepackage{amsmath} % ambienti per le equazioni
\usepackage{amsfonts}
\usepackage{amssymb} % simboli matematici
\usepackage[english]{babel} % lingua
\usepackage{subfigure}
\usepackage{graphicx} % per inserire le immagini
\usepackage{float}
\usepackage{listings,lstautogobble}
\usepackage[T1]{fontenc}

\begin{document}

\bigskip
\noindent \textbf{Slide 14 - convex combination:}

\noindent $\lambda_1$ ... $\lambda_n$ looks like mixed strategies


\bigskip
\noindent \textbf{Slide 15}

\noindent A set is convex iff every convex combination of two
elements belonging to the set is in the set
$=>$
The same holds if you consider $n$ elements instead of two.
(The proof is done by induction)

\noindent Idea: I want the smallest convex set containing C; the intersection
of convex sets is convex $=>$ I can consider the intersection
and it is the smallest convex set containing C.

\bigskip
\noindent \textbf{Slide 16}

\noindent $p$ is the point at minimum distance; I'm minimizing a continuous 
function in a closed set $=>$ I'm sure that $p$ exists. In finite
dimensions, the existence of $p$ does not depend on convexity, but 
convexity is needed for $1!$ and for the characterizations:
\begin{itemize}
\item $p \in C$
\item $<x-p, c-p> \leq 0 \forall c \in C$
\end{itemize}

\bigskip
\noindent \textbf{Slide 17}

\noindent Separation means I'm able to decide how to define a plane s.t. it
separates two sets $=>$ the intersection of the interior of the sets 
is empty. 
A "weaker" definition of separation allows the sets to share some
points (the sets cannot be open, since they have to share points
on the boundary...they cannot share points in the interior, otherwise
I cannot speak about separation)

\noindent Consider the case in which I want to separate a set C from a point
$\bar{x}$, which belongs to the closure of the complement (REMEMBER:
THE CLOSURE OF $C^c$ contains the boundary of C)

\noindent First case: $\bar{x} \in int(C^c)$

\noindent The point $\bar{x}$ is outside from the set, then it cannot coincide
with the projector.

\[\langle x^*,c \rangle \geq \langle x^*, \bar{x} \rangle \forall c \in C\]

\noindent Remark: in the relation 
\[\langle x^*,c \rangle \geq b \geq \langle x^*, \bar{x} \rangle 
\forall c \in C\]
we can replace $x^*$ with $y^* = ax^*, a > 0$

\noindent Now consider $\bar{x}$ and any sequence $\bar{x_n} \in int(C^c)$. I apply the previous step to $\bar{x_n}$.
Indeed, exists $x_n^*$ whose norm is 1 and s.t. exists $b_n$ and
$\langle x_n^*,c \rangle \geq b_n \geq
 \langle x_n^*, \bar{x_n} \rangle \forall c \in C$

\noindent Therefore $x_n^*$ converges to $x^*$ up to subsequences, the norm of
$x^*$ is 1, $\bar{x_n} \rightarrow \bar{x}$ and the same for $b_n$.

\bigskip
\noindent \textbf{Slide 19}
The theorem allows you to separate two sets as long as they are closed,
convex.
Remark: I don't make assumptions on the interior of B: it can have an
empty interior (ex. I can separate a plane and a line in $\mathbb{R}^2$, even
if the line has empty interior in $\mathbb{R}^2$)

\noindent Remark: $0 \in int(A)-C \iff 0 = a-c$ for $a \in int(A)$ and 
$c \in C$, but
this means that $a = c$, i.e. $c \in int(A)$, which is not possible since
the intersection between C and $int(A)$ in empty.
$=>$
I can separate 0 from $int(A)-C$
$=>$
exists $x^*$ different from 0 s.t.
$\langle x^*,u \rangle \geq \langle x^*,0 \rangle ~\forall u \in int(A)-C$
$=>$
$\langle x^*, a-c  \rangle \geq 0 ~\forall a,c$ s.t. $u=a-c$
$=>$
$\langle x^*,a \rangle \geq \langle x^*,c \rangle ~\forall a \in int(A),
c \in C$

\noindent \textbf{Slide 20 - proof of VN theorem}
Let P be a matrix s.t. $P_{ij} > 0 \forall i,j$ and let $p_1...p_m$
be the columns of P.
I want to find a solution for the game represented by P, which means
that I want to find $\bar{x}$ and $\bar{y}$ s.t. $f(\bar{x},y) \geq v~
\forall y$ and $f(x,\bar{y}) \leq v ~\forall x$, where $f(x,y) = 
x^TPy$.

% Mi dà enormemente fastidio il suo livello di "diseallinamento"
% (cravatta, bottoni camicia, passante cintura, etc)

\noindent REMEBER: once you fix $x$, $x^TPy$ is linear in $y$, which means that
the utility function of the second player is linear in $y$ given a
value for $x$.

\noindent Question: what happens if the bisector of the $1^{st},3^{rd}$ 
quadrant does not intersect the "polygon" of the possible outcomes?
A similar situation happens only if the first row strictly dominates
the second (or viceversa), then you know for sure which row you will
play.

\end{document} 

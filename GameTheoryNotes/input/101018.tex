%\documentclass[pt11,a4paper,twoside,reqno,openright]{paper}
%\usepackage[latin1]{inputenc} 
%\usepackage{amsmath} 
%\usepackage{amsfonts}
%\usepackage{amssymb} 
%\usepackage[english]{babel} 
%\usepackage{subfigure}
%\usepackage{graphicx} 
%\usepackage{float}
%\usepackage{listings,lstautogobble}
%\usepackage[T1]{fontenc}

%\begin{document}

%!TeX root = ./GameTheoryNotes.tex

\bigskip
\noindent \textbf{Convex set:} a set $C \in \mathbb{R}^n$ is said to be convex 
if
\[
	\lambda x + (1 - \lambda)y \in C
\]
$\forall x,y \in C$ and $\forall \lambda \in [0,1]$, i.e. a set is convex iff 
every convex combination of two elements belonging to the set is in the set.

\noindent \textbf{Remarks:}
\begin{enumerate}
	\item The empty set is convex.
	\item The intersection of an arbitrary family of convex sets is convex.
	\item A closed convex set with nonempty interior coincides with the closure 
	of its internal points.
\end{enumerate}

\bigskip
\noindent \textbf{Convex combination:} the vector $x$ is the convex combination 
of $x_1,...,x_n$ if 
\[x = \lambda_1x_1 + ... + \lambda_nx_n\]
with $\lambda_i \geq 0 ~\forall i$ and $\sum_{i=1}^n{\lambda_i} = 1$.

\noindent \textbf{Remark:} $\lambda_1$ ... $\lambda_n$ looks like mixed 
strategies.

\bigskip
\noindent \textbf{Proposition:} a set $C$ is convex iff for every $\lambda,...,
\lambda_n \geq 0$ s.t. $\sum_{i=1}^n{\lambda_i} = 1$, for every $c_1,...,c_n 
\in C$, for all $n$, then $\sum_{i=1}^n{\lambda_ic_i} \in C$.

\noindent $\implies$ the fact that a set is convex iff every convex combination 
of two elements is contained in the set holds also if you consider $n$ elements 
instead of two. (The proof is done by induction)

\bigskip
\noindent Idea: Given a generic set $C$, which may be not convex, I want the 
smallest convex set containing $C$; the intersection
of convex sets is convex $=>$ I can consider the intersection
and it is the smallest convex set containing $C$.

\bigskip
\noindent \textbf{Convex hull:} the convex hull of a set $C$, denoted by $coC$, 
is
\[
	coC = \bigcup_{A \in \mathcal{C}}A
\]
where $\mathcal{C} = \{A: C \subset A \wedge \text{A is convex}\}$

\bigskip
\noindent \textbf{Proposition:} given a set $C$, then
\[
	coC = \{\sum_{i=1}^n{\lambda_ic_i}: \lambda_i\geq0,
	        \sum_{i=1}^n{\lambda_i} = 1, c_i \in C ~\forall i,n \in 
	        \mathbb{N}\}
\]

\bigskip
\noindent \textbf{Theorem:} given a closed convex set $C$ and a point $x 
\notin C$, $\exists ! p \in C$ s.t. $||p-x|| \leq ||c-x|| ~\forall c \in C$.

\noindent Moreover, $p \in C$ and $\langle x-p, c-p \rangle \leq 0 ~\forall c 
\in C$.

\bigskip
\noindent $p$ is the point at minimum distance; I'm minimizing a continuous 
function in a closed set $\implies$ I'm sure that $p$ exists. In finite
dimensions, the existence of $p$ does not depend on convexity, but 
convexity is needed for $1!$ and for the characterizations:
\begin{itemize}
\item $p \in C$
\item $<x-p, c-p> \leq 0 \forall c \in C$
\end{itemize}

\subsection*{Separation Results}

\noindent Separation means I'm able to decide how to define a plane s.t. it
separates two sets $\implies$ the intersection of the interior of the sets 
is empty. 
A "weaker" definition of separation allows the sets to share some
points (the sets cannot be open, since they have to share points
on the boundary...they cannot share points in the interior, otherwise
I cannot speak about separation).

\bigskip
\noindent \textbf{Theorem:} let $C$ be a convex proper subset of the euclidean 
space $\mathbb{R}'$, let $\bar{x} \in \bar{C^c}$. Then there is an element 
$x^* \in \mathbb{R}, x^* \neq 0$, s.t. $\langle x^*,c \rangle \geq
\langle x^*,\bar{x}\rangle ~\forall c \in C$.

\bigskip
\noindent \textbf{Proof:} I want to separate a set $C$ from a point $\bar{x}$, 
which belongs to the closure of the $C^c$ (REMEMBER: the closure of $C^c$ 
contains the boundary of C).

\noindent We proceed in two steps:
\begin{enumerate}
	\item Suppose that $\bar{x} \in int(C^c)$, i.e. $\bar{x} \notin \bar{C}$.

	\noindent The point $\bar{x}$ is outside from the set, then it cannot 
	coincide with the projector.

	\noindent Call $p$ the projection of $\bar{x}$ on $\bar{C}$. Then 
	$\langle \bar{x}-p, c-p\rangle \leq 0 ~\forall c \in C$.

	\noindent Define $x^* = p - \bar{x} \neq 0$. Then 
	$\langle x^*, c-\bar{x}\rangle \geq ||x^*||^2$, which implies 
	$\langle x^*,c \rangle \geq \langle x^*,\bar{x} \rangle ~\forall c \in C$.

	\noindent Remark: in the relation 
	\[
		\langle x^*,c \rangle \geq b \geq \langle x^*, \bar{x} \rangle 
		\forall c \in C
	\]
	we can replace $x^*$ with $y^* = ax^*, a > 0$

	\noindent We can choose $x^*$ s.t. $||x^*|| = 1$.

	\item If $\bar{x} \in \bar{C} \setminus C$, consider a sequence $\{x_n\}
	\subset C^c$ such that $x_n \rightarrow \bar{x}$.

	\noindent From the first step, we can find $x^*_n$ s.t. $||x^*_n|| = 1$ 
	and $\langle x^*_n,c \rangle \geq \langle x^*_n,x_n \rangle ~\forall c \in 
	C$.

	\noindent Therefore $x_n^*$ converges to $x^*$ up to subsequences, the norm 
	of $x^*$ is 1 and we pass to the limit in the above inequality, recovering 
	$\langle x^*,c \rangle \geq \langle x^*,\bar{x} \rangle ~\forall c \in C$.
\end{enumerate}

\bigskip
\noindent \textbf{Corollary:} let $C$ be a closed convex set in an Euclidean 
space and let $x \in \partial C$. Then there is an hyperplane, called the 
hyperplane supporting $C$ at $x$, containing $x$ and leaving all of $C$ in one 
of the halfspaces determined by the hyperplane.

\bigskip
\noindent \textbf{Corollary:} let $C$ be a closed convex set in an Euclidean 
space. Then $C$ is the intersection of all halfspaces containing it.

\bigskip
\noindent \textbf{Theorem:} let $A, C$ be closed convex subsets of $\mathbb{R}'$ 
s.t. the interior of $A$ is not empty and $int(A) \cap C = \emptyset$. Then 
there exists $x^* \neq 0$ and $b \in \mathbb{R}$ s.t. 
$\langle x^*,a \rangle \geq b \geq \langle x^*,c \rangle ~\forall a \in A, 
~\forall c \in C$.

\bigskip
\noindent The theorem allows you to separate two sets as long as they are 
closed and convex.

\noindent Remark: I don't make assumptions on the interior of C: it can have an
empty interior (ex. I can separate a plane and a line in $\mathbb{R}^2$, even
if the line has empty interior in $\mathbb{R}^2$)

\bigskip
\noindent \textbf{Proof:} $0 \in int(A)-C \iff 0 = a-c$ for $a \in int(A)$ and 
$c \in C$, but
this means that $a = c$, i.e. $c \in int(A)$, which is not possible since
the intersection between C and $int(A)$ in empty.\\
$\implies$\\
I can separate 0 from $int(A)-C$\\
$\implies$\\
exists $x^* \neq 0$ s.t. 
$\langle x^*,u \rangle \geq \langle x^*,0 \rangle ~\forall u \in int(A)-C$\\
$\implies$\\
$\langle x^*, a-c  \rangle \geq 0 ~\forall a,c$ s.t. $u=a-c$\\
$\implies$\\
$\langle x^*,a \rangle \geq \langle x^*,c \rangle ~\forall a \in int(A),
c \in C$\\
$\implies$\\
$\langle x^*,a \rangle \geq \langle x^*,c \rangle ~\forall a \in cl(int(A)),
c \in C$ and the closure of $int(A)$ is equal to A, thus the theorem is proven.

\bigskip
\noindent $H = \{x: \langle x^*,x \rangle = b\}$ is called the 
\textbf{separating hyperplane}: A and C are contained in two different 
halfspaces generated by H.

\bigskip
\subsection*{Proof of VN theorem}

\noindent \textbf{Proof:} let P be a matrix s.t. $P_{ij} > 0 
~\forall i,j$. This can be done without losses of generality because in any 
case, if an entry is negative, I can decide to add 
a constant everywhere iot recover positive outcomes without changing the 
game (from the point of view of optimal strategies). Indeed, if I pass 
from $P_{ij}$ to $P_{ij}+10 ~\forall i,j$, I get that $f(x,y)=f(x,y)+10$.

\noindent Let $p_1...p_m$ be the columns of P.

\noindent \textbf{Idea:} I want to find a solution for the game represented 
by P, which means that I want to find $\bar{x}$ and $\bar{y}$ s.t. 
$f(\bar{x},y) \geq v~\forall y$ and $f(x,\bar{y}) \leq v ~\forall x$, where 
$f(x,y) = x^TPy$ (by definition, since I'm considering mixed strategies).

% Mi dà enormemente fastidio il suo livello di "diseallinamento"
% (cravatta, bottoni camicia, passante cintura, etc)

\noindent Geometrically, the proof is represented by the case where I have $2m$ 
strategies: the first player only has two strategies, while the second player 
has $m$ strategies (the number of columns). Therefore the columns of the matrix 
are $m$ 2D vectors inside the first orthant (they are inside for sure, since all 
the components are positive by assumption).

\noindent Now I can consider the convex hull of these vectors. Every element in 
the set $C$ is given by $\lambda_1p_1 + ... + \lambda_mp_m$ with $\lambda_i 
\geq 0$ and $\sum{\lambda_i} = 1$ because it is a convex combination of the 
points $p_1,...,p_m$.

\noindent This means that every point in $C$ is a possible mixed strategy on the 
columns. Once this is known, player1 can compute his expected values and he can 
choose the best outcome for him: if player2 uses $\lambda_1...\lambda_n$ as 
mixed str., this generates the vector $\lambda_1p_1...\lambda_np_n$ (which is a 
vector inside the polytope since it is a convex combination of the vertexes).
But, knowing the choice of player2, player1 looks at the possible outcomes 
and he chooses the best for him. The goal for player2 is to choose the convex 
combination that provides him the best outcome (i.e. the smaller amount to pay).

\noindent It is possible to show that, if there is an optimal strategy for 
player2, then it is the intersection between the polytope and the bisector of the 
$1^{st}$ quadrant.

\noindent What about the first player? If there is a solution of the game, then 
no one of the players must have an incentive to change (: we are in an 
equilibrium). This entails that the optimal strategy for player1, who knows that 
player2 plays $\bar{y}$, must have a direction orthogonal to the optimal 
strategy for player2.

\noindent REMEBER: once you fix $x$, $x^TPy$ is linear in $y$, which means that 
the utility function of the second player is linear in $y$ given a 
value for $x$.

\noindent Question: what happens if the bisector of the $1^{st},3^{rd}$ 
quadrant does not intersect the polytope of the possible outcomes?
A similar situation happens only if the first row strictly dominates
the second (or viceversa), then you know for sure which row you will
play.

%\end{document} 

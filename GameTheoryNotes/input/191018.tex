%\documentclass[pt11,a4paper,twoside,reqno,openright]{paper}
%\usepackage[latin1]{inputenc}
%\usepackage{amsmath}
%\usepackage{amsfonts}
%\usepackage{amssymb}
%\usepackage[english]{babel}
%\usepackage{subfigure}
%\usepackage{graphicx}
%\usepackage{float}
%\usepackage{listings,lstautogobble}
%\usepackage[T1]{fontenc}

%\begin{document}

\section*{Finite Games}

\noindent The sets of strategies for both the players are finite (in particular, $n$ strategies for player1 and $m$ for player2).

\noindent The game can be represented by a bimatrix whose element $ij$ is the pair $(a_{ij},b_{ij})$, where $a_{ij}$ is the utility of player1 when he plays row $i$ and the other plays column $j$ (and the same for $b_{ij}$).

\noindent We denote this game by $(A,B)$.

\bigskip
\noindent \textbf{Corollary:} a finite game admits always a Nash equilibrium profile in mixed strategies.

\bigskip
\noindent Indeed, in this case $X$ and $Y$ are simplexes and we have that $f(x,y) = x^TAy$, while $g(x,y) = x^TBy$, therefore the hypothesis of the Nash theorem are fulfilled.

\bigskip
\noindent \textbf{Slide 16}

\noindent Cosider the game:\\
(1,0)		(0,3)\\
(0,2)		(1,0)\\
and find its Nash equilibria. First of all, we check for pure Nash 
equilibria. 
Fixed the $1^{st}$ row, the best reaction for player2 is the 
$2^{nd}$ col; fixed the $2^{nd}$ row, the best reaction for player2 is the $1^{st}$ col.

\noindent In turn, fixed the $1^{st}$ col, the best reaction for player1 is the $1^{st}$ row and, fixed the $2^{nd}$ col, the best reaction for player1 is the $2^{nd}$ row.\\
$\implies$\\
There are no pure equilibria, we have to look for mixed ones. We 
have to look for $f(x_1,x_2,y_1,y_2)$. Since we know that $x_1+x_2=1$ we can 
write:\\
$x_1 = p$\\
$x_2 = 1-p$\\
$y_1 = q$\\
$y_2 = 1-q$\\
and then we get:
\[
	f(p,q) ~= ~pq + 0*p*(1-q) + 0*(1-p)*q + (1-p)*(1-q) ~=
\]
\[
	pq - p - q + pq + 1 ~=~ +p(2q-1) - q + 1
\]
We get that, if $q<\frac{1}{2}$, then the best reaction is $p=0$, while, if 
$q>\frac{1}{2}$, the best is $p=1$. Finally, if $q=\frac{1}{2}$, then 
$p\in [0,1]$.

\noindent We can do the same for the player2, i.e. writing:
\[
	g(p,q) ~=~ 3p(1-q)+2q(1-p) ~=~ q(2-5p)+3p
\]

\noindent $BR_2(p)$, i.e. the best reaction to $p$ of player2, is $q=0$ if 
$p>\frac{2}{5}$, $q=1$ if $p<\frac{2}{5}$ and $q\in [0,1]$ if $p=\frac{2}{5}$.

\noindent Overall, the mixed Nash equilibrium profile is 
$\left( \frac{2}{5},\frac{3}{5} \right)$ for player1, while it is 
$\left( \frac{1}{2},\frac{1}{2} \right)$ for player2.

\bigskip
\noindent \textbf{Observation:} suppose that there is not domination in a 2x2 case. 
This means that sometimes I would choose the first row and sometimes the 
second (and the same for the columns).

\noindent In the two by two case, when there is no domination, 
we can have either no pure equilibria and 
one mixed eq. or we can have 2 pure equilibria and one mixed eq.

\bigskip
\noindent \textbf{Remember:} when I am at the equilibrium, I must get the same 
outcome for both the rows (or the columns) $\implies$ \textbf{indifference 
principle}, which is the counterpart of the complementarity conditions 
in a general game (which is not zero-sum).

\bigskip
\noindent \textbf{Indifference principle:} suppose that $(\bar{x},\bar{y})$ is a NE in mixed strategies. Suppose that the support of $\bar{x}$, i.e. the set of indices s.t. $\bar{x_i} > 0$, is 
$\{1,2,...,k\}$ and that the support of $\bar{y}$, i.e. the set of indices s.t. $\bar{y_j} > 0$, is $\{1,2,...,l\}$. Let $f(\bar{x},\bar{y}) = v$. 

\noindent Then it holds:
\begin{equation*}
	\begin{cases}
		a_{11}\bar{y_1} + ... + a_{1l}\bar{y_l} = v\\
		...\\
		a_{k1}\bar{y_1} + ... + a_{kl}\bar{y_l} = v\\
		a_{(k+1)1}\bar{y_1} + ... + a_{(k+1)l}\bar{y_l} \leq v\\
		...\\
		a_{n1}\bar{y_1} + ... + a_{nl}\bar{y_l} \leq v
	\end{cases}
\end{equation*}
Indeed, all the rows played with positive probability must be optimal (they must provide as outcome the value of the game) while all the rows played with probability equal to zero must be suboptimal (they must provide an outcome lower than $v$, otherwise there is no reason not to play them).

\bigskip
\noindent \textbf{Note:} in general, some of the probabilities $x_k$ and $y_k$  
are equal to 0. In particular, a probability is zero if the outcome 
of the corresponding line 
is less than the value of the game, because in this case I'm sure I 
would not choose it.

\noindent Consider for example:\\
(1,0)		(0,3)\\
(0,2)		(1,0)\\
(-2,-2)		(-2,-2)\\
It is exactly the same as before, since of course the third row is 
strictly dominated by the others. The third row is played with 
probability zero (it is not optimal), which means that we can write 
that $q+0(1-q) ~=~ 0q+(1-q) \geq -2q-2(1-q)$.

\noindent The problem is that the value of the game is unknown and that 
we do not know in advance which are the rows/cols that we play with 
zero probability.

\bigskip
\noindent \textbf{Fully mixed equilibria:} there are not rows/cols that 
we play with zero probability $\implies$ the outcome of all the rows (and, separately, for all the cols) must be equal (: if it were less, we 
would not play that row).

\bigskip
\noindent \textbf{Brute force algorithm:} ipotizziamo che, ad esempio, le prime 
tre righe siano ottime; calcoliamo le rispettive probabilit� (fissando 
a zero quelle relative alle altre righe) e, a posteriori, verifichiamo 
che le altre righe non siano ottime (che abbiano un outcome inferiore). 
Se questo non si verifica, ripartiamo ipotizzando che le righe ottime 
siano altre.

\noindent \textbf{Problem:} the computational cost of such a procedure is $(2^n-1)(2^m-1)$, which is prohibitive for values of $n$ and $m$ large enough.

\section*{General Strategic Games}

\noindent Consider an $n$-players game with strategy sets $X_i$ and payoffs $f_i: X \rightarrow \mathbb{R}$, with 
$X = \prod_{i=1}^n{X_i}$.

\noindent If $x = (x_1,...,x_{i-1},x_i,x_{i+1},...,x_n)$ is a strategy profile, we define the vector $x_{-i} = (x_1,...,x_{i-1},x_{i+1},...,x_n)$, 
so that we can write $x = (x_i, x_{-i})$.

\noindent A vector $\bar{x} = (\bar{x}_i)_{i=1}^n$ is a \textbf{NEp} for a general strategic game if $\bar{x}_i \in BR_i(\bar{x}_{-i}) ~\forall i=1:n$.

\bigskip
\noindent \textbf{Nash theorem for general strategic games:} given a general strategic game with $n$ players s.t. $X_i$ are the strategy sets and $f_i: X \rightarrow \mathbb{R}$ are the payoffs, with $X = \prod_{i=1}^n{X_i}$, if
\begin{enumerate}
	\item each $X_i$ are closed bounded convex sets in $\mathbb{R}^{d_i}$
	\item $f_i$ are continuous functions
	\item $x_i \mapsto f_i(x_i,x_{-i})$ are quasi-concave functions for each $x_{-i} \in X_{-i}$
\end{enumerate}
then the game admits at least one NEp.

\subsection*{Finite Strategic Games}

\noindent Consider an $n$-players finite game with strategy sets 
$A_i$ and payoffs given by $f_i(a_1,...,a_n)$.

\noindent In the mixed extension every player selects a probability distribution $x^i \in \Sigma_{A_i}$, i.e. $x_{a_i}^i \geq 0 ~\forall a_i \in A_i$ and s.t. $\sum_{a_i \in A_i}{x_{a_i}^i} = 1$.

\noindent Denote by $A = \prod_{i=1}^n{A_i}$ the set of pure strategy profiles. The probability of having an outcome $(a_1,...,a_n) \in A$ is the product $\prod_{i=1}^n{x_{a_i}^i}$ and the expected payoffs are:
\[
	\bar{f}_i(x^1,...,x^n) = \sum_{(a_1,...,a_n)\in A}{f_i(a_1,...,a_n)
	\prod_{j=1}^n{x_{a_j}^i} = \sum_{a_i \in A_i}{x_{a_i}^iu_i(a_i,x^{-i})}}
\]
where
\[
	u_i(a_i,x^{-i}) = \sum_{a_i \in A_i, j \neq i}{f(a_1,...,a_n)\prod_{j \neq i}{x_{a_j}^j}}
\]

\bigskip
\noindent \textbf{Corollary:} every n-players finite game has at least one Nash equilibrium profile in mixed strategies.

\bigskip
\noindent \textbf{Examples:} 

\begin{enumerate}
	\item \textbf{The Braess paradox}
	
	\noindent The Nash equilibrium of Braess paradox is $\frac{1}{2}$ on the 
	north path and $\frac{1}{2}$ on the south path. Indeed, if one person 
	moves from one path to the other, immediatly the time of the second 
	path increases because the number of people increases. Therefore, there 
	is not convenience in changing path.
	
	\noindent Now what happens if someone opens a path connecting the north 
	and the south street? For every player, to go from Turin to north and 
	then to the south path is strictly dominant wrt going from Turin to 
	south. Indeed, if all the people go to north the time is $\frac{4000}{100}$, 
	which means that the path T $\rightarrow$ north $\rightarrow$ south costs at most 45, 
	which is definitely less than 50 (the cost of T $\rightarrow$ south).
	
	\noindent Adding strategies to the players can be harmful to all of them. 
	The players cannot ignore the new street, but this street creates problems 
	to everyone (the overall time of T $\rightarrow$ north $\rightarrow$ south $\rightarrow$ M is greater 
	than the time we had when there was not path from north to south).

	\item \textbf{El Farol bar}
	
	\bigskip
	\noindent \textbf{Pure coordination game:} the utility function is the 
	same for all the players.
	
	\noindent The unique Nash eq. in pure strategies is the trivial eq. "300 
	people go to the bar and 200 stay at home". Of course, it is unique 
	except for permutation, in the sense that the 300 people in the bar are 
	chosen at random.

	\noindent Problem: what actually happens in a situation like this, is that 
	people makes experiments and try to converge to a mixed Nash eq.

	\noindent Example:\\
	(7,2)		(0,0)\\
	(6,6)		(2,7)\\
	Here there are two pure Nash eq., but mixed Nash eq. are in any case 
	interesting. Pure Nash eq. are less natural to see in real situations 
	because they are less symmetric wrt the outcome of the players.

\end{enumerate}

\section{Duopoly}

%\noindent {\huge you cannot indent}

\noindent Two firms choose a quantity of a certain good they want to produce. Firm1 chooses the quantity $q_1$ while firm2 chooses the quantity $q_2$.

\noindent A quantity $a > c$ of that good, where $c$ is the unitarian cost for both firms, saturates the market.

\noindent The utility functions for the two firms are 
\[u_1(q_1,q_2) = (a-q_1-q_2)q_1 - cq_1\]
and
\[u_2(q_1,q_2) = (a-q_1-q_2)q_2 - cq_2\]
which are income minus cost of production. 
The price on the market is $p = \max{(a-q_1-q_2,0)}$, where $a$ is called the quantity 
that clears the market. If I produce a very little amount $q_1$ and $q_2$, 
the actual price is almost $a$, while it is almost zero if I produce $a$.

\noindent I can write: 
\[u_1 = -q_1^2+(a-c-q_2)q_1\]
and 
\[u_2 = -q_2^2+(a-c-q_1)q_2\]

\subsection{Monopoly}

\noindent Let's suppose that $q_2=0$, so that I 
have to maximize $-q_M^2+(a-c)q_M$ (M stands for Monopoly).

\noindent
I compute the derivative $-2q_M+a-c=0$ and finally I get: 
$q_M = \frac{a-c}{2}$, $u(q_M)=\frac{(a-c)^2}{4}$ and the price is 
$p_M = \frac{a+c}{2}$.

\subsection{Duopoly}

\noindent Now we can consider the duopoly case. We compute the partial 
derivatives of $u_1$ wrt $q_1$ and of $u_2$ wrt $q_2$. We get a symmetric 
system so that $q_1=q_2=\frac{a-c}{3}$, $u_i(q_1,q_2)=\frac{(a-c)^2}{9}$ and 
the price is $p_D=\frac{a+2c}{3}$.

\subsection{Leader \& Follower}

\noindent Now consider the case in which the two forms are not symmetric: 
there is a leader which has the possibility to decide the strategy, while 
the other player takes the strategy as granted and tries to maximize it.

\noindent In particular, the leader announces $q_1$ and the follower 
maximizes $-q_2^2+(a-c-q_1)q_2$, getting $q_2(q_1)=\frac{a-c-q_1}{2}$.

\noindent Given this, which is known to everybody, the leader maximizes 
$-q_1^2+(a-c-q_2)q_1=-q_1^2+\left(a-c-\frac{a-c-q_1}{2}\right)q_1$.

\noindent This means that he maximizes $-\frac{1}{2}q_1^2 + 
\frac{a-c}{2}q_1$, getting $q_1=\frac{a-c}{2}$.

\subsection{Comparison}

\noindent In the duopoly case, we have a greater quantity and also a 
greater utility (if in the duopoly case we consider the sum of the 
utilities of the producers). The situation leader-follower is the best 
for the consumer: the quantity is bigger, the price is smaller. 

\noindent An example of leader-follower situation is backward induction: 
it can always be applied when the game is sequential and the information 
is perfect. Here we do not have a finite number of options, but the 
utility function is continuous and the set is convex, which means that 
we can always find a maximum.

\bigskip
\noindent \textbf{Note:} We can find the equilibrium of the duopoly case 
also with elimination of strictly dominated strategies (by applying it 
infinitely many times). Starting from the derivative of $u_1$, we look for 
the values of $q_1$ s.t. $-2q_1+a-c-q_2 \leq 0 ~\forall q_2$ and we 
find $q_1 \geq \frac{a-c}{2}$. This means that the quantity 
$\bar{q}_2=\frac{a-c}{2}$ strictly dominates all strategies with 
$q > \frac{a-c}{2}$. Therefore we can go on by reducing the interval in 
which I can choose the amount of production.

\noindent Now we have $-2q_2+a-c-q_1 \geq 0 ~\forall q_1$ iff 
$\frac{a-c}{4} \geq q_2$, which means that this values dominates all $q_2$.

\noindent Because of symmetry, we have that we must stay in the 
interval $\left[\frac{a-c}{4},\frac{a-c}{2}\right]$. By proceeding in 
the same way, starting from this interval, we finally get the point 
$\frac{a-c}{3}$.

\section{Finite games with common payoffs}

\noindent Consider a finite game with strategy sets $X_i$ and suppose that all the players have the same payoff $p: X \rightarrow \mathbb{R}$, i.e. $u_i(x_1,...,x_n) = p(x_1,...,x_n)$.

\noindent Let $\bar{x} \in X$ be a strategy profile s.t. $p(\bar{x}) \geq p(x)$ for all $x \in X$. Then $\bar{x}$ is a NEp in pure strategies.

\noindent They are pure coordination games, i.e. the utility functions of 
the players are exactly the same.

\noindent In this case, I always have a Nash eq. in pure strategies (which 
is interesting because to find a Nash eq. in pure strategies has 
polynomial complexity).

\noindent Iot find the equilibrium, you start from any entry and you ask 
to all the players if they are happy with the current outcome. Since all 
the players have the same utility function, at any step a player improves 
his situation, improving at the same time the outcome for all the players. 
There are no cycles, which means that at worst in a number of steps equal 
to all the possible entries you reach the maximum.

\bigskip
\noindent \textbf{Payoff equivalence:}

\noindent Consideriamo un gioco $(X,Y,f,g)$. Invece di considerare $f(x,y)$, 
potrei considerare $f(x,y)+c_1$ (e lo stesso con $g(x,y)+c_2$) senza 
modificare la "best reaction multifunction". Per di pi�, dato che, per il 
giocatore 1, y � fissata, potrei addirittura modificare la sua funzione di 
utilit� considerando $f(x,y)+c_1(y)$, l'importante � che $c_1$ non 
dipenda da x.

\bigskip
\noindent \textbf{Definition:} two payoffs $u_i$ and 
$\tilde{u}_i$ are said to be \textbf{diff-equivalent} for player $i$ if $u_i(x_1,...,x_n) - \tilde{u}_i(x_1,...,x_n) = c_i(x_{-i})$, i.e. the difference does not depend on his decision $x_i$, but possibly on the decisions of all the other players. 

\bigskip
\noindent In any case I can find a "diff-equivalent" utility function for 
two players, \textit{de facto} I can find an utility function which holds 
for both, which means that I can consider the game as a game with common 
payoff.

%\noindent Consideriamo l'esempio della slide 49: nella matrice ridotta, 
%i due massimi sono 4 e 4, il che significa che i due elementi corrispondenti 
%nella matrice originale sono Nash eq. Di fatto, esiste un altro Nash eq. 
%(l'elemento (3,4) nella seconda riga), che per� � soltanto un ottimo 
%locale, non un ottimo globale nella matrice ridotta.
%\noindent It is easy to find Nash eq in zero-sum games because they are 
%related to solutions of linear programming.

%\end{document}

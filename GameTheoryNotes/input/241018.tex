%\documentclass[pt11,a4paper,twoside,reqno,openright]{paper}
%\usepackage[latin1]{inputenc}
%\usepackage{amsmath}
%\usepackage{amsfonts}
%\usepackage{amssymb}
%\usepackage[english]{babel}
%\usepackage{subfigure}
%\usepackage{graphicx}
%\usepackage{float}
%\usepackage{listings,lstautogobble}
%\usepackage[T1]{fontenc}

%\begin{document}

%!TeX root = ./GameTheoryNotes.tex

\noindent It is easy to find Nash eq in zero-sum games because they are 
related to solutions of linear programming.

\bigskip
\noindent \textbf{Slide 36}

\noindent When I say that $u$ depends on $x_{-i}$, I'm saying that $u$ 
depends on possibly all the variables but $x_i$.

\noindent This is reasonable because when I maximize wrt one variable, I 
take all the others as fixed $=>$ the constant through which I can modify 
the payoffs without modifying the game can depend on all the variables, but 
the one wrt which I am maximizing my payoff.

\noindent The difference between payoffs is a constant wrt the strategy 
of player $i$, even if it depends on the strategies of all the other 
players. If this happens, we are not changing best responses and therefore 
the equilibria $=>$ we can define the two utility functions as 
\textbf{diff-equivalent}.

\bigskip
\noindent \textbf{Slide 37}

\noindent \textbf{Remember:} when we compute the potential, we can fix 
the rows and look at payoffs of the second player (i.e. compute the 
difference between values on the cols) or viceversa...not mixing the 
things!

\noindent \textbf{Remeber:} using the potential we recover the pure Nash 
equilibria.

\bigskip
\noindent \textbf{Slide 39}

\noindent $o_i$ are different origins and $d_i$ different destinations. 

\noindent Some arcs are sensitive to the number of people travelling onto 
the paths themselves.

\noindent Routing games are potential games. $r_i$, the strategy of player 
$i$, is a sequence of compatible arcs.

\bigskip
\noindent \textbf{Slide 41}

\noindent An example to understand the previous slide ;)

\noindent Computation of the potential: for any arc, it is the sum, 
from 1 to the total number of players passing through that arc, of the 
cost $t_a(k)$. We have $p(r_1,r_2,r_3)~=~\sum_{k=1}^{n_a}{t_a(k)~=~...}$
\begin{enumerate}
	\item[a] only one person (player1) passes through the arc $a$, 
		therefore\\ $t_a(1)~=~10$
	\item[b] two people pass through $b$, therefore we have $t_a(1)$, 
		which is 4, plus $t_a(2)$, which is 5
	\item[c] no players pass through the arc $c$, therefore we have 0.
\end{enumerate}

\noindent For all the other arcs, since only one player is using them, we 
have only one term. If we had, for example, two players passing through 
$d$, then we would have $t_a(1)+t_a(2)~=~2+2$, since $t_d$ is independent 
on the number of players passing through $d$.

\noindent Utilities (costs) for all the players:
\begin{enumerate}
	\item he is the only player passing through arc $a$, therefore he 
		pays 10:\\ $u_1(r_1,r_2,r_3)~=~10$.
	\item there are two players passing through arc $b$, thus the time 
		spent on that arc is 5: $u_2(r_1,r_2,r_3)~=~2+5+2$.
	\item $u_3(r_1,r_2,r_3)~=~1+5+1$.
\end{enumerate}

\noindent If now we consider player2 and we compute the difference between 
the potential and the utility function of player2, we have: 
\[
	p(r_1,r_2,r_3)~=~u_2(r_1,r_2,r_3)~=~...
\]
we erase 5 from the arc $b$, 2 from the arc $d$ and 2 from the arc $f$, which 
means that we are left exactly with the potential we would have if we had 
only players 1 and 3 $=>$ the difference between $p$ and $u_2$ only depends 
on $r_1$ and $r_3$, which is what we expect from a potential.

\bigskip
\noindent \textbf{Slide 42}

\noindent Congestion games are based on the assumption that a group of 
players have to use a set of resources. The strategy for player $i$ is to 
select a certain amount of resources he has to use. The cost of a resource 
depends on the number of player that want to use it and each player only 
pays for the resources he is using.

\noindent \textbf{Note:} routing games are an example of congestion games 
where the resources coincide with the paths from an origin to a destination.

\noindent The potential is the sum over all the resources of the sum between 
1 and the number of players which use a certain resource of its cost. As in 
the case of routing games, if we erase from the sum the contribution of the 
player $i$, we are left with the sum extended to the resources that are used 
by all the players except of $i$.

\bigskip
\noindent \textbf{Slide 43}

\noindent In Network connection games we need to build a network that connects 
some nodes, knowing that we would have to pay a certain cost iot build an 
arc (: to create a connection).

\noindent In the routing games, if we use the same path we damage each other, 
while here we help each other because we share the building cost of the 
arc $=>$ the utility is the cost of an arc divided by the number of people 
who are using it.

\noindent In the formula of the potential, if you erase one player you go in 
the sum from 1 to $n_a-1$, which means that, as usual, the potential does not 
depend on the last player but only on the others.

\bigskip
\noindent \textbf{Slide 44}

\noindent Many people have to create a connection with $d$. They can directly 
connect themselves to $d$, paying the usual $\frac{1}{n}$, or they can connect 
for free to an older node $b$, knowing that the cost from $b$ to $d$ is 
$1+\epsilon$.

\noindent If all the players choose the path $o_n \rightarrow b \rightarrow 
d$, each one of them pays $\frac{1+\epsilon}{n}$. Instead, if they go directly 
to $d$, player $n$ pays $\frac{1}{n}$.

\noindent Are there Nash equilibria? And is there a potential?

\noindent Actually, this situation is exactly modeled by a routing game.

\bigskip
\noindent \textbf{SKIP LOCATION GAMES}

\bigskip
\noindent \textbf{Slide 47}

\noindent By analogy with the usual potential, we can expect that it is 
not unique: if there exists a potential, then there are infinitely many. 
In general, what is important is not the potential, but the difference 
between potentials. In physics, potentials are defined up to a constant.

\noindent In game theory, potential functions are defined up to a constant 
exactly as in physics. In particular, if I fix any strategy profile $\bar{x}$ 
and I set $p(\bar{x})~=~0$, then the potential at any other point is 
uniquely determined.

\noindent In lines 1 to n, I keep fixed every player except for player $j$ 
and I compute the difference between $p(\bar{x_1},...,x_j,...,x_n)$ and 
$p(\bar{x_1},...,\bar{x_j},...x_n)$.

\noindent Since all the other variables are fixed, the difference is given 
exactly by the difference between utility functions of player $j$.

\noindent By summing between 1 and $n$, on the rhs we find 
$p(x_1,x_2,...,x_n)-p(\bar{x_1},\bar{x_2},...,\bar{x_n})$, which is 
$p(x_1,x_2,...,x_n)$ since we have fixed $p(\bar{x})~=~0$.

\bigskip
\noindent \textbf{Slide 48} 

\noindent I can always write the sum on the lhs in the previous slide, but 
this does not mean that that sum is always a potential. IF the game admits 
a potential, then I can say that the potential is defined by that sum. On 
the other hand, if the sum defined at the lhs is independent on the order 
used, then I can say that it is a potential.

\bigskip
\noindent \textbf{Slide 49}

\noindent In order to build a potential, we usually fix the value 0 at one 
point. In particular, we fix $p_{1,1}~=~0$, i.e. we set to zero the first 
element of the matrix.

\noindent Having fixed the value of zero at north-west corner, I compute 
the first row of the potential. In particular, since I'm computing a row, 
I have to look at the differences for the second player.

\noindent Given the first row, I build the potential for the first player, 
looking at the columns.

\noindent Once I have computed the matrix, I'm sure that, if the potential 
exists, is the one I have written. However, I still cannot be sure that the 
potential exists: I have to check that the differences in all the rows but 
the first are fulfilled for player2, since I wrote them for player1, while 
I have to check that the differences in the first row are fullfilled for 
player1.

\bigskip
\noindent \textbf{Slide 50}

\noindent One of the main features of cooperative game theory is that it 
shows that in order to survive people have to aggregate, but that, once they 
aggregate, often they go in the opposite direction wrt social optima. An 
example is the Prisoners' dilemma: socially (: for the players), 
for them it is better not to confess (they would spend one year in jail), 
but this is in contrast with the elimination of strictly dominated strategies, 
then at the end they confess and they spend 5 years in jail.

\noindent It is useful to be able to quantify the "badness" of the outcome 
of a game. We need to specify what do we mean with "good" and "bad" and 
this can be done in different ways.

\noindent \textbf{Note:} we apply this theory to potential games, then we 
consider cases in which we want to minimize a cost (not to mazimize a 
positive outcome).

\bigskip
\noindent \textbf{Slide 51}

\noindent Since the maximum is always greater than the minimum, $PoA \geq PoS$.

\noindent If the price of anarchy is very high, this means that anarchy 
(: to apply individual rationality without taking care of the society) is 
very costly. 

\noindent $PoS$ is defined by assuming that, among Nash equilibria, players 
are able (and would) choose the one which is better for the society, i.e. 
the one whose distance from the optimum is smallest. On the other hand, $PoA$ 
is defined assuming that people do not care about the optimum for the society.

\noindent $PoA$ is a pessimistic estimator, while $PoS$ is optimistic.

\bigskip
\noindent \textbf{Slide 52}

\noindent We need to specify which is the function $C$ iot gave a meaning 
to the previous definitions. One of the more common definition is the 
egalitarian function. It is called like this because all the players' costs 
are treated in the same way: it doesn't matter if player1 takes 10 minutes 
to reach his destination and player2 takes one hour, these values are 
summed up without weights when we compute the social cost of the travels.

\noindent Another, less used, possibility is to define $C$ as the max 
instead of the sum: I consider people who are less lucky and I try to 
minimize their misfortune.

\bigskip
\noindent \textbf{Slide 53}

\noindent $n$ people have to move from $o$ to $d$ and they can choose either 
the path $a$ or the path $b$. The social optimum is 1. 

\noindent If all the players go to $b$, each one of them pays $\frac{1}{n}$ 
and therefore the social cost is
\[
	C_1~=~\sum_{k=1}^n{\frac{1}{n}}~=~1
\]

\noindent On the other hand, if all the players choose the path $a$, each one 
of them pays 1 (they have to share the cost $v_a=n$ among $n$ people) and 
therefore the social cost is 
\[
	C_2~=~\sum_{k=1}^n{1}~=~n
\]

\noindent Finally we have $PoS~=~\frac{C_1}{Opt}~=~1$ and 
$PoA~=~\frac{C_2}{Opt}~=~n$.

\noindent The point is: if by chance all the people choose the path $a$, you 
cannot convince them to deviate.

\bigskip
\noindent \textbf{Slide 54}

\noindent You recover the Nash equilibrium by elimination of strictly dominated 
strategies. Indeed, consider for example player $n$: if he chooses the path 
$bd$, the minimum he can pay is $\frac{1+\epsilon}{n}$, if also all the other 
players choose the same path. On the other hand, if he connects directly to 
$d$ he pays $\frac{1}{n}$, which is better. Therefore, player $n$ connects 
directly to $d$. This choice is known to player $n-1$, who have to choose 
between $\frac{1+\epsilon}{n-1}$ and $\frac{1}{n-1}$. Knowing that player $n$ 
goes directly to $d$ and therefore, in the best situation, only $n-1$ players 
go through $bd$, also for player $n-1$ the strategy $bd$ is strictly 
dominated. The same happens to all the players, therefore to connect directly 
to $d$ is the unique Nash equilibrium.

\bigskip
\noindent \textbf{Slide 55}

\noindent It is difficult, in general, to compute $PoS$ and $PoA$.

\noindent In the proof, I consider the minimum because I'm considering 
costs and not positive outcomes.

\noindent The chain of inequalities holds for every $x$, then I can say 
that it holds also if I consider $C(x)~=~Opt$, which means that 
$\frac{1}{\alpha}C(\bar{x})~\leq~\beta Opt$ and therefore 
$C(\bar{x})~\leq~\alpha \beta Opt$.

\bigskip
\noindent \textbf{Slide 57}

\noindent What happens if I consider utilities instead of costs? I still 
want $PoA$ to be very high when I am in a bad situation.

%\end{document}

\documentclass[pt11,a4paper,twoside,reqno,openright]{paper}
\usepackage[latin1]{inputenc}
\usepackage{amsmath}
\usepackage{amsfonts}
\usepackage{amssymb}
\usepackage[english]{babel}
\usepackage{subfigure}
\usepackage{graphicx}
\usepackage{float}
\usepackage{listings,lstautogobble}
\usepackage[T1]{fontenc}

\begin{document}

\noindent {\huge \textbf{Cooperative1}}

\bigskip
\noindent \textbf{Note:} the fact that we are speaking about cooperative 
games does not change the assumptions on the players: they decide to 
cooperate not because they are generous, but only because they, being 
egoistic, think that something good for them can happen if they cooperate. 
The main difference between cooperative and non-cooperative game theory is that 
in the first one we assume that agreements are binding (: vincolanti).

\noindent History of game theory models:
\begin{enumerate}
	\item Zero-sum-games (J.V.Neumann), which do not need the concept of 
		Nash equilibria and use optimal strategies and conservative 
		values.
	\item Cooperative theory (J.V.Neumann), which was developed especially 
		to study non zero-sum games. Indeed, the main feature of 
		constant sum games is that, if you move from an outcome to 
		another which is better, for the other player this is a 
		problem, because if somethig is better for you, than it is 
		worse for me. This is not necessarily true in general games 
		$=>$ players can decide to cooperate because cooperation can 
		be useful.
	\item Non-cooperative theory (Nash)
\end{enumerate}

\bigskip
\noindent \textbf{Slide 3}

\noindent V is a multifunction defined on the power set of N (the set of all 
its subsets).

\bigskip
\noindent \textbf{Slide 4}

\noindent In TU games, I assume that utilities are expressed by some value, 
for example euros.

\bigskip
\noindent \textbf{Slide 5}

\noindent Value of player1, if he is alone, is $a$, because he owns the 
good and he evaluates the good exactly $a$. Values of player2 and player3, 
if they are alone, is 0 because the do not own the good.

\noindent Value for the coalition $\{1,2\}$ is $b$ because they can make an 
agreement and, since player2 evaluates the good $b$, player1 would not 
sell it for less than $b$.

\noindent The value for the "grand coalition" N is $c$, which is the maximum 
value of the good.

\bigskip
\noindent \textbf{Slide 6}

\noindent What it is useful is to make pair of gloves $=>$ the value of a 
coalition is the number of pairs that can be produced if the players share 
the gloves. For example, if there are two peoples with a right glove and 
two people with a left glove, only two pairs can be produced $=>$ the value 
of the coalition is the minimum between the number of left and right gloves 
inside the coalition.

\noindent This is a good model for all the situations in which you produce 
something that has no value by itself, but that needs to be matched with 
somethig else (ad esempio, due persone che producono ciascuna un pezzo 
necessario per costruire un oggetto).

\noindent How the players will form pairs? Who keeps the gloves? These are the 
questions to which cooperative game theory has to give an answer.

\bigskip
\noindent \textbf{Slide 7}

\noindent The value of any coalition A is 1000 if its cardinality is greater 
than 2 (i.e. the coalition contains at least twice the same name). It is zero 
if all players vote for themselves.

\bigskip
\noindent \textbf{Slide 8}

\noindent In this case, the value of the coalition is not an utility; the 
value 1 means that the coalition wins.

\noindent \textbf{Note:} weighted majority games are in general quite easy to 
solve, while this is not true for cooperative games, since the number of 
coalitions that can be formed is huge (it grows exponentially wrt the number 
of players).

\noindent We set to 1 the wheights of all the non permanent players. We need 
to choose $w_1,...,w_5$. We need the vote of all the permanent members iot 
enhance a resolution $=>$ we cannot substitute a permanent player by 6 non 
permament players. Four non permanent player are needed to form a 
winning coalition, the other 6 are useless $=>$ we should fix $w_j~=~7~
\forall j=1:5$. The quota is $\sum_{j=1}^5{w_j}$ plus the sum of the 
weights of four non permament players, i.e. it is 39.

\noindent We can verify that if we have all the non permanent players and four 
permanent players in the coalition, the resolution cannot pass; indeed, the 
value we have is $7 \times 4 ~+~ 1 \times 10 ~=~ 38 ~<~ 39$.

\noindent Notice that this kind of model can be useful to analyse the power 
of the parties, which is not always trivially correlated to the number of 
members. For example, if we have three parties, two with 49 members and one 
which 2 members, we can observe that any combination of two parties creates 
a majority and therefore the relative power of all the parties is practically 
symmetric, even if one of them has far less members than the others.

\bigskip
\noindent \textbf{Slide 9}

\noindent Suppose that I have an estate of 100 and I have to pay 50 to one 
person, 60 to another and 25 to another (50, 60 and 25 are called "claims"), 
so that the total amount that I 
have to pay is greater than what I have (: this is the situation of 
bankrupticity). I have to decide how to distribute my money. I can see this 
situation as a cooperative game.

\noindent Consider a situation as the one above. I can say that $v(\{1\})~=~15$, 
which means that player1 is sure he will get at least 15. Indeed, player2 gets 
at most 60 and player3 gets at most 25, which means that $100~-~60~-~25~=~15$ 
are left to player1. Analogously, $v(\{2\})~=~25$ and $v(\{3\})~=~0$ (the claims 
of player1 and player2, together, gives more than 100 and I cannot ask player3 
to pay, therefore he will simply get nothing).

\noindent There is another possibility, which is less realistic: all the 
coalitions are convinced to be my friends, therefore they say: "He will give 
us as much as possible, not exceeding the sum of our claims". {\huge a questo 
devo pensare...qualcosa non mi convince fino in fondo}

\bigskip
\noindent \textbf{Slide 10}

\noindent Suppose for example that there are three companies and that the 
first one need an 1km truck, the second one needs a 2km truck, while the 
third one needs a 3km truck. Suppose also that the cost for 1km is $c_1$, 
the cost for 2km is $c_2$ and the cost for 3km is $c_3$, where of course 
$c_1~<~c_2~<~c_3$, but there is a scale so that $c_2~<~2c_1$ and so on.

\noindent How can we fairy share the cost between the companies? Of course 
the sum of the amounts they have to pay is $c_3$ (the greater cost).

\noindent \textbf{Note:} here we consider costs and not utilities! We have to 
remember to reverse all the inequalities we might need to define the game.

\bigskip
\noindent \textbf{Slide 12}

\noindent The value of the coalition $\{1,2\}$ is exactly equal to the value 
of the coalition $\{1,2,5\}$. Indeed, I can sum up the values of two nodes 
only if they are directly connected one to the other; since 5 is not 
connected to 1 or 2, it is meaningless to add it to the coalition, as long 
as we do not add also the node 3. The same happens, for example, with the 
coalitions $\{1,2,4\}$ and $\{1,2,4,5\}$.

\noindent The structure of the game depends on the structure of the tree.

\bigskip
\noindent \textbf{Slide 13}

\noindent Consider for example a game with 3 players; all the possible 
coalitions are
\[
	\{\{1\}, \{2\}, \{3\}, \{1,2\}, \{1,3\}, \{2,3\}, N\},
\]
where $N$ is the coalition made by all the players, i.e. $N~=~\{1,2,3\}$.

\noindent To each one of these coalitions I have to attach a number $v_i$, 
which represents its value $=>$ I can say that the game corresponds to the 
vector of the values $[v_1, v_2, ..., v_{2^n-1}]$, where $2^n-1$ is the 
number of all the possible coalitions.

\noindent I can say that the space of the games is a vector space of 
dimension $2^n-1$. I want to find a basis for this vector space.

\noindent The canonical basis is made by the games which assign to coalition 
$T$ the value 1 if $A~=~T$, 0 otherwise $=>$ there is one winning coalition 
and all the others are losing coalitions $=>$ this basis is not very 
meaningful because there are few games which have this kind of model.

\noindent Iot make this basis more meaningful, we can consider to set $u_A(T)$ 
to 1 when $A \subseteq T$. Indeed, the relation $A \subseteq T$ means that 
all players of $A$ are essential to make the coalition win. 

\bigskip
\noindent \textbf{Slide 14} 

\noindent In the case of additive games, I put together people staying in A 
and the ones staying in B and therefore I simply have to sum their values, 
if the intersection is empty.

\noindent Consider for example the coalition $\{1,2\}$. The global utility 
that I can produce is the sum of the utilities that player1 and player2 
produce when they are alone, which means that to stay together is neutral 
for the players: the do not gain more and they do not gain less than what 
they gain when they are alone (and this happens for all the possible 
coalitions).

\noindent The set of additive games is a vector space (the sum of two 
additive games is still an additive game), which has dimension $n$, where $n$ 
is the number of players. Indeed, what I need to know iot compute the value of 
any coalition is the value of all the players, since the value of a coalition 
is the sum of the values of all the players belonging to it.

\noindent \textbf{Note:} notice that superadditive games are games in which 
to stay together can be better than to stay alone, since the value we may 
get can be greater than the sum of the values we get when we are alone. 
Whenever a game is superadditive, we can assume that players want to stay 
alltogether: it would be meaningless to split in subgroups, since the 
value of the union could be greater than the sum of the single values $=>$ 
each coalition which is proposed to us is more convenient than to stay alone 
and the most convenient coalition is the grand coalition.

\bigskip
\noindent \textbf{Slide 15}

\noindent Convex implies superadditive. Indeed, iot check superadditivity I 
have to consider disjoint coalitions. But if A and B are disjoint, then 
$A \cap B~=~\emptyset$ and, since $v(\emptyset)~=~0$, we are left exactly 
with $v(A\cup B) \geq v(A)+v(B)$, which is the definition of subadditivity.

\noindent \textbf{Example:} the children game is not convex. Indeed, if I 
consider the coalitions $\{1,2\}$ and $\{2,3\}$, their union is the grand 
coalition $N~=~\{1,2,3\}$, whose value is 1, but their intersection is $\{2\}$, 
which is a single element and therefore has a value of 0.

\noindent \textbf{Note:} if our values represent costs rather than utilities, 
superadditivity holds with $\leq$ instead of $\geq$. Indeed, a game is 
superadditive if it is convenient to stay together, which, in this context, 
means that if we stay together we pay less than what we pay if we stay alone.

\bigskip
\noindent \textbf{Slide 16}

\noindent The monotonicity property "$A \subseteq C \implies v(A) \leq v(C)$" 
means that, if I add a player to a winning coalition, it remains a winning 
coalition. This is not always true in real life situations!

\noindent Moreover, I have to assume that at least the grand coalition is 
winning. Indeed, if this is not the case, then all the coalitions must be
losing coalitions because of monotonicity and the game is not interesting 
in practice.

\noindent Simple games are characterized by giving the list of all the 
winning coalitions.

\noindent In a minimal winning coalitions, all players are crucial: if I 
remove just one of them, the coalition is no more a winning coalition 
(\textbf{example:} if I have three parties and one has 45, another 3, 
another 7, the coalition is winning but it is not minimal because, for example, 
the party with 3 is not really needed).

\bigskip
\noindent \textbf{Slide 17}

\noindent \textbf{Remember:} a solution for a non-cooperative game is a list 
of strategies for all the players and, consequently, a list of utilities 
all the players get playing the given strategies. Utilities are in the 
background: what I am interested in is the Nash equilibrium profile of 
strategies, given which I can compute the outcome.

\noindent In cooperative games I only have utilities, which means that the 
solution can only be a distribution of utilities (which does not explain 
how the players reach a certain outcome). The solution is abstract, it is 
not practical as it would happen in non-cooperative games.

\noindent This explains why in cooperative games we have a lot of solution 
concepts, while in non-cooperative games I essentially have only best reaction 
as general concept. Here we can have different ideas that produce different 
solution. 

\noindent A solution concept is a multifunction from the set of the games to 
$\mathbb{R}^n$, which means that I have to give something to players, I am 
not looking to what happens to coalitions: coalitions are information that I 
have to look at, but what really matters is what I give players (because 
they are egoistic; the only reason why they work together is that it is 
convenient).

\bigskip
\noindent \textbf{Slide 18}

\noindent First condition: you cannot assign to a player an utility which is 
smaller than the utility he can get if he stays alone (individual 
rationality: if you want a player to take part in the game, you have to 
guarantee a priori that this would be better than to stay alone).

\noindent Second condition: the leading idea of solution is that the grand 
coalition is {\huge is what?? min 1:16 circa, ma non ho capito lo stesso} 
$=>$ you cannot distribute more than $v(N)$ and, at the same time, it would be 
stupid to distribute less than $v(N)$.

\noindent We can say that the first one is an efficiency condition, while 
the second one is a feasibility condition.

\noindent The situation $v(N) \geq \sum_i{v(\{i\})}$ can happen in general 
games that, for example, are not superadditive. Consider for example the 
situation in which we have three players, $v(\{1\})=2$, $v(\{2\})=3$, 
$v(\{3\})=6$ and $v(\{1,2\})=0$ because player1 and player2 hate each 
other. In this case the game is not superadditive because 
$v(\{1,2\})~<~v(\{1\})~+~v(\{2\})$, but we may have that $v(N)~=~100$, 
which means that, even if player1 and player2 hate each other and they do 
not want to make a pair, they find a great advantage in staying in a group 
with also player3, rather than staying alone.

\bigskip
\noindent \textbf{Slide 19}

\noindent In additive games, it is absolutely the same to stay alone or 
together, therefore each player simply keeps his own utility.

\noindent The imputation solution is an interesting concept but it is not 
complete: indeed, it takes care only of the grand coalition and the coalitions 
made by a single player, but actually there are all the intermediate 
coalitions that we could consider.

\bigskip
\noindent \textbf{Slide 20} 

\noindent Imputation is efficient and accepted by every single player; core 
is efficient and accepted by every coalition.

\bigskip
\noindent \textbf {Slide 21}

\noindent The idea is: player1 should sell his good to player3, since $c>a$ 
and this means that player3 is always able to make a better offer.

\noindent At the same time, player1 should actually sell his good, since the 
other players evaluate it more than him.

\noindent We get that, since it must be $x_1+x_3\geq c$ and $x_1+x_2+x_3=c$, 
it should be $x_2<0$, which is not admissible because utilities are always 
non negative. Therefore it must be $x_2=0$. Now I have $x_1\geq a$ and 
$x_1 \geq b$, which means that, being $b>a$, it must be $x_1\geq b$, and 
so on.

\bigskip
\noindent \textbf{Slide 23}

\noindent The fact that the core can be empty is not strange: if you think 
in term of coalitions, player1 can go to player2 and say: "Vote for me and 
I'll give you 50\%". Player2 would agree, but this is not convenient for 
player3, which could say: "No: vote for me and I'll give you 51\%", which 
for him is convenient because $49>0$. There can never be an agreement.

\end{document}

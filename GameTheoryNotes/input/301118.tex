%\documentclass[pt11,a4paper,twoside,reqno,openright]{paper}
%\usepackage[latin1]{inputenc}
%\usepackage{amsmath}
%\usepackage{amsfonts}
%\usepackage{amssymb}
%\usepackage[english]{babel}
%\usepackage{subfigure}
%\usepackage{graphicx}
%\usepackage{float}
%\usepackage{listings,lstautogobble}
%\usepackage[T1]{fontenc}

%\begin{document}

%!TeX root = ./GameTheoryNotes.tex

\section{Bargaining problem according to Nash}

\bigskip
\noindent Problems are pairs where $d$ is the disagreement point and C is the 
set of possible utilities for the players. We suppose that utilities are given: 
we do not speak about strategies. We suppose that there exists at least one 
possibility to reach an agreement in which both players can have better 
outcome than in the disagreement point. There is no reason to reach an 
agreement which is not efficient.

\bigskip
\noindent \textbf{Slide 22}

\noindent Is it possible to change some of these assumptions? The assumption 
we often may try to change is the third one: in some situations, it does not 
appear fair for all the players. The most popular attempt to change this 
assumption gives another definition of solution (see slide 32).

\bigskip
\noindent \textbf{Slide 32}

\noindent I consider a problem which is described by a closed and convex set. 
The exact shape of the problem is not relevant as long as what I add are 
unefficient solutions (the red line in the picture). I consider a point s.t. 
the intersection between the set of the problem and the cone (green in the 
picture) is only given by the point. If the intersection is not a single point, 
then I take $U_2$.

\noindent We define the utopia point as the point in which both the players get 
the maximum utility (it is usually impossible to reach this point in practice). 

\noindent We take the disagreement point and the utopia point and we consider 
a way to connect them and to take the only efficient distribution of utilities 
which stays on the line that connects $d$ to $U$.

\noindent Is there a way to define a function that computes this solution, 
given any problem? Yes and it is weel-defined because there is one and only 
one possible efficient solution.

\noindent \textbf{Remark:} if the game is symmetric, Nash solution and this 
solution coincide because Nash solution is the only efficient solution on the 
diagonal.

\noindent Kalai-Smorodinski solution does not satisfy the third assumption of 
the Nash theorem, while it fullfills the others: utopia and disagreement do not 
change if you modify the scale of the problem; moreover, the solution is 
efficient and symmetric. Which property must we define in place of the third 
one?

\bigskip
\noindent \textbf{Slide 34}

\noindent Kind of proof of uniquenes $\implies$ do it as an exercise {(\huge "or 
ignore it: it will not affect so much your life" - cit.)}

\bigskip
\noindent \textbf{Some considerations:}

\noindent Why is it useful to know the properties of these functions? 
{\huge Other than to take a good grade at the exam, if the theoretical question 
is related to this...(again cit....oggi stiamo evidentemente dando il meglio 
XD)} 
See examples on the notebook.

\bigskip
\noindent {\huge Social choices}

\bigskip
\noindent \textbf{Slide 2}

\noindent The primitive data for social choice problems are the sets of 
alternatives of the agents. 

\bigskip
\noindent \textbf{Slide 7}

\noindent Anonimity means that I can have a permutation of the choices without 
changing the result, i.e. it is not important if I am labelled as player1 and 
you as player2 or viceversa. When anonimity is not satisfied? For example, 
when I have a tie and I have to decide that, for example, between two players 
with the same score the youngest is the winner (ex. concorsi per assegnazioni di 
150ore). Another example: between two players with the same score, I choose the 
woman because I have to fullfill "quote rosa".

\noindent Iot avoid to kill anonimity, people toss a coin.

\bigskip
\noindent \textbf{Slide 8}

\noindent Neutrality is the same as anonimity, but related on alternatives: you 
cannot distinguish two alternatives if all the cryteria are the same. 

\bigskip
\noindent \textbf{Slide 9}

\noindent Example: suppose that I have 3 alternatives:\\
x 	y 	z (player1)\\
y 	z	x (player2)\\
z	x 	y (player3)\\
and function $f$ selects $x$ as a result. Now suppose that I have another 
profile:\\
x 	z 	y (player1)\\
z 	x 	y (player2)\\
x 	z 	y (player3)\\
function $f$ must select again $x$ because $x$ is ranked better in this second 
case than in the previous one and, since he was winning in the first case, he 
cannot have a worse result than this.

\bigskip
\noindent \textbf{Slide 11}

\noindent Remember: simple majority rule means mutual comparison between 
alternatives. In this case you cannot get a ranking, but a circular argument! 
This means that simple majority rule cannot be extended to more than two 
alternatives.

\bigskip
\noindent \textbf{Slide 12}

\noindent Since the numbers can be totally ordered, in this case I always get 
a ranking.

%\end{document}

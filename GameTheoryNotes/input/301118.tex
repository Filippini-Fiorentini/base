%\documentclass[pt11,a4paper,twoside,reqno,openright]{paper}
%\usepackage[latin1]{inputenc}
%\usepackage{amsmath}
%\usepackage{amsfonts}
%\usepackage{amssymb}
%\usepackage[english]{babel}
%\usepackage{subfigure}
%\usepackage{graphicx}
%\usepackage{float}
%\usepackage{listings,lstautogobble}
%\usepackage[T1]{fontenc}

%\begin{document}

%!TeX root = ./GameTheoryNotes.tex

\section{Bargaining problem according to Nash}

\bigskip
\noindent Problems are pairs where $d$ is the disagreement point and C is the 
set of possible utilities for the players. We suppose that utilities are given: 
we do not speak about strategies. We suppose that there exists at least one 
possibility to reach an agreement in which both players can have better 
outcome than in the disagreement point. There is no reason to reach an 
agreement which is not efficient.

\noindent We forget alternate offers and strategies, we only speak about 
utilities. The set $C$ is the set of all the possible distributions of utilities; 
in principle we do not know how these utilities are obtained.

\noindent Disagreement point $d$: utility if we do not reach an agreement.

\noindent This is a Non Transferrable Utility cooperative game.

\noindent \textbf{Remark:} this approach is possible only with two players.

\bigskip
\noindent \textbf{Set of bargaining problems:}

\noindent Assumptions: the set $\mathcal{C} = \{(C,d)\}$ is s.t.
\begin{enumerate}
	\item $C$ is a closed bounded convex subset of $\mathbb{R}^2$.
	
	\noindent This is obvious: we always take closed sets, otherwise we could 
	miss existence.
	
	\item $d \in \mathbb{R}^2$. 
	
	\noindent Oh, be'
	
	\item $\exists x \in C$ s.t. $x_1 > d_1$ and $x_2 > d_2$.
	
	\noindent This means that players can reach an agreement s.t. their utility 
	is greater than what they get in case of disagreement. If this assumption 
	were not fullfilled, then there would be at least one player for which 
	there is not reason to bargain. Iot have an interesting bargaining, 
	players must have the possibility to meet somewhere.
\end{enumerate}

\bigskip
\noindent \textbf{Solution concept:}

\noindent A solution for the bargaining problem is a function $f: \mathcal{C} \rightarrow \mathbb{R}^2$ s.t. $f[(C,d)] \in C ~\forall (C,d) \in \mathcal{C}$.

\bigskip
\noindent \textbf{Properties of the solution:}
\begin{enumerate}
	\item Invariance with respect to admissible transformations of the utility functions: let $L: \mathbb{R}^2 \rightarrow \mathbb{R}^2$ be the transformation $L(x_1,x_2) = (ax_1+c, bx_2 + e)$, with $a,b > 0$ and $c,e \in \mathbb{R}$. Then $f[L(C),L(d)] = Lf(C,d)$.
	
	\noindent \textbf{Remark:} the transformation $L$ is invertible and its inverse $L^{-1}(y_1,y_2) = \left( \frac{y_1}{a} - \frac{c}{a}, \frac{y_2}{b} - \frac{e}{b} \right)$ is an admissible transformation too.
	
	\item Symmetry: let $S: \mathbb{R}^2 \rightarrow \mathbb{R}^2$ be s.t. $S(x_1,x_2) = (x_2,x_1)$. Suppose moreover that the game $(C,d)$ satisfies $(S(C),S(d)) = (C,d)$. Then $f(C,d) = S(f(C,d))$.
	
	\noindent \textbf{Note:} in a game s.t. $(S(C),S(d)) = (C,d)$, the players are symmetric.
	
	\item Independence from irrelevant alternatives (IIA): given two problems $(A,d)$ and $(C,d)$, if $C \subset A$ and $f(A,d) \in C$, then $f(A,d) = f(C,d)$.
	
	\item Efficiency: let $u,y \in C$ be s.t. $u_1 > y_1$ and $u_2 > y_2$. Then $f(C,x) \neq y ~\forall x \in C$.
\end{enumerate}

\bigskip
\noindent \textbf{Nash bargaining theorem:} There exists one and only one function $f$ satisfying the properties (1)-(4). In particular, if $(C,d) \in \mathcal{C}$, $f(C,d)$ is the point that maximizes the function
\[
	g(u,v) = (u - d_1)(v - d_2)
\]
on the set $C \cap \{(u,v): u \geq d_1, v \geq d_2\}$.

\noindent $\implies$ players must maximize the product of their utilities over the set of the interesting outcomes.

\bigskip
\noindent \textbf{Outline of the proof:} 
\begin{itemize}
	\item $f$ is well defined: $g$ is continuous and $C$ is closed, convex and bounded $\implies$ the maximum exists. Moreover, $g$ is quasi-concave $\implies$ the maximum is unique.
	
	\item $f$ satisfies all the properties (1)-(4).
	
	\item $f$ is the unique function that satisfies the properties. Indeed, let $h$ be another function s.t. (1)-(4) are fulfilled. Symmetry and efficiency imply that $h = f$ on the subclass of symmetric games.
	
	\noindent Consider a generic game $(C,d)$ and apply invariance wrt admissible transformations iot have $d = (0,0)$ and $f(C,d) = (1,1)$. Then $L(C) \subset A = \{(u,v): u,v \geq 0 \wedge u+v \leq 2\}$.
	
	\noindent $(A,0)$ is a symmetric game, then $h(A,0) = f(A,0) = (1,1)$.
	
	\noindent IIA entails $h(L(C),0) = f(L(C),0) = (1,1)$.
	
	\noindent We can apply again the invariance wrt admissible transformations and we recover $h = f$.
\end{itemize}

\bigskip
\noindent \textbf{Remark:} the more risk averse a player is, the less he gets from the bargaining.

\noindent Ideed, consider the case in which two players have to divide a good, so that player1 gets $x$ and player2 gets $1-x$. Their utility functions are $u_1(x)$ and $u_2(1-x)$, with $u_i$ increasing, concave, twice differentiable and s.t. $u_i(0) = 0$.\\
$\implies$\\
$x$ must maximize $g(z) = u_1(z)u_2(1-z)$, then it must be $g'(x) = 0$.\\
$\implies$\\
We have: $\frac{u'_1(x)}{u_1(x)} = \frac{u'_2(1-x)}{u_2(1-x)}$, which means that the two curves $\frac{u'_1(z)}{u_1(z)}$ and $\frac{u'_2(1-z)}{u_2(1-z)}$ must intersect in the unique point with abscissa $x$.

\noindent Suppose now that player2 changes his utility function from $u_2$ to $h \circ u_2$ (he becomes more risk averse), mantaining the same properties, and call $y$ the new quantitiy assigned to player1.

\noindent As before, it must be $\frac{u'_1(y)}{u_1(y)} = \frac{h'(u_2(1-y))u'_2(1-y)}{h(u_2(1-y))}$.

\noindent Since for every $z$ it holds $\frac{u'_2(1-z)}{u_2(1-z)} \geq \frac{h'(u_2(1-z))u'_2(1-z)}{h(u_2(1-z))}$, we have that $y > x$, which means that player1 gets more if player2 is more risk averse.

\bigskip
\noindent \textbf{Question:}

\noindent Is it possible to change some of the assumptions (1)-(4)? The assumption 
we often may try to change is the third one: in some situations, it does not 
appear fair for all the players. The most popular attempt to change this 
assumption gives another definition of solution (see slide 32).

\bigskip
\noindent \textbf{Alternative assumption:}

\noindent I consider a problem which is described by a closed and convex set. 
The exact shape of the problem is not relevant as long as what I add are 
unefficient solutions (the red line in the picture). I consider a point s.t. 
the intersection between the set of the problem and the cone (green in the 
picture) is only given by the point. If the intersection is not a single point, 
then I take $U_2$:
\begin{equation*}
	g_C(x) = \begin{cases}
	y \hspace{.5cm} \text{if the intersection is a single point}\\
	U_2 \hspace{.5cm} \text{otherwise} 
	\end{cases}
\end{equation*}

\bigskip
\noindent \textbf{Utopia point:} $U = (U_1,U_2)$ is the point in which both the players get 
the maximum utility (it is usually impossible to reach this point in practice), i.e. $U_i = \max u_i$ on the set $C \cap \{(u_1,u_2): u_1 \geq d_1 \wedge u_2 \geq d_2\}$.

\bigskip
\noindent We take the disagreement point and the utopia point and we consider 
a way to connect them and to take the only efficient distribution of utilities 
which stays on the line that connects $d$ to $U$.

\noindent Is there a way to define a function that computes this solution, 
given any problem? Yes and it is weel-defined because there is one and only 
one possible efficient solution.

\bigskip
\noindent \textbf{Monotonicity assumption:} let $f: \mathcal{C} \rightarrow \mathbb{R}^2$ be a solution for the bargaining problem. $f$ sarisfies the monotonicity assumption for player1 if for every pair of problems $(C,d)$ and $(\hat{C},d)$ s.t. $U_1(C,d) = U_1(\hat{C},d)$ and $g_C \leq g_{\hat{C}}$ it holds $f_2(\hat{C},d) \geq f_2(C,d)$.

\bigskip
\noindent \textbf{Kalai-Smorodinski theorem:} there exists one and only one solution $f: \mathcal{C} \rightarrow \mathbb{R}^2$ for the bargaining problem satisfying invariance wrt admissible transformations of the utility functions, symmetry, efficiency and monotonicity for both players. $f$ associates to the game $(C,d)$ the efficient point lying on the line that connects $d$ to $U$.

\bigskip
\noindent \textbf{Remark:} if the game is symmetric, Nash solution and this 
solution coincide because Nash solution is the only efficient solution on the 
diagonal.

\noindent Kalai-Smorodinski solution does not satisfy the third assumption of 
the Nash theorem, while it fulfills the others: utopia and disagreement do not 
change if you modify the scale of the problem; moreover, the solution is 
efficient and symmetric. Which property must we define in place of the third 
one?

\bigskip
\noindent \textbf{Remark}

\noindent Kind of proof of uniquenes $\implies$ do it as an exercise {(\huge "or 
ignore it: it will not affect so much your life" - cit.)}

\bigskip
\noindent \textbf{Some considerations:}

\noindent Why is it useful to know the properties of these functions? 
{\huge Other than to take a good grade at the exam, if the theoretical question 
is related to this...(again cit....oggi stiamo evidentemente dando il meglio 
XD)} 
See examples on the notebook.

% Social choices ----> moved to the beginning of 051218.tex
%
%\bigskip
%\noindent {\huge Social choices}

%\end{document}

%\documentclass[pt11,a4paper,twoside,reqno,openright]{paper}
%\usepackage[latin1]{inputenc}
%\usepackage{amsmath}
%\usepackage{amsfonts}
%\usepackage{amssymb}
%\usepackage[english]{babel}
%\usepackage{subfigure}
%\usepackage{graphicx}
%\usepackage{float}
%\usepackage{listings,lstautogobble}
%\usepackage[T1]{fontenc}
%
%\begin{document}

\section{Arrow's Theorem}

\bigskip
\noindent \textbf{Notation:}
\begin{itemize}
	\item [-] $N$: set of agents
	\item [-] $O$: set of outcomes (i.e. set of candidates on which the agents 
	must express their preference), with $|O| \geq 3$.
	\item [-] $L$: set of all possible strict preference orderings over $O$.
	\item [-] $\succ_i$: a preference ordering for the agent $i$.
	\item [-] $W$: social welfare function.
	\item [-] $\succ_W$: preference ordering selected by the social welfare 
	function $W$.

	\noindent In particular, we denote by $\succ_{W_{([\succ'])}}$ the social 
	order selected by $W$ when it receives as input $[\succ']$.
\end{itemize}

\bigskip
\noindent \textbf{Pareto Efficiency (PE) - Unanimity property}

\noindent A social welfare function $W$ is Pareto efficient if for any $o_1, o_2 
\in O$, $\forall~i \in N$, $o_1 \succ_i o_2$ implies $o_1 \succ_W o_2$.

\bigskip
\noindent \textbf{Independence of irrelevant alternatives (IIA)}

\noindent $W$ is independent of irrelevant alternatives if, for any $o_1,o_2 
\in O$ and two preference profiles $[\succ']$ and $[\succ''] \in L^n$, 
$\forall~i \in N$ ($o_1 \succ' o_2$ if and only if $o_1 \succ'' o_2$) implies 
that ($o_1 \succ_{W_{[\succ']}} o_2$ if and only if $o_1 \succ_{W_{[\succ'']}} 
o_2$).

\noindent This means that the selected ordering between two outcomes depends 
only on their relative ordering given by the agents.

\noindent \textit{NOTE:} the fact that we assumed $|O| \geq 3$ depends on the 
fact that the IIA property is not meaningful in the case of only two outcomes.

\bigskip
\noindent \textbf{Non-dictatorship (ND)}

\noindent $W$ is non-dictatorial if it does not have a dictator, i.e. 
$\nexists ~i \in N$ s.t. $\forall ~o_1,o_2 \in O$, $o_1 \succ_i o_2 \implies 
o_1 \succ_W o_2$.

\bigskip
\noindent \textbf{Arrow's Theorem}

\noindent Any social welfare function $W$ that is Pareto efficient and 
independent of irrelevant alternatives is dicatorial.

\bigskip
\noindent \textbf{Proof:}

\noindent The proof of the Arrow's theorem is subdivided into four steps:
\begin{enumerate}
	\item \textit{If every voter puts an outcome $b$ at either the very top or 
	the very bottom of his preference list, then $b$ must be at either the very 
	top or the very bottom of $\succ_W$ as well}.

	\noindent Consider an arbitrary preference profile $[\succ]$ in which every 
	voter ranks some $b \in O$ either at the very bottom or at the very top and 
	assume by contradiction that the assumption above does not hold. Then, there 
	must exist some pair $a,c \in O$ s.t. $a \succ_W b \succ_W c$.

	\noindent Let's modify $[\succ]$ so that every voter moves $c$ just above 
	$a$ in his preference list and let's call this new preference profile 
	$[\succ']$. \textbf{IIA} entails that, if we want $a \succ_W b$ or $b 
	\succ_W c$ to change, we need the pairwise relationship between $a$ and $b$ 
	and/or the one between $b$ and $c$ to change as well. However, since $b$ 
	occupies an extremal position for all the voters, the relative position 
	between $a$ and $c$ can be changed without modifying the one of $b$. Thus, 
	in preference profile $[\succ']$ it is also the case that $a \succ_W b 
	\succ_W c$. By transitivity, we have that $a \succ_W c$. However, in 
	$[\succ']$ every voter ranks $a$ better than $c$ and then \textbf{PE} 
	requires $c \succ_W a$.

	\noindent $\implies$ CONTRADICTION.

	\item \textit{There is some voter $n^*$ who is \textbf{extremely pivotal}, 
	in the sense that, by changing his vote at some profile, he can move a 
	given output $b$ from the bottom to the top of the social ranking}.

	\noindent Consider a preference profile $[\succ]$ n which every voter ranks 
	$b$ at the very bottom of his preference list, while all the other 
	preferences are arbitrary. By \textbf{PE}, also $W$ must rank $b$ last.

	\noindent Now let voters from 1 to $n$ successively modify $[\succ]$ by 
	moving $b$ from the bottom to the top, preserving all the other relative 
	rankings. Denote as $n^*$ the first voter whose change causes the social 
	ranking of $b$ to change. This $n^*$ exists for sure: when all the voters 
	rank $b$ at the top of their preference list, by \textbf{PE} $b$ must be 
	ranked at the top also by the social welfare function.

	\noindent Let's denote by $[\succ^1]$ the preference profile just before 
	$n^*$ changes the position of $b$ and let's denote by $[\succ^2]$ the 
	preference profile just after this change. In $[\succ^1]$, $b$ is at the 
	bottom in $\succ_W$. In $[\succ^2]$ $b$ has changed its position in 
	$\succ_W$. Since also in $[\succ^2]$ all the voters rank $b$ either at the 
	top or at the bottom of their preference lists, by \textbf{step1} $b$ must 
	be ranked at the top in $\succ_W$.

	\noindent $\implies$ Q.E.D.

	\item \textit{$n^*$ is a dictator over any pair $a,c$ not involving $b$}

	\noindent First of all, we choose the element $a$ from the pair $a,c$ (of 
	course, we can choose it without loss of generality). Then, we construct a 
	new preference profile $[\succ^3]$ from $[\succ^2]$ by making the following 
	changes:
	\begin{itemize}
		\item We move $a$ to the top of $n^*$'s preference list, leaving all 
		the others unchanged, so that $a \succ_{n^*} b \succ_{n^*} c$.
		\item We arbitrarily rearrange he relative rankings of $a$ and $c$ for 
		all the agents but $n^*$, while leaving $b$ in its extremal position.
	\end{itemize}
	In $[\succ^1]$ we had $a \succ_W b$ since $b$ was at the bottom of 
	$\succ_W$. When we compare $[\succ^1]$ and $[\succ^3]$, relative rankings 
	for $a$ and $b$ are the same for all voters. Thus, by \textbf{IIA}, we must 
	have $a \succ_W b$ also in $[\succ^3]$.

	\noindent In $[\succ^2]$ we had $b \succ_W c$, since $b$ was at the very top 
	of the ranking. Relative rankings between $b$ and $c$ are the same in 
	$[\succ^2]$ and $[\succ^3]$, which entails that $b \succ_W c$ in $[\succ^3]$.

	\noindent By transitivity, it must be $a \succ_W c$ in $[\succ^3]$.

	\noindent Now we construct one more preference profile $[\succ^4]$ by 
	changing $[\succ^3]$ in two ways:
	\begin{itemize}
		\item We arbitrarily change the position of $b$ in each voter's ordering 
		while keeping all other relative preferences the same.
		\item We move $a$ to an arbitrary position in $n^*$'s preference list, 
		with the only constraint that $a \succ_{n^*} c$.
	\end{itemize}
	Observe that all voters but $n^*$ have entirely arbitrary references in 
	$[\succ^4]$, while $n^*$'s preferences are arbitrary but for the fact that 
	$a \succ_{n^*} c$.

	\noindent In $[\succ^3]$ and $[\succ^4]$ all agents have the same relative 
	preferences between $a$ and $c$. Thus, since $a \succ_W c$ in $[\succ^3]$ 
	and by \textbf{IIA}, $a \succ_W c$ in $[\succ^4]$. 

	\noindent $\implies$ we determined the social ranking between $a$ and $c$ 
	by assuming only that $a \succ_{n^*} c$ $\implies$ Q.E.D.

	\item \textit{$n^*$ is a dictator over all the pairs $a,b$}

	\noindent Consider an arbitrary outcome $c$. By the argument in 
	\textbf{step2}, there is a voter $n^{**}$ who is extremely pivotal for $c$. 
	By the argument in \textbf{step3}, $n^{**}$ is a dictator over all the pairs 
	$\alpha,\beta$ not involving $c$. Of course, the pair $a,b$ is one of such 
	pairs $\alpha,\beta$. We have already observed that $n^*$ is able to affect 
	$W$'s $a,b$ ranking (for example, when $n^*$ was able to change 
	$a \succ_{W_{([\succ^1])}} b$ in $b \succ_{W_{([\succ^2])}} a$). Hence $n^*$ and 
	$n^{**}$ must be the same agent.

	\noindent $\implies$ Q.E.D.
\end{enumerate}

%\end{document}

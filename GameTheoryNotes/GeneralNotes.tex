\documentclass[pt11,a4paper,twoside,reqno,openright]{paper}
\usepackage[latin1]{inputenc}
\usepackage{amsmath}
\usepackage{amsfonts}
\usepackage{amssymb}
\usepackage[english]{babel}
\usepackage{subfigure}
\usepackage{graphicx}
\usepackage{float}
\usepackage{listings,lstautogobble}
\usepackage[T1]{fontenc}

\begin{document}

\section{General Notes}

\begin{itemize}
	\item[-] \textbf{Number of pure strategies} for player $i$:
		\[
			\prod_{j \in P_i}{\left(number~of~children 
			~at~node~j\right)}
		\]
		where $P_i$ is the set of all the decision steps for 
		player $i$.
	\item[-] \textbf{Mixed strategies} are probability distributions 
		over the pure strategies.
	\item[-] \textbf{Behavioral strategies} are probability 
		distributions over the actions at each decision step.

		\noindent They are meaningful only if the game is given in 
		extensive form.
	\item[-] A mixed strategy and a behavioral strategy are equivalent 
		if they lead with the same probability to the same node.
	\item[-] \textbf{Theorem:} for every behavioral strategy of player 
		$i$, there exists an equivalent mixed strategy for player 
		$i$ iff each information 
		set of player $i$ intersects a path from the root of the 
		tree at most once (i.e. the play is reasonable).

		\noindent \underline{Counterexample:} when a player does 
		not remember his previous choice, a path from the origin 
		intersects the tree in more than one point (for example 
		in two points); in this case, you can find a behavioral 
		strategy which does not have an equivalent mixed strategy.
	\item[-] \textbf{Theorem:} If player $i$ has perfect recall in the 
		game, then for every mixed strategy there exists an 
		equivalent behavioral strategy.

		\noindent \textbf{Note:} we say that a player has 
		\textbf{perfect recall} if each of his information sets 
		intersects a path from the origin at most once and, if 
		you have two different paths bringing to the same 
		information set (let's call it $S$), then these paths must 
		pass through the same information sets at the previous step 
		and you must arrive in $S$, following each one of the paths, 
		performing the same actions.

		\noindent When the information sets are singletons (i.e. 
		single nodes), then all the players have perfect recall.
	\item[-] By combining the two theorems, we have that, if player $i$ 
		has perfect recall, mixed strategies and behavioral 
		strategies are equivalent.
	\item[-] \textbf{Indifference principle:} at the equilibrium of the 
		game, all the strategies that are played with positive 
		probability must give you the same value $v$.

		\noindent For example, if player1 plays $\left(p,1-p\right)$ 
		and player2 plays $\left(q,1-q\right)$ (and all these 
		values are strictly $>~0$, which means that both players 
		choose mixed strategies, not pure ones), then what the first 
		player receives from the first row should be equal to what he 
		receives from the second row, whatever is the choice of the 
		second player (and viceversa). In other words, the outcome 
		which player1 has from the first row when player2 plays 
		$\left(q,1-q\right)$ should be equal to the outcome he has 
		from the second row when player2 plays $\left(q,1-q\right)$ 
		and this outcome must be equal to the value of the game.
\end{itemize}

\end{document}

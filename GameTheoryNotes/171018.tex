\documentclass[pt11,a4paper,twoside,reqno,openright]{paper}
\usepackage[latin1]{inputenc}
\usepackage{amsmath}
\usepackage{amsfonts}
\usepackage{amssymb}
\usepackage[english]{babel}
\usepackage{subfigure}
\usepackage{graphicx}
\usepackage{float}
\usepackage{listings,lstautogobble}
\usepackage[T1]{fontenc}

\begin{document}

\section{The Nash model}

\noindent Summary of our path 'til now:
\begin{enumerate}
	\item Elimination of strictly dominated strategies
	\item Perfect information games in extensive form
	\item Zero sum games: player1 maximizes the function
		$\alpha(x) = \inf_y{f(x,y)}$ and player2 minimizes
		$\beta(x) = \sup_x{f(x,y)}$
\end{enumerate}
In all those cases, the presence of another player is important but not 
decisive: we can analyse the game by ourselves, since we have all the 
information we need.

\noindent However, many situations (ex. prisoner's dilemma) are not 
zero-sum because there is a true interaction between players.

\noindent When we need real interaction it is natural to assume that the 
players make coalitions (remember: players always want to maximize 
their utility function $=>$ they cooperate if it is convenient).

{\huge Then stop kidding Quarteroni $-.-'$ }

\bigskip
\noindent \textbf{Non-cooperative games - slide 3}

\noindent Remark: we speak about 2 players because it is easyer than 
to consider more than 2. This is not a problem in noncooperative games, 
while in cooperative games to consider 2 players is the same as to 
consider a degenerate game, in which actually you don't see anything.

\noindent Notice that a zero-sum-game is a noncooperative game where 
$g=-f$.

\bigskip
\noindent \textbf{Slide 4}

\noindent We're not talking about optimality, only about equilibria.

\noindent Player1: "if I know that player2 will play $\bar{y}$, there 
is no reason to deviate from $\bar{x}$, since it is optimal for me. 
And I can believe that player2 will actually play $\bar{y}$ since, 
knowing that in that case I will play $\bar{x}$, $\bar{y}$ is optimal 
for him"

\noindent Player2: "Viceversa"

\noindent $=>$ overall, we can say that the Nash profile is a joint 
combination of strategies, since the decision of player1 and of 
player2 are strictly correlated. In principle, we can decide both to 
deviate from our decision, but it will never happen that only one 
player decides to deviate.

\noindent Optimal strategy is a concept valid for the single player: I 
have mine and you have yours. Nash equilibrium profile is for both. Example: 
to go to see the movie together is a Nash eq. profile because it is 
the best choice for both if they are together. On the other hand, one 
player has as optimal str. to go to see the movie and the other to 
stay home...even that what they both prefer is to stay together.

\bigskip
\noindent \textbf{Slide 5}

\noindent The new model can be applied to former concept. For example, 
if by strictly dominated strategies we recover a solution that is not 
a Nash eq., then the definition of Nash eq. has some problems. The point 
is: we can state the definition of rationality in a different way, but, 
overall, rationality must be always the same concept.

\noindent Given a strictly dominant str., it is easy to find the Nash 
eq. profile. Indeed,
\[
	f(\bar{x},\bar{y}) \geq f(x,\bar{y}) ~\forall x,~\forall y
\]
of course implies
\[
	f(\bar{x},\bar{y}) \geq f(x,\bar{y}) ~\forall x
\]

\bigskip
\noindent \textbf{Slide 6}

\noindent Example: a mother and her son have to decide whether to 
buy or not to buy an ice-cream.

\noindent Mother's possibilities: to buy or not to buy

\noindent Son's possibilities: to cry or not to cry

\noindent ...this is the question

\noindent Supposing that for the mother not to buy is better and 
supposing that for the son not to cry is better, we have: 
Outcome: (20, -1); strategies: (notBuy, notCry).

\noindent Same game in grategic form:\\
(1,20)	(1,20)\\
(-5,-2)	(20,-1)\\

\noindent We want to find the Nash equilibria for a game which is in 
matrix form. The Nash eq. are the optimal outcomes when we have a fixed 
strategy for both the players $=>$ we find two Nash eq., i.e. $(1,20)$ 
and $(20,-1)$. The second is the same we recovered from backward 
induction, but here we find one more. Notice that the first outcome is 
strange: we arrive there when the mother buys and the son cries, which 
is strange because, looking at the game in extensive form, we see that 
when the mother buys the son does not cry $=>$ the child does something 
which is not optimal...which makes sense because the point is: the 
situation we are talking about is never reached by the game (there is 
no point in crying when you have an ice-cream). This second Nash eq. is 
based on a threat: son says: "I cry" and therefore the mother buys the 
ice-cream. The threat is not really believable: we know that the son 
does not cry if you buy the ice-cream...but this is only a matter of 
announcements.

\noindent In general, the first thing we can do is to look for backward 
induction solutions because these solutions cannot have objections...we 
can look for other Nash equilibria in a second moment.

\noindent Example on the slide:\\
Outcome: (1,0)\\
Strategy for player1: $x=1$\\
Strategy for player2: "say always yes $\forall x$"

\noindent Consider the same example in strategic form and look for Nash 
eq. profile. Claim: $(x,1-x)$ is the result of a Nash eq. profile 
$\forall x$. How is this possible?? Why (0.02, 0.98) should be a Nash eq? 
In this case, str. of player1 is $x=0.02$, while strategy for player2 is 
"yes to $x=0.02$, no to all other x". For both is not convenient to change: 
for player2 because in any case he cannot get more than 0.02 (which is 
the offer), while player1 cannot offer a different value of x because 
player2 says he will refuse all offers $x\neq 0.02$.

\noindent Actually, the more natural strategy for player2 would be 
"yes to $x \leq 2$, no otherwise".

\noindent $=>$ Nash equilibria can be counterintuitive.

\bigskip
\noindent \textbf{Slide 7}

\noindent $g=-f$ in zero-sum-games $=>$ condition (1) says that 
$(\bar{x},\bar{y})$ is a Nash eq. for the game.

\noindent (i) says that $\bar{x}$ is optimal for player1, while (ii) says 
that $\bar{y}$ is optimal for player2 (it minimizes something and 
player2 wants to minimize, since he has to pay).

\noindent We have two optimal solutions and they agree.

\noindent What I actually build in zer-sum-games are Nash equilibria; 
the point is that zero-sum-games are easier to solve because we can use 
linear programming, which we cannot use in general iot recover Nash eq.

\noindent \textbf{proof of the theorem:} suppose that $(\bar{x},\bar{y})$ 
is a Nash eq. Then I can say $f(\bar{x},\bar(y)) \leq f(\bar{x},y)$. But 
then it is $ = \inf_y{f(\bar{x},y) \leq \sup_x{\inf_y{f(x,y)}}} = V_1$.

\noindent We also have $V_2 = \inf_y{\sup_x{f(x,y)}} \leq 
\sup_x{f(x,\bar{y})}$.

\noindent Then $V_2 \leq V_1$...but it is always $V_1 \leq V_2$ and 
therefore we must have $V_1 = V_2$, which means that all the inequalities 
above are actually equalities $=>$ we proved that (1) implies (2).

\noindent On the other side: 
$\sup_x{\inf_y{f(x,y)}} = \inf_y{f(\bar{x},y)} \leq f(\bar{x},\bar{y}) 
\leq \sup_x{f(x,\bar{y})} = \inf_y{\sup_x{f(x,y)}}$

\noindent But we have (2) as hypothesis, which means that all the 
ineq. are equalities, which is exactly the same as to say (1).

\bigskip
\noindent \textbf{Slide 10}

\noindent \textbf{Multifunctions:} The only requirement we make to 
$f: X \rightarrow Y$ iot be a function is that $f(x)$ must be an unique 
element in $Y$. Example: $\sqrt{y^2}$ is not a function in $\mathbb{R}$ 
if we do not write $\sqrt{y^2}=|y|$. We use the word "multifunction" 
when we do not know if something is actually a function.

\noindent $BR_1(y)$: Best Reaction of player1 to a strategy $y$ of player2.

\noindent $(\bar{x},\bar{y})$ is a Nash eq. if $\bar{x}$ is the best 
reaction of $\bar{y}$ and viceversa.

\noindent Remember: $g: Z \rightarrow Z, ~g(z) = z$ is a fixed point.

\noindent The Nash eq. is a fixed point for a multifunction ("multi" 
because the best reaction can be not unique) $=>$ it is enough to 
find a couple which belongs to $BR(\bar{x},\bar{y})$, we do not require $=$.

\bigskip
\noindent \textbf{Slide 11}

\noindent "F has closed graph" means that:\\
$(u_k,v_k) \in Graph(F)$ AND 
$u_k \rightarrow u$ AND 
$v_k \rightarrow v$ 
$=> ~(u,v \in Graph(F))$

\bigskip
\noindent \textbf{Slide 12}

\noindent Quasi-concave means that the upper level sets are convex. Of 
course, a concave function is quasi concave. Example: $\frac{1}{x}$ is 
quasi concave. Indeed, all the upper level sets are half-lines, which of 
course are convex.

\noindent Example: if I'm indifferent to $x,y$, then I can assume that any 
convex combination of x and y is better than anyone of them. In this case, 
I can say that quasi concave functions are useful.

\noindent \textbf{Proof of the theorem:} I want to prove that $BR(x,y)$ is 
not empty, closed, convex $\forall x,y$ and it has closed graph iot use 
the theorem of slide 11. I do the proof for $BR_1$.

\noindent $BR_1(y)$ is nonempty and closed by Weierstrass theorem and 
it is closed by the quasi concavity assumption. I need to prove that 
the graph is closed. I assume that:
\begin{enumerate}
	\item $(x_k,y_k) \in Graph(BR_1)$
	\item $x_k \rightarrow x$ and $y_k \rightarrow y$.
\end{enumerate}

(1) implies that $f(x_k,y_k) \geq f(x,y_k) ~\forall x \in X$, which means, 
by teorema della permanenza del segno, that $f(\bar{x},\bar{y}) \geq 
f(x,\bar{y}) ~\forall x \in X$.

\end{document}

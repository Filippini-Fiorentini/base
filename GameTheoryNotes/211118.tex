\documentclass[pt11,a4paper,twoside,reqno,openright]{paper}
\usepackage[latin1]{inputenc}
\usepackage{amsmath}
\usepackage{amsfonts}
\usepackage{amssymb}
\usepackage[english]{babel}
\usepackage{subfigure}
\usepackage{graphicx}
\usepackage{float}
\usepackage{listings,lstautogobble}
\usepackage[T1]{fontenc}

\begin{document}

\noindent {\huge Repeated games - Correlated equilibria}

\bigskip
\noindent Idea: sometimes the inefficiency of the Nash equilibrium can be 
wickened having some more information.

\bigskip
\noindent \textbf{Slide 3}

\noindent Each player has three strategies. If you forget the last row and col, 
you have a $2 \times 2$ matrix with a strictly dominant strategy that 
provides a very bad result (like in prisoners' dilemma). The game is symmetric 
$\implies$ you can expect a symmetric solution (which cannot be (3,3) because 
of the strictly dominant strategy).

\noindent Last row and col are strictly dominated by everything: I'm adding 
very bad possible outcomes for the players.

\noindent Suppose that the game is played several times, so that there is more 
information (I know what happened in the previous situations). Let's suppose 
that the players play once per day.

\noindent In principle, the last day you have a one-shot play and therefore 
you apply the strictly dominant strategy. Now we can simply apply backward 
induction and we see that we always apply strictly dominant strategy because, 
given the strategy I would play the last day, the "last - 1" day is a 
one-shot game and therefore I apply strictly dominant strategy etc.

\noindent Actually, the point is: are we sure that (1,1) is the only Nash 
equilibrium? Maybe there is another Nash eq. in mixed str.

\bigskip
\noindent \textbf{Slide 4}

\noindent What I'm claimin is that I will get $3-a$ on average, not in every 
single day. 

\noindent We can also prove that it is possible to get even more than this!

\bigskip
\noindent \textbf{Slide 5}

\noindent Remember: this means that in the first $N-k$ days we gain exactly 
3, since both of us play the first strategy. In the last $k$ days, we play the 
strictly dominant strategy, which means that we gain 1.

\noindent Players are threatening each other: if one changes strategy, the other 
plays the last strategy, which is the worst for both (negative outcome).

\noindent I giocatori scelgono di giocare la prima strategia per $N-k$ giorni 
perché hanno bisogno di lasciarsi la possibilità di punire l'altro se cerca di 
fregarli. Infatti, in caso contrario l'ultimo giorno converrebbe giocare la 
seconda strategia, che garantisce al giocatore un outcome di 10, se l'altro 
rimane fedele alla prima strategia. Questo non sarebbe necessario in un gioco 
con infinite partite, perché in quel caso avrei sempre il tempo di punirti se 
cercassi di fregarmi. In giochi finiti, i giocatori sono collaborativi per 
$N-k$ giorni e poi giocano la strategia dominante.

\bigskip
\noindent \textbf{Slide 6}

\noindent I know that you will punish me if I deviate, therefore I should 
cheat you the last day before you change strategy: in this way, I take 3 
until day $N-k-1$, then I cheat and I get 10 for one day and $-1$ for the 
remaining $k$ days.

\bigskip
\noindent \textbf{Slide 8}

\noindent Nash equilibrium is always a rational outcome.

\noindent Observation: the threat is not really credible: % siamo sicuri che
														  % questa parola
														  % esista ?! no
it is based on the fact that I will punish you by also punish myself.

\noindent In Life::real, people use dominant strategies and this is very bad 
for the society...but there are situations in which people actually decide to 
cooperate...even if we do not really know that we will play a certain 
number of times and therefore we will have time to punish each other. This is 
a very unstable equilibrium: if one cheat, then also the others start to cheat 
and the situation becomes even worse than before.

\noindent In one-shot days, we do always have to assume that people use the 
strictly dominant strategy, but they sometimes collaborated if the game is 
repeated.

\bigskip
\noindent \textbf{Slide 9}

\noindent \textbf{Remark:} there must be a pure mixed Nash equilibrium because 
there is not a strictly dominant strategy. The only possibilities in this case 
are one eq. in fully mixed str. and two in pure str. or only one in mixed str.

\noindent The game is symmetric $\implies$ it is more rational for the 
players to arrive to a symmetric output $\implies$ in Life::real, the mixed 
Nash equilibrium is observed more often.

\noindent Under the mixed Nash eq., the probability distribution of the 
outcomes is:\\
$\frac{4}{9}$	$\frac{2}{9}$\\
$\frac{2}{9}$	$\frac{1}{9}$\\

\bigskip
\noindent \textbf{Slide 10}

\noindent Suppose that the players can talk to each other and make agreements.

\noindent I would like to use this probability distribution: why should we 
go, even if with small probability, for the outcome (0,0)? Actually, this 
probability distribution cannot be retreived by using a strategy as: 
$[p,(1-p)]$, because we cannot reach a probability equal to 0. However, the 
outcome that we would have is better.

\noindent Even if this produces a better oucome, this is not possible in 
the usual context in which players have perfect information: suppose that you 
agree to this behaviour: lanciamo un dado; se esce 1 o 2, giochiamo per (6,6), 
se esce 3 o 4, giochiamo per (2,7), se esce 5 o 6, giochiamo per (7,2) (non 
giochiamo mai per (0,0)). Una simile situazione non funziona nel caso di 
informazione perfetta: se esce 1 o 2, so che l'altro giocatore giocherà la 
prima strategia e quindi a me conviene giocare la seconda: 7 è meglio che 6.

\bigskip
\noindent \textbf{Slide 11}

\noindent Per fare in modo che quanto è descritto sopra, dobbiamo dare ai 
giocatori un'informazione parziale, in modo che non sappiano cosa giocherà 
l'altro.

\bigskip
\noindent \textbf{Slide 14}

\noindent The player can be told to play row $\bar{i}$. What I want is that 
to play $\bar{i}$ is better than to play any other row, otherwise I would 
change strategy.

\noindent Analogously, for player2 the best strategy must be column $\bar{j}$.

\bigskip
\noindent \textbf{Slide 16}

\noindent We want to show that, having a Nash eq. profile, we can build a 
correlated equilibria. Since in a game we can also find a Nash equilibrium 
profile, then we can also say that we can always find a correlated equilibrium. 

\noindent We claim that $p_{ij} = \bar{x_i}\bar{y_j}$ is a correlated 
equilibrium.

\bigskip
\noindent \textbf{Slide 19}

\noindent If I tell you to play line $\bar{i}$, the point is that you have to 
agree and to be sure that in any case you would not get anything better if 
you change strategy. This of course cannot be true with strictly dominated 
strategies: how can I convince you to play a strictly dominated strategy?! It 
is always worse than anything else, for you.

\noindent Counterexample for weakly dominated row:\\
(5,5)	(0,5)\\
(5,0)	(1,1)\\
Remark: from this I can see that a Nash equilibrium is still a correlated 
equilibrium. For example, the probability distribution:\\
1	0\\
0	0\\
leads me to the Nash equilibrium (5,5).

\end{document}

\documentclass[pt11,a4paper,twoside,reqno,openright]{paper}
\usepackage[latin1]{inputenc}
\usepackage{amsmath}
\usepackage{amsfonts}
\usepackage{amssymb}
\usepackage[english]{babel}
\usepackage{subfigure}
\usepackage{graphicx}
\usepackage{float}
\usepackage{listings,lstautogobble}
\usepackage[T1]{fontenc}

\begin{document}

\noindent \textbf{Game:} process that can be described by:
\begin{enumerate}
	\item a set $N$ of players s.t. $|N| > 1$
	\item an initial situation
	\item the list of the possible actions of the players
	\item the list of all the possible final situations
	\item the preferences of all the players on the set of the final 
	situations.
\end{enumerate}

\noindent \textbf{Crucial assumptions on the players:} they are
\begin{enumerate}
	\item egoistic: they only care about their own preferences
	\item rational: there are five rationality assumptions that must hold if 
	we want to define a game.
	\begin{enumerate}
		\item The agents are able to provide a \textbf{preference relation} 
		(i.e. either a total preorder\footnote{We call \textbf{total preorder} 
		or \textbf{ranking} a reflexive, transitive and total binary relation 
		over a set X; we call \textbf{total order} a total preorder which is 
		also antisymmetric.} 
		or a total order) over the outcomes of the game.

		\noindent The fact that the preference relation (i.e. the ordering) 
		must be total means that players cannot refuse to make a choice.

		\noindent Notice that two alternatives have the same rate if the player 
		is indifferent to choose among them.

		\item The agents are able to provide an \textbf{utility function} 
		representing their preference relations, whenever necessary.

		\noindent Given a set $X$ and a preference relation $\succeq$ over $X$, 
		an utility function is a function $u:~X\rightarrow~\mathbb{R}$ 
		such that $u(x) \geq u(y) \iff x \succeq y ~\forall x,y \in X$.

		\noindent Notice that an utility function always exists if the set $X$ 
		is finite (: you can always assing a rating to a finite number of 
		alternatives). Moreover, if an utility function exists, then there are 
		infinitely many: utility functions are equivalent up to strictly 
		increasing transformations.

		\noindent In general, we also assume that any player knows the 
		preference relations that hold for the other players and their 
		utility functions (\textbf{complete knowledge}).

		\item Players use consistently the probability laws; in particular, 
		they are consistent wrt the computation of expected utilities and 
		they can update probabilities according to the Bayes 
		rule\footnote{Idea of the Bayes rule is the following: you start to 
		analyse a problem $\implies$ you get some information $\implies$ you 
		update the probabilities associated to different events according to 
		those information $\implies$ etc.}.

		\item The players are able to evaluate the consequences of all the 
		actions and the consequence of this information on any other player.

		\item Players are able to apply decision theory, whenever it is 
		possible, i.e., given a set of alternatives $X$ and an utility function 
		$u$, any player looks for an $\bar{x} \in X$ s.t. 
		$u(\bar{x}) \geq u(x) ~\forall x \in X$.
	\end{enumerate}	
\end{enumerate}

\noindent \textbf{Elimination of strictly dominated actions:} this principle 
holds as a consequence of the assumptions described above. In particular, 
supposing that player1 can choose among actions $x \in X$ and player2 can 
choose among actions $y \in Y$,we say that $x$ is strictly dominated by 
$x^* \in X$ if $u(x^*,y) > u(x,y) ~\forall y \in Y$. The principle states that 
a player does not take an action $x$ if there is an admissible action $x^*$ s.t. 
$x^*$ provides him a strictly better result than $x$, no matter what the other 
players do.

\noindent \textbf{Remark:} we can also state a principle of weakly dominated 
actions, but this is not always acceptable because to keep weakly dominated 
actions can lead to an overall better outcome for the players.

\end{document}

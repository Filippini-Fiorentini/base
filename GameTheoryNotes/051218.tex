\documentclass[pt11,a4paper,twoside,reqno,openright]{paper}
\usepackage[latin1]{inputenc}
\usepackage{amsmath}
\usepackage{amsfonts}
\usepackage{amssymb}
\usepackage[english]{babel}
\usepackage{subfigure}
\usepackage{graphicx}
\usepackage{float}
\usepackage{listings,lstautogobble}
\usepackage[T1]{fontenc}

\begin{document}

\noindent {\huge Social Choice}

\bigskip
\noindent It is a mathematical model that looks for a way to aggregate 
individual preferences in social preferences. We want to derive a good social 
ranking given individual rankings (example: elections).

\noindent Problem: there is not an individual way to define fairness. A property 
that should be fullfilled is that if we agree on a list of preferences, there 
is no reason to change them (unanimity property). There other properties that 
we must agree on iot define a solution for the problem.

\noindent Remember: a social choice function only needs to determine a winner: 
we do not care about the second (example: elezione del Papa).

\bigskip
\noindent \textbf{Slides 13 - 15}

\noindent Properties:

\begin{itemize}
\item Meaning of independence from irrelevant alternatives: suppose that 
people give me a ranking and then they change their mind and they give me a 
different ranking. If, in both the rankings, agent $x$ is ranked better than 
agent $y$. Then I do not care if in the first ranking $x$ is the first and $y$ 
the last, while in the second ranking $x$ is the first and $y$ is the second: 
other agents do not matter, the only important thing is their relative position 
in the ranking. Any time you assign scores to your position (ex. sport: you make 
a mutual comparison between two soccer teams, you assign a certain amount 
of points to the winner or another to a tie and then you sum all the points) 
this assumption is not fullfilled: maybe Milan wins over Inter, but its position 
in the final ranking is worse.

\item Dictator: no matter what all the other individuals say, the ranking 
for the society is decided only by one player. Remember: this is only an 
\textit{a priori} analysis, which means that the society does what I decide, no 
matter what it is.

\noindent This means that the function $F$ turns out to be a projection: it 
selects, among the vector of preferences, the one related to the choice of the 
dictator.
\end{itemize}

\bigskip
\noindent \textbf{Slide 24}

\noindent Example: we are three people and we want to select a tv program to 
watch together. We are only interested to decide who is the winner, we are not 
interested in a full ranking (for the society: of course any player has his 
own full ranking). With more than two alternatives, this is relevant (with only 
two alternatives, once you have the winner you also have the full ranking): we 
are requiring less...are we in the same situation of the Arrow theorem or can 
we get something more? Of course we have to change some properties, for example 
the irrelevant alternative: since I do not have to write a full ranking for the 
society, this property is useless.

\bigskip
\noindent \textbf{Slide 25}

\noindent Properties:
\begin{enumerate}
	\item We are not decreasing the level of ranking of $x$ among different 
	players. Suppose that I have a ranking $\succ$ s.t.\\
	A \hspace{.6cm} C \hspace{.6cm} C\\
	B \hspace{.6cm} B \hspace{.6cm} A\\
	C \hspace{.6cm} A \hspace{.6cm} B\\
	and a function $f$ that selects A as the winner. If I have another ranking 
	$\supset$ s.t.\\
	A \hspace{.6cm} C \hspace{.6cm} A\\
	C \hspace{.6cm} A \hspace{.6cm} B\\
	B \hspace{.6cm} B \hspace{.6cm} C\\
	Then, if $f$ is monotonic, it must select again A as the winner.
\end{enumerate}

\bigskip
\noindent \textbf{Slide 27}

\noindent Suppose that I have a ranking $\succ_1$ s.t.\\
A \hspace{.6cm} C \hspace{.6cm} C\\
B \hspace{.6cm} B \hspace{.6cm} B\\
C \hspace{.6cm} A \hspace{.6cm} A\\
and a function $f$ that selects C as the winner.

\noindent Now suppose that I change my ranking in $supset_1$ getting\\ 
B \hspace{.6cm} C \hspace{.6cm} C\\
A \hspace{.6cm} B \hspace{.6cm} B\\
C \hspace{.6cm} A \hspace{.6cm} A\\
and that now the function $f$ selects B as the winner. This means that the 
function is manipulable: player1 can lie and submit the ranking $\supset$ iot 
get a result (B) which is better for him in the true ranking $succ$.

\noindent Of course, this precise situation cannot happen in Life::real: it is 
based on the assumption that player1 knows the ranking of other players and 
they do not know his ranking (otherwise he could not lie). However, true 
manipulation can occur when I have strong feelings on what you could choose.

\bigskip
\noindent \textbf{Slide 28 - proof of the GS theorem}

\noindent If I prove that non-manipulability $\implies$ monotonicity, then I can 
use the previous theorem (that required monotonicity and unanimity), since 
I have unanimity also here. I go by contradiction: monotonicity would imply 
that the winner is $a$ with both the rankings $\succ$ and $supset$, since it is 
always better than $c$.

%\noindent \textbf{Remark:} with "minimal", I mean that only one player is 
%changing his preferences list, while all the others remain fixed.

\noindent \textbf{Claim:} by only claiming that the change is minimal, I want 
to prove that, coming back to the previous situation, I get again $a$ as the 
winner.

\end{document}

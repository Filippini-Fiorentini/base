\documentclass[pt11,a4paper,twoside,reqno,openright]{paper}
\usepackage[latin1]{inputenc}
\usepackage{amsmath}
\usepackage{amsfonts}
\usepackage{amssymb}
\usepackage[english]{babel}
\usepackage{subfigure}
\usepackage{graphicx}
\usepackage{float}
\usepackage{listings,lstautogobble}
\usepackage[T1]{fontenc}

\begin{document}

\noindent {\huge Shapley value and power indices}

\noindent The core solution is a multifunction from the set of the game to 
real values. Nucleulus is not a multifunction: it attaches 1! value to the 
vectors of all the players.

\noindent The Shapley value is another solution concept; it is a solution 
function, not a multifunction: you only have one distribution of utilities.

\bigskip
\noindent \textbf{Slide 3}

\begin{enumerate}
	\item Efficiency: the sum of the utilities that $\phi$ distributes 
	coincides with $v(N)$.
	
	\item Symmetry: $i,j$ are players s.t. if A is a coalition that does not 
	cointain neither $i$ nor $j$, then, adding $i$ or $j$ to A gives exactly 
	the same result.
	\textbf{Example:} in a Parliament, two parties with exactly the same 
	number of seats.

	\item Suppose that player $i$ is useless: you do not increase (or decrease) 
	the utility of a coalition by adding the player $i$. Then, to that player, 
	you should assign utility (or cost) equal to zero.

	\item Additivity: remember that v and w are vectors and you know how to 
	sum vectors. Problem: we do not actually know what's the meaning of a game 
	which is defined as the sum of two games. We still consider this property 
	because it is useful iot define a solution.
\end{enumerate}

\bigskip
\noindent \textbf{Slide 5}

\noindent If you believe that a function should fullfill all the properties of 
slide 3, then you are obliged to consider $\sigma$, since it is the only 
function that fullfills these properties.

\noindent $v(S \cup \{i\} - v(S))$: I can say that the difference is what 
player $i$ brings as marginal contribution to the coalition $S \cup \{i\}$.

\noindent The Shapley solution of player $i$ is a weighted sum of all the 
marginal contributions of player $i$. 

\noindent What's the meaning of the coefficient? 

\noindent \textbf{Notation:} $s = |S|$: small s is the number of elements in 
the coalition S.

\noindent Suppose that a group of players arrange to meet and suppose that 
the order of arrivals is random. Coefficient: probability that player $i$ 
enters the room and finds there only the players of coalition S. Indeed, the 
number of possible arrival orders is given by all the possible permutations of 
players, i.e. it is $n!$. Player $i$ makes a coalition with S only if all the 
people in S are already there when he enters the room. Of course, the internal 
ordering of players in S does not matter: he will consider all the possible 
permutations of the elements in S. Analogously, also the ordering of all the 
players who will arrive after $i$ does not matter, which means that you 
consider $s!$ and $(n-s-1)!$.

\noindent Consider the example of \textbf{slide 11} for the interpretation of 
the coefficient.

\noindent First column: list of all the possible arrival orderings (example: 
first row: they arrive in order "player1 - player2 - player3").

\noindent In the first row, player1 makes a coalition with no one, therefore 
his marginal contribution is $v(\{1\}) = 0$. Player2 makes a coalition with 
player1, therefore his marginal contribution is $v(\{1,2\}) - v(\{1\}) = 4 - 0$. 
Etc.

\noindent Comment: there are repeated values over each column; this is obvious 
because there are multiple situations in which, for example, player1 is the 
first to arrive. 

\noindent All the coalitions that do not contain player1 are:
\[
	\emptyset,~ \{2\},~ \{3\},~ \{2,3\}
\]
therefore marginal values are:
$v(1) - v(\emptyset)$\\
$v(\{1,2\}) - v(2)$\\
$v(\{1,3\}) - v(3)$\\
$v(N) - v(\{2,3\})$.

\bigskip
\noindent \textbf{Slide 7, 8, 9 - proof of Shapley theorem}

\begin{enumerate}
	\item Efficiency: I write in $\sum_{i=1}^n{\sigma_i(v)}$ the definition 
	of $\sigma_i$ given by the Shapley theorem. $v(N)$ appears in the sum 
	only when $S = N - {i}$, because the internal sum is defined over S s.t. 
	$S \subseteq N - {i}$. 
	We consider any coalition A and we prove that v(A) appears with coefficient 
	zero (because it appears sometimes with positive and sometimes with negative 
	coefficient, therefore, overall, its contribution is zero). 
	We have positive coefficient when $A = S \cup \{i\}$, i.e. when $i$ belongs 
	to A. If $A = S \cup \{i\}$, the caridinality of A, which is a, is 
	given by $s-1$. 
	We have the negative coefficient when player $i$ does not belong to A, i.e. 
	it appears $n-a$ times, exactly when $A = S$.

	\item Symmetry: I should prove that $\sigma_i(v) = \sigma_j(v)$. 
	Remember: these are two different sums: in $\sigma_i$ I consider coalitions 
	that do not contain $i$, but of course they may contain $j$ (and viceversa). 
	I proceed by splitting the sum between the coalitions that do not contain 
	neither $i$ nor $j$ and then I add what I need. Therefore, in the second 
	term of $\sigma_1$, for example, the index is S (which does not contain 
	neither $i$ nor $j$), but the actual coalition is $S \cup \{j\}$. 
	With this trick, the two formulas are exactly the same, I only have to 
	interchange $i$ and $j$. Since the union is symmetric and since in each 
	sum I have the same coefficients, I have exactly that $\sigma_1 = \sigma_2$, 
	which corresponds to symmetry.

	\item Null player property means that $v(S \cup \{i\}) - v(S)$ is always 
	zero, which of course entails that $\sigma_i$ is always zero (I can have 
	whatever coefficient).

	\item Additivity: also here the coefficients do not matter. I only have to 
	apply distributive law (it. raccoglimento a fattor comune).
\end{enumerate}

\noindent What we still did not prove is uniqueness of $\sigma$. Remember the 
unanimity game, which is a basis for the set of games $\mathcal{G}(N)$. It is 
enough to prove that if I consider the game $Cu_A$, where $C$ is a constant 
(remember: a game is a vector and I can always multiply a vector by a 
constant), then the function must be the Shapley value. 
What happens is the following:
\begin{itemize}
	\item players inside A are symmetric (they are all necessary to make 
	the coalition winning)

	\item players outside A are null players, because if A is a winning 
	coalition it has value 1 and this value remains 1 no matter how many 
	players I add.
\end{itemize}
Therefore players inside A must take 0 and players outside A must take all the 
same value. But the game is efficient $\implies$ players outside A must take 
$\frac{C}{n}$.

\bigskip
\noindent \textbf{Slide 10}

\noindent Simple games are the one in which the value of a coalition is either 
0 or 1. In this case, it is easy to compute marginal contributions: they can 
only be given by $1-1$, $0 - 0$ or $1 - 0$. The only important case is the 
last one: to add a player to a winning coalition does not change anything and 
to add a player to a losing coalition does not tell anything interesting if 
the new coalition is still losing. 
The case $0 - 1$ cannot occur: if I add a player to a winning coalition, it 
cannot become a losing coalition.

\noindent Iot compute $\sigma_i$, we must count the number of times in which 
player $i$ is crucial for a coalition.

\bigskip
\noindent \textbf{Slide 12}

\noindent The game: player1 only need 1km truck, player2 needs 2km and player3 
needs 3km. The total cost is $c_3$. How to share this cost among the players? 

\noindent For player2, we have:
$v(2) - v(\emptyset) = c_2$\\
$v(\{1,2\}) - v(1) = c_2 - c_1$\\
$v(\{2,3\}) - v(3) = 0$\\
$v(N) - v(\{1,3\}) = 0$\\
The last two differences are zero because if we have player3 we must build 
3km and therefore the arrival of player2 does not change anything (we do not 
have kms to add only because player2 has joint the coalition). 
Therefore $\sigma_2 = \frac{c_2}{3} - \frac{c_2-c_1}{6}$

\bigskip
\noindent \textbf{Slide 13}

\noindent In simple games, the main goal is to divide coalitions in winning 
and losing ones. The interpretation of a solution is not utility: what is 
important is the fact that, for example, if two parties are strong enough to 
get the majority, they have a lot of power. The game is not expressed in terms 
of utility. Shapley values express the fraction of power owned by every player, 
rather than their utility.

\noindent \textbf{Example:} consider two different situations. You have 
four players and each one of them owns a portion of the stocks of a 
company. In the first case, we have that their percentages are:\\
$\{10, 20, 30, 40\}$\\
while in the second case we have: \\
$\{10, 21, 30, 39\}$.

\noindent The two situations appears to be almost the same, but in fact they are 
not: in the first case, sometimes player1 is crucial iot build a winning 
coalition, since a coalition is winning iff it owns more than $50\%$. 
In the second case, actually, player1 is the null player: all the other 
can make winning coalitions without him.

\noindent On the other hand, a third situation\\
$\{11, 20, 30, 39\}$\\
turns out to be almost equal to the first one. The point are not numbers, but 
the real meaning of those numbers.

\noindent \textbf{Remark:} when I'm measuring the power of the players, what 
actually means is the ratio: it is the fact, for example, that my fraction 
of power is twice the yours. In this context, efficiency is not a 
requirement I must have.

\bigskip
\noindent \textbf{Slide 14}

\noindent Remember: $s$ is the cardinality of coalition $S$. What I am saying 
is that the probability to make a coalition with coalition A is equal to 
the probability to make a coalition with B, if both A and B have the same 
cardinality. This assumption is not reasonable, for example, in a parliament, 
when I do not make coalitions with some parties only because they have the 
right number of seats {\huge ma certo XD}.

\bigskip
\noindent \textbf{Slide 15}

\noindent Banzhaf value: it says that the number of players in a coalition 
does not matter when I have to decide to which coalition I want to belong. 
Every coalition is equally likely. This is sometimes reasonable, when I 
know that players aggregate no matter the number they are.

\noindent I use dictatorial value when I have the feeling that people might 
aggregate in coalitions, but this is unlikely.

\bigskip
\noindent \textbf{Slide 16}

\noindent All non-veto players are symmetric and all veto players are 
symmetric $\implies$ I only have to compute one Shapley value and then 
apply symmetry and additivity. Calculations should be equal for non-veto 
players. A non-veto player is crucial if he joins a coalition where we already 
have 5 veto players and 3 non-veto players. To be in this situation, I have 
nine players (the remaining non-veto players when I am considering player 
$i$) and I have to take 3.

\noindent On the other hand, a veto player is crucial when I have 4 veto 
players and at least 4 non-veto players.

\noindent The ratio between the Banzhaf of non-veto and the Banzhaf of 
veto players is almost 10, while for Shapley it is almost 100 $\implies$ 
there is a factor of 10 among them.

\noindent Properties of semivalues: if you consider Shapley, they fullfill 
all the properties of Shapley but for efficiency.

\end{document}
